% ------------------------------------------------------------------------------------------------------------------
% Basic configuration and packages
% ------------------------------------------------------------------------------------------------------------------
% Package for discovering wrong and outdated usage of LaTeX.
% More information to be found in l2tabu English version.
\RequirePackage[l2tabu, orthodox]{nag}
% Class of LaTeX document: {size of paper, size of font}[document class]
\documentclass[a4paper,11pt]{article}



% ------------------------------------------------------
% Packages not tied to particular normal language
% ------------------------------------------------------
% This package should improved spaces in the text.
\usepackage{microtype}
% Add few important symbols, like text Celcius degree
\usepackage{textcomp}



% ------------------------------------------------------
% Polonization of LaTeX document
% ------------------------------------------------------
% Basic polonization of the text
\usepackage[MeX]{polski}
% Switching on UTF-8 encoding
\usepackage[utf8]{inputenc}
% Adding font Latin Modern
\usepackage{lmodern}
% Package is need for fonts Latin Modern
\usepackage[T1]{fontenc}



% ------------------------------------------------------
% Setting margins
% ------------------------------------------------------
\usepackage[a4paper, total={14cm, 25cm}]{geometry}



% ------------------------------------------------------
% Setting vertical spaces in the text
% ------------------------------------------------------
% Setting space between lines
\renewcommand{\baselinestretch}{1.1}

% Setting space between lines in tables
\renewcommand{\arraystretch}{1.4}



% ------------------------------------------------------
% Packages for scientific papers
% ------------------------------------------------------
% Switching off \lll symbol, that I guess is representing letter ``Ł''.
% It collide with `amsmath' package's command with the same name
\let\lll\undefined
% Basic package from American Mathematical Society (AMS)
\usepackage[intlimits]{amsmath}
% Equations are numbered separately in every section.
\numberwithin{equation}{section}

% Other very useful packages from AMS
\usepackage{amsfonts}
\usepackage{amssymb}
\usepackage{amscd}
\usepackage{amsthm}

% Package with better looking calligraphy fonts
\usepackage{calrsfs}

% Package with better looking greek letters
% Example of use: pi -> \uppi
\usepackage{upgreek}
% Improving look of lambda letter
\let\oldlambda\Lambda
\renewcommand{\lambda}{\uplambda}




% ------------------------------------------------------
% BibLaTeX
% ------------------------------------------------------
% Package biblatex, with biber as its backend, allow us to handle
% bibliography entries that use Unicode symbols outside ASCII.
\usepackage[
language=polish,
backend=biber,
style=alphabetic,
url=false,
eprint=true,
]{biblatex}

\addbibresource{Teoria-dystrybucji-Bibliography.bib}





% ------------------------------------------------------
% Defining new environments (?)
% ------------------------------------------------------
% Defining enviroment ``Wniosek''
\newtheorem{corollary}{Wniosek}
\newtheorem{definition}{Definicja}
\newtheorem{theorem}{Twierdzenie}





% ------------------------------------------------------
% Local packages
% You need to put them in the same directory as .tex file
% ------------------------------------------------------
% Package containing various command useful for working with a text
\usepackage{general-commands}
% Package containing commands and other code useful for working with
% mathematical text
\usepackage{math-commands}

% Package containing commands created for writing about theory of
% distributions
\usepackage{teoria-dystrybucji}





% ------------------------------------------------------
% Package "hyperref"
% They advised to put it on the end of preambule
% ------------------------------------------------------
% It allows you to use hyperlinks in the text
\usepackage{hyperref}










% ------------------------------------------------------------------------------------------------------------------
% Title and author of the text
\title{Teoria dystrybucji \\
  {\Large Błędy i~uwagi}}

\author{Kamil Ziemian, korekta Wojciech Dyba}


% \date{}
% ------------------------------------------------------------------------------------------------------------------










% ####################################################################
\begin{document}
% ####################################################################





% ######################################
% Title of the text
\maketitle
% ######################################





% ######################################
\section{P.~Antosik, J.~Mikusiński, R.~Sikorski \\
  \textit{Theory~of distributions. The~sequential
    approach},
  \parencite{Antosik-Mikusinski-Sikorski-Theory-of-distributions-Pub-1973}}
% ######################################



% ############################
\subsection{Uwagi do~konkretnych stron}

\label{subsec:AMS-Theory-ETC-Uwagi-do-konkrentych-stron}
% ############################


% \rightrightarrows, \leftleftarrows


\noindent
\Str{???}






% ##################
% \CenterBoldFont{Błędy}


% \begin{center}

%   \begin{tabular}{|c|c|c|c|c|}
%     \hline
%     Strona & \multicolumn{2}{c|}{Wiersz} & Jest
%                               & Powinno być \\ \cline{2-3}
%     & Od góry & Od dołu & & \\
%     \hline
%     % & & & & \\
%     % & & & & \\
%     % & & & & \\
%     % & & & & \\
%     % & & & & \\
%     % & & & & \\
%     % & & & & \\
%     % & & & & \\
%     \hline
%   \end{tabular}

% \end{center}

\VerSpaceSix



% ############################










% ######################################
\section{I.M. Gelfand, G.E. Shilov \\
  \textit{Generalized Functions: Volume~I, Properties and~Operations},
  \parencite{Gelfand-Shilov-Generalized-Functions-Vol-I-Pub-1964}}
% ######################################



% ##################
\CenterBoldFont{Błędy}


\begin{center}

  \begin{tabular}{|c|c|c|c|c|}
    \hline
    & \multicolumn{2}{c|}{} & & \\
    Strona & \multicolumn{2}{c|}{Wiersz} & Jest
                              & Powinno być \\ \cline{2-3}
    & Od góry & Od dołu & & \\
    \hline
    12  & \hphantom{0}4 & & $f_{ 0 } = 0$ & $x_{ 0 } = 0$ \\
    28  & 16 & & $r \leq a$ & $r \geq a$ \\
    % & & & & \\
    % & & & & \\
    \hline
  \end{tabular}

\end{center}

\VerSpaceSix



% ############################










% ######################################
\section{Armen H.~Zemanian \\
  \textit{Teoria dystrybucji i~analiza transformat},
  \parencite{Zemanian-Teoria-dystrybucji-ETC-Pub-1969}}
% ######################################



% ############################
\subsection{Uwagi do~konkretnych stron}

\label{sec:Uwagi-do-konkrentych-stron}
% ############################



\noindent
\Str{16} Rozważania o~zamkniętości ze~względu na zbieżność
przestrzeni funkcji próbnych $\Dcal$~są w~mojej ocenie trochę
chaotyczne, spróbuję więc je jakoś rozjaśnić. W~istocie chodzi o~to,
że~definiujemy zbieżność ciągu funkcji próbnych
$\{ \varphi_{ \nu }( t ) \}$, żądając od~niego dwóch własności:
1)~Dla każdego $k$, ciąg $\{ \varphi_{ \nu }^{ ( k ) }( t ) \}$ jest
zbieżny jednostajnie do jakiejś funkcji
$f_{ k }( t ), \; k = 0, 1, \ldots$ \\
2)~Nośniki wszystkich funkcji $\{ \varphi_{ \nu }( t ) \}$ zawarte są
w~wspólnym zbiorze zwartym. \\
Zwróćmy uwagę, że~na~razie nie możemy mówić, iż~ciąg
$\{ \varphi_{ \nu }( t ) \}$ jest zbieżny do funkcji
$f( t ) = f_{ 0 }( t )$, bo~nie wiemy, czy znajomość funkcji $f( t )$
pozwala nam odtworzyć funkcje $f_{ 1 }( t ), f_{ 2 }( t ), \ldots$ W~tym
momencie właściwsze byłoby powiedzenie, że~ciąg
$\{ \varphi_{ \nu }( t ) \}$ jest zbieżny do~rodziny funkcji
$f_{ k }( t )$.

Jednak na mocy twierdzeń z~analizy matematycznej, które można znaleźć
np.~w~książce Schwartza, str.~649--652,~\cite{SchwartzKursAnalizyMatematycznejVolI1979}, przy tych
warunkach funkcja $f( t )$ jest klasy $\Ccal^{ \infty }( \Rbb )$ i~zachodzi
$f^{ ( k ) }( t ) = f_{ k }( t )$. Tym samy pokazaliśmy, że~zbieżność
zdefiniowana wyżej, jest rzeczywiście zbieżnością do~jakiejś funkcji.
Z~drugiego warunku zbieżności ciągu funkcji próbnych wynika,
że~funkcja $f( t )$ również ma~zwarty nośnik i~tym samym należy
do~$\Dcal$.

Tym samym możemy stwierdzić, że~jeśli istnieje funkcja zespolona,
do~której ciąg funkcji próbnych jest zbieżny w~podanym wyżej sensie,
to~funkcja graniczna również jest funkcją próbną.

\VerSpaceFour





\noindent
\Str{19} Fakt, że~funkcja dana wzorem (4) jest klasy
$\Ccal^{ \infty }( \Rbb )$ został, chyba przyjęty w~domyśle. Należy jednak~się
nad nim zatrzymać i~dowieść tego faktu. ???

\VerSpaceFour





\noindent
\Str{20} Aby wywód był pełny należy jeszcze dowieść,
że~$[ \varphi( t ) ]^{ \frac{ 1 }{ n } } = \sqrt[ n ]{ \varphi( t ) }$ jest
klasy $\Ccal^{ \infty }( \Rbb )$.

\VerSpaceFour





\noindent
\Str{21} Warto zatrzymać~się tu na~chwilę, nad faktem który
bardzo słusznie jest stale przypominany w~kontekście teorii
dystrybucji: nie czegoś takiego jak wartość dystrybucji w~punkcie.
Dystrybucja $f$ z~przestrzeni $\Dcal'( \Rbb ??? )$ jest określona
na~funkcjach z~$\Dcal( ??? )$, a~nie na $\Rbb ???$ i~nawet dla dystrybucji
regularnej wartość $f( t )$ w~konkretnym punkcie nie ma zwykle sensu.
Punkt ma bowiem miarę Lebesgue’a równą~0, więc wartość funkcji która
reprezentuje tę~dystrybucję można w~nim dowolnie zmienić.

Jedynie jeśli w~klasie abstrakcji funkcji reprezentujących daną
dystrybucję istnieje jedna funkcja wyróżniona, to~wartość tej funkcji
jest też wartością dystrybucji w~konkretnym punkcie. Jest tak
np.~jeśli jedna z~tych funkcji jest ciągła.

\VerSpaceFour





\noindent
\Str{29} Użycie we wzorze~(3) symbolu
\begin{equation}
  \label{eq:Zemanian-01}
  \lim_{ \epsilon \to 0^{ + } } \int_{ \epsilon }^{ b } t^{ -3 / 2 } \varphi( t ) \, dt,
\end{equation}
jest trochę mylące. Zdaje~się bowiem sugerować, że~całka
$\int_{ 0 }^{ b } t^{ -3 / 2 } \varphi( t ) \, dt$ istniej i~jest równa tej
granicy, jednak całka ta jest rozbieżna jeśli $\varphi( 0 ) \neq 0$.
Poprawniejsze byłoby następujące rozumowanie.

Najpierw rozpatrzmy całkę
\begin{equation}
  \label{eq:Zemanian-02}
  \int_{ \epsilon }^{ b } t^{ -3/2 } \varphi( t ) \, dt =
  \frac{ 2 \varphi( 0 ) }{ \sqrt{ \epsilon } } - \frac{ 2 \varphi( 0 ) }{ \sqrt{ b } }
  + \int_{ \epsilon }^{ b } \frac{ \psi( t ) }{ \sqrt{ t } } \, dt.
\end{equation}
Całka ta jest skończona dla każdego $\varepsilon$ i~powyższy wzór pozwala
nam zidentyfikować źródło rozbieżności w~granicy $\varepsilon \searrow 0$.
Dysponując tą wiedzą, możemy zdefiniować dystrybucję
$\Pf \, t^{ -3 / 2 } \HeavisideFunPlus( t )$ wzorem
\begin{equation}
  \label{eq:Zemanian-03}
  \langle \Pf \, t^{ -3/2 } \HeavisideFunPlus( t ), \varphi( t ) \rangle =
  \lim_{ \epsilon \to 0^{ + } } \left[ \int_{ \epsilon }^{ +\infty } t^{ -3/2 } \varphi( t ) \, dt
    - \frac{ 2 \varphi( 0 ) }{ \sqrt{ \epsilon } } \right].
\end{equation}

\VerSpaceFour





\noindent
\Str{32} W~tym miejscu można bez trudności, i~nawet byłoby to
bardziej naturalne, wprowadzić pseudofunkcję
$\Pf \frac{ \HeavisideFunPlus( -t )}{ t }$.

\VerSpaceFour





\noindent
\Str{33--34} Przyjmijmy najpierw, że~jeżeli dana jest krzywa
$A$, to przez $A( t )$ będziemy oznaczać taką funkcję, że~krzywa
ta~ma~przedstawienie $( t, A( t ) )$. Teraz należy dokonać takiej
zmienny w~linii 2~(od~dołu) na~stronie~34. \\
\Jest Dlatego przesunięta cześć krzywej $B$ \\
\PowinnoByc Dlatego pole pod~krzywą $\varphi( t ) B( t )$ na~przedziale
$\tau \leq t \leq \epsilon$

\VerSpaceFour





\noindent
\Str{36} W~tym miejscu po raz pierwszy chyba w~książce
pojawia~się termin „obszar”. Z~kontekstu wynika, że~należy przez
niego rozumieć dowolny podzbiór $???$.

\VerSpaceFour





\noindent
\Str{41} Obok nazwy \textit{zbiór zer dystrybucji}, proponowałbym
również używać dla tego pojęcia nazwy \textit{zbiór zerowy dystrybucji}.

\VerSpaceFour




\noindent
\Str{42} Logiczniej byłoby zaraz po~paragrafie \S 1.6 umieścić
paragraf \S 1.8. Pojęcie zbioru zerowego i~nośnika dystrybucji nie ma
wielkiego sensu 1.8.1 i~wynikających z~niego konsekwencji. Paragraf \S
1.7 \textit{Kilka operacji na~dystrybucjach} najlepiej byłby umieścić
jako \S 1.6.

\VerSpaceFour





\noindent
\Str{47--51} Przedstawiony tu dowód twierdzenia 1.8.1, które
jest niezmiernie ważne, zawiera wiele luk, które postaram~się
w~dalszych podpunktach uzupełnić, tu zaś zbiorę potrzebne do~tego
informacje.

Zacznijmy od~tego, że~aby nie obciążać Czytelnika nowymi pojęciami,
nie używa pojęcia zbioru zwartego, zamiast tego mówi o~zbiorach
domkniętych i~ograniczonych\footnote{Jak wiadomo z~twierdzenia
  Heinego-Borela, zbiór w~$\Rbb^{ n }$ jest zwarty wtedy i~tylko
  wtedy, gdy jest domknięty i~ograniczony.} w~$\Rbb^{ n }$, ponieważ jednak
jest to niewygodne, będę mówił o~zbiorach zwartych. Będziemy jeszcze
potrzebowali dwóch twierdzeń odnośnie tych zbiorów.





% #############
\begin{theorem}[Walter Rudin, twr. 2.7,
  str.~45,~\cite{RudinAnalizaRzeczywistaIZespolona1998}]
  \label{thm:Zemanian-01}

  Niech $X$ będzie lokalnie zwartą przestrzenią Hausdorffa,
  $K$~zbiorem zwartym, $U$~zbiorem otwartym i~$K \subset U$. Istnieje
  zbiór otwarty o~zwartym domknięciu taki,~że
  \begin{equation}
    \label{eq:Zemanian-04}
    K \subset V \subset \overline{ V } \subset U.
  \end{equation}

\end{theorem}
% #############





Sens tego twierdzenia jest następujący. Jeśli $K \notin \{ \emptyset, X \}$
i~przestrzeń $X$ jest spójna, to wewnątrz zbioru otwartego $U$, można
powiększyć zbiór zwarty $K$ do~zbioru zwartego $\overline{ V }$, przy czym
$K \neq \overline{ V }$. Jest tak dlatego, że~$V$ jest otwarty,
a~w~przestrzeni Hausdorffa zbiór zwarty jest domknięty. Jeśli więc
przestrzeń jest spójna to nie może zajść równość $K = V$, chyba,
że~$K = \emptyset$ lub~$K = X$.

Jeśli przestrzeń nie jest spójna, to~może~się zdarzyć, że~$K$ jest
maksymalną składową spójną i tym samym jest otwarto-domkniętym
zbiorem. Wtedy jak najbardziej może~się zdarzyć, iż~$K = V$.





% #############
\begin{theorem}[Laurent Schwartz,
  str??\cite{SchwartzKursAnalizyMatematycznejVolI1979}]
  \label{thm:Zemanian-02}

  Niech $( X, d )$ będzie przestrzenią metryczną, w~której każdy zbiór
  domknięty i~ograniczony jest zwarty. Jeśli $K$ jest zbiorem zwartym,
  $D$ zbiorem domkniętym i~$K \cup D = \emptyset$, to odległość zbiorów $K$
  i~$D$
  \begin{equation}
    \label{eq:Zem-s01-05}
    d( K, D ) > 0,
  \end{equation}
  do~tego odległość ta przyjmuje w~pewnym punkcie minimum.

\end{theorem}
% #############





\noindent
Schwartz formułuje równoważne założenie twierdzenia, że~każda
kula domknięta jest zwarta. Tą~równoważność łatwo pokazać.

Będę również używał oznaczenia $\supp \varphi$ na~oznaczenie nośnika
funkcji $\varphi( t )$ (ang.~\textit{support}).

\VerSpaceFour





\noindent
\Str{48--49} \textbf{Lemat 1, uzupełnienie.} Na~podstawie twierdzenia
\eqref{thm:Zemanian-01} istniej zbiór $\Psi$ o~żądanych własnościach. Teraz
na~mocy twierdzenia \eqref{thm:Zemanian-02} mamy
$d( \Theta, \complement \Psi ) = d_{ 1 } > 0$, jeśli więc $\alpha_{ 1 } < d_{ 1 }$, to~nośnik funkcji
$\gamma_{ \alpha_{ 1 } }( t - \tau )$ zawiera~się w~$\Psi$ dla~każdego $t \in \Theta$, tym samym
$\varphi( t ) = 1$. Gwarantuje to~też, iż~funkcja podcałkowa jest klasy
$\Ccal^{ \infty }$. Dowiedliśmy więc, że~$\varphi( t ) = 1$ na~$\Theta$.

Jeśli teraz oznaczymy odległość $d( \Psi, \complement \Omega ) = d_{ 2 } > 0$ i~przyjmiemy
$\alpha_{ 2 } < d_{ 2 }$, to dla każdego $t \in \complement \Omega$ nośnik funkcji
$\gamma_{ \alpha_{ 1 } }( t - \tau )$ nie przecina~się z~$\Psi$, więc $\varphi( t ) = 0$.

\VerSpaceFour





\noindent
\Str{49} W~języku polskim znaczniej lepiej od~nazwy
\textit{lokalnie skończonego pokrycia} brzmi określenie \textit{pokrycie
  lokalnie skończone}. Jego też będę dalej używał.

\VerSpaceFour





\noindent
\Str{49} \textit{Lemat 2, uzupełnienie.} Zauważmy, że~każdy na~mocy
założeń, każdy domknięty podzbiór $\Omega_{ k }$ jest zwarty, więc
istnienie zbiorów $\Lcal_{ k }$ o~zadanych własnościach wynika ponownie
z~twierdzenia \eqref{thm:Zemanian-01}.

\VerSpaceFour





\noindent
\Str{50} \textbf{Lemat 4, uzupełnienie.} Niech $A$ będzie zbiorem
ograniczonym w~$\Rbb^{ n }$, czyli zawiera~się on w~kuli $K( 0, r )$. Trzeba
teraz pokazać, że~tylko skończona ilość zbiorów $\Ocal_{ \alpha }$ z~nowego
pokrycia może przecinać~się z~tą kulą. Oznaczymy majorantę średnicy
zbiorów przez $\Diam_{ \Cfrak }$, a~przez $M$ minimalną
odległość punktu prostopadłościanu reprezentowanego układem liczb
$\{ m_{ 1 }, \ldots, m_{ n } \}$ od~punktu $0$, czyli
\begin{equation}
  \label{eq:Zemanian-06}
  M = \left( \sum_{ i = 0 }^{ n } m_{ i } \right)^{ \frac{ 1 }{ 2 } }.
\end{equation}
Zbiór $\Ocal_{ \alpha }$ może przeciąć zadaną kulę, tylko jeśli
$M - \Diam_{ \Cfrak } < r$, czyli mówiąc prościej,
jeśli~prostopadłościan jest na tyle blisko, że~zbiór $\Ocal_{ \alpha }$
może dosięgnąć kuli $K( 0, r )$. Ponieważ istnieje tylko skończona
ilość układów liczb $\{ m_{ 1 }, \ldots, m_{ n } \}$ spełniających tę
nierówność, więc tylko skończoną ilość razy będziemy wybierali
skończoną rodzinę zbiorów $\Ocal_{ \alpha }$ takich, że~jest możliwe
by~$K( 0, r ) \cap \Ocal_{ \alpha } \neq \emptyset$.

\VerSpaceFour





\noindent
\Str{51} Cel przedstawionej tu konstrukcji jest następujący.
Potrzebujemy pokrycia lokalnie skończonego~$\Omega_{ \nu }$, którego
elementy albo nie przecinają~się z~$\Xi$, albo
$\Omega_{ \nu } \subset \Theta$, gdy $\Omega_{ \nu } \cap \Xi \neq \emptyset$.
Rozkład jedności dla tego pokrycia pozwoli pokazać, że~funkcję~$\psi$
można rozłożyć na~funkcje o~nośnikach w~zbiorach otwartych na~których
dystrybucja $f$~jest równa $0$.

Wydaje mi~się, że~do tej konstrukcji zbiór $\Xi_{ 1 }$ jest
wprowadzony niepotrzebnie.

\VerSpaceFour





\noindent
\Str{59}

\VerSpaceFour





\noindent
\Str{62}

\VerSpaceFour





\noindent
\Str{71} Wzór
\begin{equation}
  \label{eq:Zemanian-07}
  f( t ) =
  f_{ c }( t )
  - \sum_{ \nu = -1 }^{ -\infty } \Delta f_{ \nu } \HeavisideFunPlus( t_{ \nu } - t )
  + \sum_{ \nu = 0 }^{ \infty } \Delta f_{ \nu } \HeavisideFunPlus( t - t_{ \nu } ),
\end{equation}
zawiera pewną subtelność, którą warto wyjaśnić. ????

\VerSpaceFour





\noindent
\Str{74} Fakt, że~iloraz różnicowy
\begin{equation}
  \label{eq:Zemanian-08}
  \frac{ \varphi( t - \Delta ) - \varphi( t ) }{ \Delta }
\end{equation}
dąży do $-\varphi^{ ( 1 ) }( t )$ przy $\Delta \to 0$ wynika od~razu z~definicji
pochodnej i~twierdzenia o~składaniu granic\footnote{Udowodnij i~zapisz
  gdzieś to twierdzenie.}, problem jest tylko taki, że~jest to
zbieżność punktowa. Aby zaś skorzystać z~ciągłości dystrybucji
potrzebujemy udowodnić zbieżność jednostajną tego ilorazy
do~$-\varphi^{ ( 1 ) }( t )$ oraz jednostajną zbieżność jego $k$-pochodnej
do~$-\varphi^{ ( k + 1 ) }( t )$. Inaczej mówiąc dla $k = 0, 1, 2, \ldots$
musi zachodzić
\begin{equation}
  \label{eq:Zemanian-09}
  \lim_{ \Delta \to 0 }
  \frac{ \varphi^{ ( k ) }( t - \Delta ) - \varphi^{ ( k ) }( t ) }{ \Delta } =
  -\varphi^{ ( k + 1 ) }( t ),
\end{equation}
gdzie zbieżność jest rozumiana jako zbieżność jednostajna.

Dowód podany przez Zemaniana nie jest jedynym możliwym, ale~jest
bezpośredni rachunkowo, przez co stosunkowo łatwy do~zrozumienia.





% ############################
\subsection{Błędy i~literówki}

\label{sec:Bledy-i-literowski}
% ############################

\VerSpaceThree


\begin{center}

  \begin{tabular}{|c|c|c|c|c|}
    \hline
    Strona & \multicolumn{2}{c|}{Wiersz} & Jest
                              & Powinno być \\ \cline{2-3}
    & Od góry & Od dołu & & \\
    \hline
    12  & & \hphantom{0}6 & wartości & tylko rzeczywiste \\
    14  & \hphantom{0}6 & & całka
    & całka przedstawiająca $\varphi_{ \alpha }'( t )$ \\
    16  & & \hphantom{0}6 & ciągi & funkcje \\
    17  & & \hphantom{0}2 & \textit{przestrzeni} & \textit{funkcjonału} \\
    17  & & \hphantom{0}3 & która ma & które mają \\
    17  & & \hphantom{0}4 & z przestrzeni & na przestrzeni \\
    18  & \hphantom{0}6 & & $\{ \varphi_{ \nu } \}\nu \to \infty$
    & $\{ \varphi_{ \nu } \}$, przy $\nu \to \infty$ \\
    18  & & 10 & dystrybuantę & dystrybucję \\
    21  & \hphantom{0}9 & & $\Dcal$ & $\Dcal'$ \\
    25  & & \hphantom{0}1 & dąży & dąży punktowo \\
    26  & \hphantom{0}5 & & pochodnym od & pochodną \\
    27  & \hphantom{0}7 & & $\delta^{ ( 1 ) }\boldsymbol{ ( t } )$
    & $\delta^{ ( 1 ) }( t )$ \\
    27  & & \hphantom{0}3 & $\varphi^{ ( k \boldsymbol{ ) } }\boldsymbol{ ( t ) } $
           & $\varphi^{ ( k ) }( t )$ \\
    27  & & \hphantom{0}9 & $< f,\, \varphi >$ & $\langle f, \, \varphi \rangle$ \\
    28  & & \hphantom{0}4 & $\delta^{ ( 2 ) }( 0 )$ & $\varphi^{ ( 2 ) }( 0 )$ \\
    28  & & 17 & dążą & jednocześnie dążą \\
    29  & & \hphantom{0}8 & $t = 0$ & $t \neq 0$ \\
    31  & \hphantom{0}6 & & $\varphi( )$ & $\varphi( t )$ \\
    31  & \hphantom{0}8 & & $\Pf \, | t |^{ \beta } \; 1_{ + }( t )$
           & $\Pf \, | t |^{ \beta } \; 1_{ + }( -t )$ \\
    33  & & \hphantom{0}1 & skończoną & nieskończoną \\
    34  & & \hphantom{0}1 & $\varphi( 0 ) \log \epsilon$ & $\varphi( 0 ) \log \epsilon$. \\
    35 & \hphantom{0}1 & & tj.~$\langle B, \varphi \rangle$. Wartość & Wartość \\
    36 & & \hphantom{0}7 & $\partial t_{ 2 }^{ \: k_{ 2 } } \ldots \partial t_{ n }^{ \: k_{ n } }$
           & $\partial t_{ 2 }^{ k_{ 2 } } \ldots \partial t_{ n }^{ k_{ n } }$ \\
    37 & & \hphantom{0}1 & wyboru układu & układu \\
    37 & & 16 & \textit{mamy $\varphi_{ \alpha }( t )$ równe}
           & \textit{$\varphi_{ \alpha }( t )$ jest równa} \\
    38  & \hphantom{0}4 & & $\{ 0, 0, \ldots, 0 ) \}$ & $\{ 0, 0, \ldots, 0 \}$) \\
    41  & & \hphantom{0}2 & dystrybucji & dystrybucji regularnej \\
    45  & & 10 & $\boldsymbol{f}$ & $f$ \\
    50  & 13 & & \textit{jedną} & \textit{stałą} \\
    % 50 & 13 & & \textit{jedną} & \textit{jedną z~nich} \\
    \hline
  \end{tabular}



  \begin{tabular}{|c|c|c|c|c|}
    \hline
    Strona & \multicolumn{2}{c|}{Wiersz} & Jest
                              & Powinno być \\ \cline{2-3}
    & Od góry & Od dołu & & \\
    \hline
    50  & & \hphantom{0}3 & $C$ & $\Cfrak$ \\
    55  & \hphantom{0}9 & & zbieżny w~$\Dcal'$ & zbieżny \\
    59  & \hphantom{0}3 & & $\theta_{ \nu }$ & $\theta$ \\
    64  & \hphantom{0}6 & & $\nu f( \nu^{ n } t )$ & $\nu^{ n } f( \nu t )$ \\
    71 & & \hphantom{0}1 & $\leq M | \nu |^{ -2 }$ & $\leq M | \nu |^{ -1 }$ \\
    71 & & \hphantom{0}4 & gdzie $p$ & gdzie $p = k - 1$, a $g( t )$ \\
    71 & & \hphantom{0}5 & $M$ i~$k$ stałych
    & $M$ rzeczywistego i~$k$ naturalnego \\
    % & & & & \\
    % & & & & \\
    % & & & & \\
    % & & & & \\
    % & & & & \\
    % & & & & \\
    % & & & & \\
    % & & & & \\
    % & & & & \\
    \hline
  \end{tabular}

\end{center}

\VerSpaceSix


\noindent
\StrWierszDol{21}{10} \\
\Jest opisem niewykończonego \\
\PowinnoByc niepełnym opisem pewnego \\
\StrWierszGora{34}{2} \\
\Jest Odpowiednie nachylenie wskazuje \\
\PowinnoByc Odpowiednią funkcje przedstawia \\
\StrWierszDol{38}{4} \\
\Jest choć może nie być sprecyzowana jego wartość \\
\PowinnoByc choć jego wartość może nie być podana jawnie \\
\StrWierszDol{40}{6} \\
\Jest wartości liczbowych \\
\PowinnoByc wartości liczbowej w~punkcie \\
\StrWierszGora{50}{14} \\
\Jest \textit{pokrycie $\Rbb^{ n }$.} \\
\PowinnoByc \textit{pokrycie $\Rbb^{ n }$, przy czym średnice wszystkich
  jego zbiorów, są~ograniczone przez tą samą stałą, co~dla rodziny
  $\Cfrak_{ \HorSpaceEight T }$.} \\
\StrWierszDol{71}{5} \\
\Jest \textit{$M$ i $k$ stałych rzeczywistych} \\
\PowinnoByc \textit{$M > 0$ i~$k$ naturalnego} \\
\StrWierszDol{71}{4} \\
\Jest \textit{gdzie $p$} \\
\PowinnoByc \textit{gdzie $p = k + 2$, a~$g( t )$} \\



% ############################




% ############################
\newpage

\section{Armen H.~Zemanian \textit{Teoria dystrybucji i~analiza transformat},
  \parencite{Zemanian-Teoria-dystrybucji-ETC-Pub-1969}}


% ##################
\CenterBoldFont{Uwagi do konkretnych stron}


\noindent
\Str{16} Rozważania o~zamkniętości ze~względu na zbieżność
przestrzeni funkcji próbnych $\Dcal$~są w~mojej ocenie trochę
chaotyczne, spróbuję więc je jakoś rozjaśnić. W~istocie chodzi o~to,
że~definiujemy zbieżność ciągu funkcji próbnych
$\{ \varphi_{ \nu }( t ) \}$, żądając od~niego dwóch własności:
1)~Dla każdego $k$, ciąg $\{ \varphi_{ \nu }^{ ( k ) }( t ) \}$ jest
zbieżny jednostajnie do jakiejś funkcji
$f_{ k }( t ), \; k = 0, 1, \ldots$ \\
2)~Nośniki wszystkich funkcji $\{ \varphi_{ \nu }( t ) \}$ zawarte są
w~wspólnym zbiorze zwartym. \\
Zwróćmy uwagę, że~na~razie nie możemy mówić, iż~ciąg
$\{ \varphi_{ \nu }( t ) \}$ jest zbieżny do funkcji
$f( t ) = f_{ 0 }( t )$, bo~nie wiemy, czy znajomość funkcji $f( t )$
pozwala nam odtworzyć funkcje $f_{ 1 }( t ), f_{ 2 }( t ), \ldots$ W~tym
momencie właściwsze byłoby powiedzenie, że~ciąg
$\{ \varphi_{ \nu }( t ) \}$ jest zbieżny do~rodziny funkcji
$f_{ k }( t )$.

Jednak na mocy twierdzeń z~analizy matematycznej, które można znaleźć
np.~w~książce Schwartza, str.~649--652,~\cite{SchwartzKursAnalizyMatematycznejVolI1979}, przy tych
warunkach funkcja $f( t )$ jest klasy $\Ccal^{ \infty }( \Rbb )$ i~zachodzi
$f^{ ( k ) }( t ) = f_{ k }( t )$. Tym samy pokazaliśmy, że~zbieżność
zdefiniowana wyżej, jest rzeczywiście zbieżnością do~jakiejś funkcji.
Z~drugiego warunku zbieżności ciągu funkcji próbnych wynika,
że~funkcja $f( t )$ również ma~zwarty nośnik i~tym samym należy
do~$\Dcal$.

Tym samym możemy stwierdzić, że~jeśli istnieje funkcja zespolona,
do~której ciąg funkcji próbnych jest zbieżny w~podanym wyżej sensie,
to~funkcja graniczna również jest funkcją próbną.

\VerSpaceFour





\noindent
\Str{19} Fakt, że~funkcja dana wzorem (4) jest klasy
$\Ccal^{ \infty }( \Rbb )$ został, chyba przyjęty w~domyśle. Należy jednak~się
nad nim zatrzymać i~dowieść tego faktu. ???

\VerSpaceFour





\noindent
\Str{20} Aby wywód był pełny należy jeszcze dowieść,
że~$[ \varphi( t ) ]^{ \frac{ 1 }{ n } } = \sqrt[ n ]{ \varphi( t ) }$ jest
klasy $\Ccal^{ \infty }( \Rbb )$.

\VerSpaceFour





\noindent
\Str{21} Warto zatrzymać~się tu na~chwilę, nad faktem który
bardzo słusznie jest stale przypominany w~kontekście teorii
dystrybucji: nie czegoś takiego jak wartość dystrybucji w~punkcie.
Dystrybucja $f$ z~przestrzeni $\Dcal'( \Rbb ??? )$ jest określona
na~funkcjach z~$\Dcal( ??? )$, a~nie na $\Rbb ???$ i~nawet dla dystrybucji
regularnej wartość $f( t )$ w~konkretnym punkcie nie ma zwykle sensu.
Punkt ma bowiem miarę Lebesgue’a równą~0, więc wartość funkcji która
reprezentuje tę~dystrybucję można w~nim dowolnie zmienić.

Jedynie jeśli w~klasie abstrakcji funkcji reprezentujących daną
dystrybucję istnieje jedna funkcja wyróżniona, to~wartość tej funkcji
jest też wartością dystrybucji w~konkretnym punkcie. Jest tak
np.~jeśli jedna z~tych funkcji jest ciągła.

\VerSpaceFour





\noindent
\Str{29} Użycie we wzorze~(3) symbolu
\begin{equation}
  \label{eq:Zemanian-01}
  \lim_{ \epsilon \to 0^{ + } } \int_{ \epsilon }^{ b } t^{ -3 / 2 } \varphi( t ) \, dt,
\end{equation}
jest trochę mylące. Zdaje~się bowiem sugerować, że~całka
$\int_{ 0 }^{ b } t^{ -3 / 2 } \varphi( t ) \, dt$ istniej i~jest równa tej
granicy, jednak całka ta jest rozbieżna jeśli $\varphi( 0 ) \neq 0$.
Poprawniejsze byłoby następujące rozumowanie.

Najpierw rozpatrzmy całkę
\begin{equation}
  \label{eq:Zemanian-02}
  \int_{ \epsilon }^{ b } t^{ -3/2 } \varphi( t ) \, dt =
  \frac{ 2 \varphi( 0 ) }{ \sqrt{ \epsilon } } - \frac{ 2 \varphi( 0 ) }{ \sqrt{ b } }
  + \int_{ \epsilon }^{ b } \frac{ \psi( t ) }{ \sqrt{ t } } \, dt.
\end{equation}
Całka ta jest skończona dla każdego $\varepsilon$ i~powyższy wzór pozwala
nam zidentyfikować źródło rozbieżności w~granicy $\varepsilon \searrow 0$.
Dysponując tą wiedzą, możemy zdefiniować dystrybucję
$\Pf \, t^{ -3 / 2 } \HeavisideFunPlus( t )$ wzorem
\begin{equation}
  \label{eq:Zemanian-03}
  \langle \Pf \, t^{ -3/2 } \HeavisideFunPlus( t ), \varphi( t ) \rangle =
  \lim_{ \epsilon \to 0^{ + } } \left[ \int_{ \epsilon }^{ +\infty } t^{ -3/2 } \varphi( t ) \, dt
    - \frac{ 2 \varphi( 0 ) }{ \sqrt{ \epsilon } } \right].
\end{equation}

\VerSpaceFour





\noindent
\Str{32} W~tym miejscu można bez trudności, i~nawet byłoby to
bardziej naturalne, wprowadzić pseudofunkcję
$\Pf \frac{ \HeavisideFunPlus( -t )}{ t }$.

\VerSpaceFour





\noindent
\Str{33--34} Przyjmijmy najpierw, że~jeżeli dana jest krzywa
$A$, to przez $A( t )$ będziemy oznaczać taką funkcję, że~krzywa
ta~ma~przedstawienie $( t, A( t ) )$. Teraz należy dokonać takiej
zmienny w~linii 2~(od~dołu) na~stronie~34. \\
\Jest Dlatego przesunięta cześć krzywej $B$ \\
\PowinnoByc Dlatego pole pod~krzywą $\varphi( t ) B( t )$ na~przedziale
$\tau \leq t \leq \epsilon$

\VerSpaceFour





\noindent
\Str{36} W~tym miejscu po raz pierwszy chyba w~książce
pojawia~się termin „obszar”. Z~kontekstu wynika, że~należy przez
niego rozumieć dowolny podzbiór $???$.

\VerSpaceFour





\noindent
\Str{41} Obok nazwy \textit{zbiór zer dystrybucji}, proponowałbym
również używać dla tego pojęcia nazwy \textit{zbiór zerowy dystrybucji}.

\VerSpaceFour




\noindent
\Str{42} Logiczniej byłoby zaraz po~paragrafie \S 1.6 umieścić
paragraf \S 1.8. Pojęcie zbioru zerowego i~nośnika dystrybucji nie ma
wielkiego sensu 1.8.1 i~wynikających z~niego konsekwencji. Paragraf \S
1.7 \textit{Kilka operacji na~dystrybucjach} najlepiej byłby umieścić
jako \S 1.6.

\VerSpaceFour





\noindent
\Str{47--51} Przedstawiony tu dowód twierdzenia 1.8.1, które
jest niezmiernie ważne, zawiera wiele luk, które postaram~się
w~dalszych podpunktach uzupełnić, tu zaś zbiorę potrzebne do~tego
informacje.

Zacznijmy od~tego, że~aby nie obciążać Czytelnika nowymi pojęciami,
nie używa pojęcia zbioru zwartego, zamiast tego mówi o~zbiorach
domkniętych i~ograniczonych\footnote{Jak wiadomo z~twierdzenia
  Heinego-Borela, zbiór w~$\Rbb^{ n }$ jest zwarty wtedy i~tylko
  wtedy, gdy jest domknięty i~ograniczony.} w~$\Rbb^{ n }$, ponieważ jednak
jest to niewygodne, będę mówił o~zbiorach zwartych. Będziemy jeszcze
potrzebowali dwóch twierdzeń odnośnie tych zbiorów.





% #############
\begin{theorem}[Walter Rudin, twr. 2.7,
  str.~45,~\cite{RudinAnalizaRzeczywistaIZespolona1998}]
  \label{thm:Zemanian-01}

  Niech $X$ będzie lokalnie zwartą przestrzenią Hausdorffa,
  $K$~zbiorem zwartym, $U$~zbiorem otwartym i~$K \subset U$. Istnieje
  zbiór otwarty o~zwartym domknięciu taki,~że
  \begin{equation}
    \label{eq:Zemanian-04}
    K \subset V \subset \overline{ V } \subset U.
  \end{equation}

\end{theorem}
% #############





Sens tego twierdzenia jest następujący. Jeśli $K \notin \{ \emptyset, X \}$
i~przestrzeń $X$ jest spójna, to wewnątrz zbioru otwartego $U$, można
powiększyć zbiór zwarty $K$ do~zbioru zwartego $\overline{ V }$, przy czym
$K \neq \overline{ V }$. Jest tak dlatego, że~$V$ jest otwarty,
a~w~przestrzeni Hausdorffa zbiór zwarty jest domknięty. Jeśli więc
przestrzeń jest spójna to nie może zajść równość $K = V$, chyba,
że~$K = \emptyset$ lub~$K = X$.

Jeśli przestrzeń nie jest spójna, to~może~się zdarzyć, że~$K$ jest
maksymalną składową spójną i tym samym jest otwarto-domkniętym
zbiorem. Wtedy jak najbardziej może~się zdarzyć, iż~$K = V$.





% #############
\begin{theorem}[Laurent Schwartz,
  str??\cite{SchwartzKursAnalizyMatematycznejVolI1979}]
  \label{thm:Zemanian-02}

  Niech $( X, d )$ będzie przestrzenią metryczną, w~której każdy zbiór
  domknięty i~ograniczony jest zwarty. Jeśli $K$ jest zbiorem zwartym,
  $D$ zbiorem domkniętym i~$K \cup D = \emptyset$, to odległość zbiorów $K$
  i~$D$
  \begin{equation}
    \label{eq:Zem-s01-05}
    d( K, D ) > 0,
  \end{equation}
  do~tego odległość ta przyjmuje w~pewnym punkcie minimum.

\end{theorem}
% #############





\noindent
Schwartz formułuje równoważne założenie twierdzenia, że~każda
kula domknięta jest zwarta. Tą~równoważność łatwo pokazać.

Będę również używał oznaczenia $\supp \varphi$ na~oznaczenie nośnika
funkcji $\varphi( t )$ (ang.~\textit{support}).

\VerSpaceFour





\noindent
\Str{48--49} \textbf{Lemat 1, uzupełnienie.} Na~podstawie twierdzenia
\eqref{thm:Zemanian-01} istniej zbiór $\Psi$ o~żądanych własnościach. Teraz
na~mocy twierdzenia \eqref{thm:Zemanian-02} mamy
$d( \Theta, \complement \Psi ) = d_{ 1 } > 0$, jeśli więc $\alpha_{ 1 } < d_{ 1 }$, to~nośnik funkcji
$\gamma_{ \alpha_{ 1 } }( t - \tau )$ zawiera~się w~$\Psi$ dla~każdego $t \in \Theta$, tym samym
$\varphi( t ) = 1$. Gwarantuje to~też, iż~funkcja podcałkowa jest klasy
$\Ccal^{ \infty }$. Dowiedliśmy więc, że~$\varphi( t ) = 1$ na~$\Theta$.

Jeśli teraz oznaczymy odległość $d( \Psi, \complement \Omega ) = d_{ 2 } > 0$ i~przyjmiemy
$\alpha_{ 2 } < d_{ 2 }$, to dla każdego $t \in \complement \Omega$ nośnik funkcji
$\gamma_{ \alpha_{ 1 } }( t - \tau )$ nie przecina~się z~$\Psi$, więc $\varphi( t ) = 0$.

\VerSpaceFour





\noindent
\Str{49} W~języku polskim znaczniej lepiej od~nazwy
\textit{lokalnie skończonego pokrycia} brzmi określenie \textit{pokrycie
  lokalnie skończone}. Jego też będę dalej używał.

\VerSpaceFour





\noindent
\Str{49} \textit{Lemat 2, uzupełnienie.} Zauważmy, że~każdy na~mocy
założeń, każdy domknięty podzbiór $\Omega_{ k }$ jest zwarty, więc
istnienie zbiorów $\Lcal_{ k }$ o~zadanych własnościach wynika ponownie
z~twierdzenia \eqref{thm:Zemanian-01}.

\VerSpaceFour





\noindent
\Str{50} \textbf{Lemat 4, uzupełnienie.} Niech $A$ będzie zbiorem
ograniczonym w~$\Rbb^{ n }$, czyli zawiera~się on w~kuli $K( 0, r )$. Trzeba
teraz pokazać, że~tylko skończona ilość zbiorów $\Ocal_{ \alpha }$ z~nowego
pokrycia może przecinać~się z~tą kulą. Oznaczymy majorantę średnicy
zbiorów przez $\Diam_{ \Cfrak }$, a~przez $M$ minimalną
odległość punktu prostopadłościanu reprezentowanego układem liczb
$\{ m_{ 1 }, \ldots, m_{ n } \}$ od~punktu $0$, czyli
\begin{equation}
  \label{eq:Zemanian-06}
  M = \left( \sum_{ i = 0 }^{ n } m_{ i } \right)^{ \frac{ 1 }{ 2 } }.
\end{equation}
Zbiór $\Ocal_{ \alpha }$ może przeciąć zadaną kulę, tylko jeśli
$M - \Diam_{ \Cfrak } < r$, czyli mówiąc prościej,
jeśli~prostopadłościan jest na tyle blisko, że~zbiór $\Ocal_{ \alpha }$
może dosięgnąć kuli $K( 0, r )$. Ponieważ istnieje tylko skończona
ilość układów liczb $\{ m_{ 1 }, \ldots, m_{ n } \}$ spełniających tę
nierówność, więc tylko skończoną ilość razy będziemy wybierali
skończoną rodzinę zbiorów $\Ocal_{ \alpha }$ takich, że~jest możliwe
by~$K( 0, r ) \cap \Ocal_{ \alpha } \neq \emptyset$.

\VerSpaceFour





\noindent
\Str{51} Cel przedstawionej tu konstrukcji jest następujący.
Potrzebujemy pokrycia lokalnie skończonego~$\Omega_{ \nu }$, którego
elementy albo nie przecinają~się z~$\Xi$, albo
$\Omega_{ \nu } \subset \Theta$, gdy $\Omega_{ \nu } \cap \Xi \neq \emptyset$.
Rozkład jedności dla tego pokrycia pozwoli pokazać, że~funkcję~$\psi$
można rozłożyć na~funkcje o~nośnikach w~zbiorach otwartych na~których
dystrybucja $f$~jest równa $0$.

Wydaje mi~się, że~do tej konstrukcji zbiór $\Xi_{ 1 }$ jest
wprowadzony niepotrzebnie.

\VerSpaceFour





\noindent
\Str{59}

\VerSpaceFour





\noindent
\Str{62}

\VerSpaceFour





\noindent
\Str{71} Wzór
\begin{equation}
  \label{eq:Zemanian-07}
  f( t ) =
  f_{ c }( t )
  - \sum_{ \nu = -1 }^{ -\infty } \Delta f_{ \nu } \HeavisideFunPlus( t_{ \nu } - t )
  + \sum_{ \nu = 0 }^{ \infty } \Delta f_{ \nu } \HeavisideFunPlus( t - t_{ \nu } ),
\end{equation}
zawiera pewną subtelność, którą warto wyjaśnić. ????

\VerSpaceFour





\noindent
\Str{74} Fakt, że~iloraz różnicowy
\begin{equation}
  \label{eq:Zemanian-08}
  \frac{ \varphi( t - \Delta ) - \varphi( t ) }{ \Delta }
\end{equation}
dąży do $-\varphi^{ ( 1 ) }( t )$ przy $\Delta \to 0$ wynika od~razu z~definicji
pochodnej i~twierdzenia o~składaniu granic\footnote{Udowodnij i~zapisz
  gdzieś to twierdzenie.}, problem jest tylko taki, że~jest to
zbieżność punktowa. Aby zaś skorzystać z~ciągłości dystrybucji
potrzebujemy udowodnić zbieżność jednostajną tego ilorazy
do~$-\varphi^{ ( 1 ) }( t )$ oraz jednostajną zbieżność jego $k$-pochodnej
do~$-\varphi^{ ( k + 1 ) }( t )$. Inaczej mówiąc dla $k = 0, 1, 2, \ldots$
musi zachodzić
\begin{equation}
  \label{eq:Zemanian-09}
  \lim_{ \Delta \to 0 }
  \frac{ \varphi^{ ( k ) }( t - \Delta ) - \varphi^{ ( k ) }( t ) }{ \Delta } =
  -\varphi^{ ( k + 1 ) }( t ),
\end{equation}
gdzie zbieżność jest rozumiana jako zbieżność jednostajna.

Dowód podany przez Zemaniana nie jest jedynym możliwym, ale~jest
bezpośredni rachunkowo, przez co stosunkowo łatwy do~zrozumienia.





% ##################
\newpage

\CenterBoldFont{Błędy}


\begin{center}

  \begin{tabular}{|c|c|c|c|c|}
    \hline
    Strona & \multicolumn{2}{c|}{Wiersz} & Jest
                              & Powinno być \\ \cline{2-3}
    & Od góry & Od dołu & & \\
    \hline
    12  & &  6 & wartości & tylko rzeczywiste \\
    14  &  6 & & całka & całka przedstawiająca $\varphi_{ \alpha }'( t )$ \\
    16  & &  6 & ciągi & funkcje \\
    17  & &  2 & \textit{przestrzeni} & \textit{funkcjonału} \\
    17  & &  3 & która ma & które mają \\
    17  & &  4 & z przestrzeni & na przestrzeni \\
    18  &  6 & & $\{ \varphi_{ \nu } \}\nu \to \infty$
           & $\{ \varphi_{ \nu } \}$, przy $\nu \to \infty$ \\
    18  & & 10 & dystrybuantę & dystrybucję \\
    21  &  9 & & $\Dcal$ & $\Dcal'$ \\
    25  & &  1 & dąży & dąży punktowo \\
    26  &  5 & & pochodnym od & pochodną \\
    27  &  7 & & $\delta^{ ( 1 ) }\boldsymbol{ ( t } )$ & $\delta^{ ( 1 ) }( t )$ \\
    27  & &  3 & $\varphi^{ ( k \boldsymbol{ ) } }\boldsymbol{ ( t ) } $
           & $\varphi^{ ( k ) }( t )$ \\
    27  & &  9 & $< f,\, \varphi >$ & $\langle f, \, \varphi \rangle$ \\
    28  & &  4 & $\delta^{ ( 2 ) }( 0 )$ & $\varphi^{ ( 2 ) }( 0 )$ \\
    28  & & 17 & dążą & jednocześnie dążą \\
    29  & &  8 & $t = 0$ & $t \neq 0$ \\
    31  &  6 & & $\varphi( )$ & $\varphi( t )$ \\
    31  &  8 & & $\Pf \, | t |^{ \beta } \; 1_{ + }( t )$
           & $\Pf \, | t |^{ \beta } \; 1_{ + }( -t )$ \\
    33  & &  1 & skończoną & nieskończoną \\
    34  & &  1 & $\varphi( 0 ) \log \epsilon$ & $\varphi( 0 ) \log \epsilon$. \\
    35 & 1 & & tj.~$\langle B, \varphi \rangle$. Wartość & Wartość \\
    36 & & 7 & $\partial t_{ 2 }^{ \: k_{ 2 } } \ldots \partial t_{ n }^{ \: k_{ n } }$
           & $\partial t_{ 2 }^{ k_{ 2 } } \ldots \partial t_{ n }^{ k_{ n } }$ \\
    37 & & 1 & wyboru układu & układu \\
    37 & & 16 & \textit{mamy $\varphi_{ \alpha }( t )$ równe}
           & \textit{$\varphi_{ \alpha }( t )$ jest równa} \\
    38  &  4 & & $\{ 0, 0, \ldots, 0 ) \}$ & $\{ 0, 0, \ldots, 0 \}$) \\
    41  & &  2 & dystrybucji & dystrybucji regularnej \\
    45  & & 10 & $\boldsymbol{f}$ & $f$ \\
    50  & 13 & & \textit{jedną} & \textit{stałą} \\
    % 50 & 13 & & \textit{jedną} & \textit{jedną z~nich} \\
    50  & &  3 & $C$ & $\Cfrak$ \\
    55  &  9 & & zbieżny w~$\Dcal'$ & zbieżny \\
    59  &  3 & & $\theta_{ \nu }$ & $\theta$ \\
    64  &  6 & & $\nu f( \nu^{ n } t )$ & $\nu^{ n } f( \nu t )$ \\
    71 & & 1 & $\leq M | \nu |^{ -2 }$ & $\leq M | \nu |^{ -1 }$ \\
    71 & & 4 & gdzie $p$ & gdzie $p = k - 1$, a $g( t )$ \\
    71 & & 5 & $M$ i~$k$ stałych & $M$ rzeczywistego i~$k$ naturalnego \\
    % & & & & \\
    % & & & & \\
    % & & & & \\
    % & & & & \\
    % & & & & \\
    % & & & & \\
    % & & & & \\
    % & & & & \\
    % & & & & \\
    \hline
  \end{tabular}

\end{center}

\VerSpaceSix


\noindent
\StrWierszDol{21}{10} \\
\Jest opisem niewykończonego \\
\PowinnoByc niepełnym opisem pewnego \\
\StrWierszGora{34}{2} \\
\Jest Odpowiednie nachylenie wskazuje \\
\PowinnoByc Odpowiednią funkcje przedstawia \\
\StrWierszDol{38}{4} \\
\Jest choć może nie być sprecyzowana jego wartość \\
\PowinnoByc choć jego wartość może nie być podana jawnie \\
\StrWierszDol{40}{6} \\
\Jest wartości liczbowych \\
\PowinnoByc wartości liczbowej w~punkcie \\
\StrWierszGora{50}{14} \\
\Jest \textit{pokrycie $\Rbb^{ n }$.} \\
\PowinnoByc \textit{pokrycie $\Rbb^{ n }$, przy czym średnice wszystkich
  jego zbiorów, są~ograniczone przez tą samą stałą, co~dla rodziny
  $\Cfrak_{ T }$.} \\
\StrWierszDol{71}{5} \\
\Jest \textit{$M$ i $k$ stałych rzeczywistych} \\
\PowinnoByc \textit{$M > 0$ i~$k$ naturalnego} \\
\StrWierszDol{71}{4} \\
\Jest \textit{gdzie $p$} \\
\PowinnoByc \textit{gdzie $p = k + 2$, a~$g( t )$} \\



% ############################






% ######################################
\section{Algebry operatorów i~algebry topologiczne}
% ######################################













% ############################
\newpage

{ % Autor i tytuł dzieła
  Masamichi Takesaki \\
  \textit{Theory of Operator Algebras. Volume I}, cite\{MTTOAI\}}



% ##################
\CenterBoldFont{Błędy}


\begin{center}

  \begin{tabular}{|c|c|c|c|c|}
    \hline
    & \multicolumn{2}{c|}{} & & \\
    Strona & \multicolumn{2}{c|}{Wiersz}& Jest
                              & Powinno być \\ \cline{2-3}
    & Od góry & Od dołu & & \\
    \hline
    5 & & 6 & $x_{ 0 }^{ -1 } + \sum\limits_{ n = 0 }^{ \infty }
              [ x_{ 0 }^{ -1 } ( x_{ 0 } - x ) ]^{ n } x_{ 0 }^{ -1 }$
           & $\sum\limits_{ n = 0 }^{ \infty } [ x_{ 0 }^{ -1 } ( x_{ 0 } - x ) ]^{ n }
             x_{ 0 }^{ -1 }$ \\
    7 & & 9 & $[ ( 1 / \lambda ) x - 1 ]^{ -1 }$
           & $[ 1 - ( 1 / \lambda ) x ]^{ -1 }$ \\
    7 & & 8 & $\frac{ 1 }{ \lambda } \big( \frac{ 1 }{ \lambda } - x \big)^{ -1 }$
           & $-\frac{ 1 }{ \lambda } \big( 1 - \frac{ 1 }{ \lambda } x \big)^{ -1 }$ \\
    7 & & 3 & $\sum\limits_{ n = 0 }^{ \infty } ( 1 / \lambda^{ n + 1 } ) x^{ n }$
           & $-\sum\limits_{ n = 0 }^{ \infty } ( 1 / \lambda^{ n + 1 } ) x^{ n }$ \\
    8 & 3 & & $( \lambda - \lambda_{ 0 } )^{ n } f( \lambda_{ 0 } )^{ n + 1 }$
           & $-( \lambda_{ 0 } - \lambda )^{ n } f( \lambda_{ 0 } )^{ n + 1 }$ \\
    & & & & \\
    \hline
  \end{tabular}

\end{center}

\VerSpaceSix



% ############################










% ####################################################################
% ####################################################################
% Bibliography

\printbibliography





% ############################
% End of the document

\end{document}
