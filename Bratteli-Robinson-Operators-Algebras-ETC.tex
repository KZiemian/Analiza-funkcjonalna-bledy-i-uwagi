% ---------------------------------------------------------------------
% Basic configuration and packages
% ---------------------------------------------------------------------
% Package for discovering wrong and outdated usage of LaTeX.
% More information to be found in l2tabu English version.
\RequirePackage[l2tabu, orthodox]{nag}
% Class of LaTeX document: {size of paper, size of font}[document class]
\documentclass[a4paper,11pt]{article}



% ---------------------------------------
% Packages not tied to particular normal language
% ---------------------------------------
% This package should improved spaces in the text.
\usepackage{microtype}
% Add few important symbols, like text Celcius degree
\usepackage{textcomp}



% ---------------------------------------
% Polonization of LaTeX document
% ---------------------------------------
% Basic polonization of the text
\usepackage[MeX]{polski}
% Switching on UTF-8 encoding
\usepackage[utf8]{inputenc}
% Adding font Latin Modern
\usepackage{lmodern}
% Package is need for fonts Latin Modern
\usepackage[T1]{fontenc}



% ---------------------------------------
% Setting margins
% ---------------------------------------
% Package for easy settings of margins. Unit of measurement is inch.
\usepackage{vmargin}
\setmarginsrb
{ 0.7in}  % left margin
{ 0.6in}  % top margin
{ 0.7in}  % right margin
{ 0.8in}  % bottom margin
{  20pt}  % head height
{0.25in}  % head sep
{   9pt}  % foot height
{ 0.3in}  % foot sep



% ---------------------------------------
% Setting vertical spaces in the text
% ---------------------------------------
% Setting space between lines
\renewcommand{\baselinestretch}{1.1}

% Setting space between lines in tables
\renewcommand{\arraystretch}{1.4}



% ---------------------------------------
% Packages for scientific papers
% ---------------------------------------
% Switching off \lll symbol, that I guess is representing letter ``Ł''.
% It collide with `amsmath' package's command with the same name
\let\lll\undefined
% Basic package from American Mathematical Society (AMS)
\usepackage[intlimits]{amsmath}
% Equations are numbered separately in every section.
\numberwithin{equation}{section}

% Other very useful packages from AMS
\usepackage{amsfonts, amssymb, amscd, amsthm}
% Better looking calligraphy fonts
\usepackage{calrsfs}

% Better looking greek letters
% Example of use: pi -> \uppi
\usepackage{upgreek}
% Improving look of lambda letter
\let\oldlambda\Lambda
\renewcommand{\lambda}{\uplambda}





% ---------------------------------------
% Defining new environments (?)
% ---------------------------------------
% Defining enviroment ``Wniosek''
\newtheorem{corollary}{Wniosek}
\newtheorem{definition}{Definicja}
\newtheorem{theorem}{Twierdzenie}





% ------------------------------
% Private packages
% You need to put them in the same directory as .tex file
% ------------------------------
% Contains various command useful for working with a text
\usepackage{latexgeneralcommands}
% Contains definitions useful for working with mathematical text
\usepackage{mathcommands}

% Package for use in text about functional analysis
\usepackage{functionalanalysiscommands}





% ------------------------------
% Package ``hyperref''
% They advised to put it on the end of preambule
% ------------------------------
% It allows you to use hyperlinks in the text
\usepackage{hyperref}











% ---------------------------------------------------------------------
% Defining title and author of the text
\title{Ola Bratteli, Derek W.~Robinson \\
  \textit{Operator Algebras and~Quantum Statistical Mechanics~1} \\
  \textit{$C^{ * }$- and $W^{ * }$-Algebras, Symmetry Groups,} \\
  \textit{Decoposition~of States},
  \cite{BratteliRobinsonOperatorAlgebrasETCVolI2002}}

\author{Kamil Ziemian}


% \date{}
% ---------------------------------------------------------------------










% ####################################################################
\begin{document}
% ####################################################################





% ######################################
% Title of the text
\maketitle
% ######################################





% ######################################
\section{Uwagi ogólne}

\label{sec:Uwagi-ogolne}
% ######################################




\noindent
Podziękowania za pomoc przy zrozumieniu fragmentów książki dla: Kamila
Niczyja, Jackowi Jakimiukowi, Jaceka Krajczoka.

\VerSpaceFour





\noindent
W~notatkach do omawianej książki zawsze będziemy przyjmować, że~zbiór liczb
naturalnych $\Nbb$ zawiera~$0$. Symbolem $\Nbb_{ + }$ będziemy oznaczać
zbiór dodatnich liczb naturalny.

\VerSpaceFour





\noindent
Jeżeli nie powiedziano inaczej, to będziemy przyjmować, że~wszystkie
omawiane w~książce i~w~tych notatkach przestrzenie wektorowe oraz algebry są
określone nad ciałem liczb zespolonych $\Cbb$. Ewentualnie nad ciałem
liczb rzeczywistych $\Rbb$.

\VerSpaceFour







% ######################################
\section{Uwagi do~konkretnych stron}

\label{sec:Uwagi-do-konkrentych-stron}
% ######################################



\noindent
\Str{19} Autorzy podając definicję algebry zapomnieli podać jeden z~warunków
jakie musi spełniać mnożenie wewnętrzne w~tej strukturze algebraicznej. Tym
warunkiem jest liniowość w~lewym argumencie:
\begin{equation}
  \label{eq:Bratteli-Robinson-Operator-Algebras-ETC-Vol-I-s01-01}
  ( A + B ) C = A C + B C.
\end{equation}

\VerSpaceFour





\noindent
\Str{22} W~dowodzie propozycji 2.1.5 czytamy, że~wykazanie nierówności
iloczynowej dla algebry $\UfrakTilde$ jest prosty w~sprawdzeniu, lecz ja nie
znalazłem prostego, elementarnego dowodu tej nierówności. Najbardziej
bezpośredni dowód jaki znalazłem, to ten z~widea
\href{https://www.youtube.com/watch?v=6NC0UWjDcBQ}{\textit{C*-algebras 3:
    Unitization}}, który mówi, żeby
potraktować elementy $\UfrakTilde$ jako operatory liniowe działające
na przestrzeni Banach $\Ufrak$. Przedyskutujemy tą idę teraz dokładniej.

Niech $( \alpha, A ) \in \UfrakTilde$. Definiujemy jego działanie na element
$V \in \Ufrak$ za pomocą zależności
\begin{equation}
  \label{eq:Bratteli-Robinson-Operator-Algebras-ETC-Vol-I-s01-02}
  ( \alpha, A ) V = \alpha V + A \, V.
\end{equation}
Bez problemu wykazuje~się, że jest to operator liniowy. W~tej sytuacji
norma
\begin{equation}
  \label{eq:Bratteli-Robinson-Operator-Algebras-ETC-Vol-I-s01-03}
  \Vert ( \alpha, A ) \Vert =
  \sup_{ \Vert B \Vert = 1 } \Vert \alpha B + A \, B \Vert,
\end{equation}
jest standardową normą operatora liniowego w~przestrzeni
Banacha\footnote{Aby wprowadzić tę normę na przestrzeni operatorów nad
  $\Ufrak$, wystarczy by~przestrzeń ta była unormowana, ale~w~związku
  z~tym, że~i~tak rozpatrujemy tu przestrzenie Banacha, nie będziemy~się
  zagłębiać tak bardzo w~szczegóły.}
$\Ufrak$, tym samym dowód nierówności
\begin{equation}
  \label{eq:Bratteli-Robinson-Operator-Algebras-ETC-Vol-I-s01-04}
  \Vert ( \alpha, A ) ( \beta, B ) \Vert \leq
  \Vert ( \alpha, A ) \Vert \, \Vert ( \beta, B ) \Vert,
\end{equation}
można skopiować z~książek do analizy funkcjonalnej. Warto zadać sobie
pytanie, czy nie dałoby~się tych dowodów zmodyfikować w~taki sposób, by nie
musieć~się uciekać do interpretacji $\UfrakTilde$ jako przestrzeni
operatorów liniowych? Może uda się nam do tego problemu powrócić.

Niezależnie od tego, należy przedyskutować pewne własności algebry
$\UfrakTilde$ i~uzupełnić dowód tej propozycji. Zacznijmy od nierówności
\begin{equation}
  \label{eq:Bratteli-Robinson-Operator-Algebras-ETC-Vol-I-s01-05}
  \Vert ( \alpha, A ) \Vert \leq | \alpha | + \Vert A \Vert.
\end{equation}
Jej dowód jest bardzo prosty, mianowicie
\begin{equation}
  \label{eq:Bratteli-Robinson-Operator-Algebras-ETC-Vol-I-s01-06}
  \Vert ( \alpha, A ) \Vert =
  \sup_{ \Vert B \Vert = 1 } \Vert \alpha B + A \, B \Vert \leq
  \sup_{ \Vert B \Vert = 1 } | \alpha | \, \Vert B \Vert +
  \sup_{ \Vert B \Vert = 1 } \Vert A \Vert \, \Vert B \Vert =
  | \alpha | + \Vert A \Vert.
\end{equation}
Zachodzą ponadto dwie równości

\negVerSpaceFour


\begin{subequations}

  \begin{align}
    \label{eq:Bratteli-Robinson-Operator-Algebras-ETC-Vol-I-s01-07-A}
    \Vert ( \alpha, 0 ) \Vert &= | \alpha |, \\
    \label{eq:Bratteli-Robinson-Operator-Algebras-ETC-Vol-I-s01-07-B}
    \Vert ( 0, A ) \Vert &= \Vert A \Vert.
  \end{align}

\end{subequations}

\noindent
Dowód pierwszej jest ponownie bardzo prosty, wystarczy bowiem zauważyć
\begin{equation}
  \label{eq:Bratteli-Robinson-Operator-Algebras-ETC-Vol-I-s01-08}
  \Vert ( \alpha, 0 ) \Vert =
  \sup_{ \Vert B \Vert = 1 } \Vert \alpha B \Vert =
  | \alpha | \sup_{ \Vert B \Vert = 1 } = | \alpha |.
\end{equation}
Aby dowieść zależności
\eqref{eq:Bratteli-Robinson-Operator-Algebras-ETC-Vol-I-s01-07-B} zacznijmy
od spostrzeżenia, że
\begin{equation}
  \label{eq:Bratteli-Robinson-Operator-Algebras-ETC-Vol-I-s01-09}
  \Vert ( 0, A ) \Vert =
  \sup_{ \Vert B \Vert = 1 } \Vert A \, B \Vert \leq
  \sup_{ \Vert B \Vert = 1 } \Vert A \Vert \, \Vert B \Vert = \Vert A \Vert.
\end{equation}
Mamy więc nierówność $\Vert ( 0, A ) \Vert \leq \Vert A \Vert$, teraz wystarczy pokazać,
że~istnieje taki element $B$ o~normie równej jeden, że~zachodzi
$\Vert A B \Vert = \Vert A \Vert$. Weźmy mianowicie $B = A^{ * } / \Vert A \Vert$. Korzystają
z~własności algebr~$C^{ * }$ dostajemy
\begin{equation}
  \label{eq:Bratteli-Robinson-Operator-Algebras-ETC-Vol-I-s01-10}
  \left\Vert A \, \frac{ A^{ * } }{ \Vert A \Vert } \right\Vert =
  \frac{ 1 }{ \Vert A \Vert } \Vert A \, A^{ * } \Vert =
  \frac{ 1 }{ \Vert A \Vert } \Vert A \Vert^{ 2 } = \Vert A \Vert.
\end{equation}

Z~nierównością
\eqref{eq:Bratteli-Robinson-Operator-Algebras-ETC-Vol-I-s01-05} związany
jest następujące problem, kiedy mianowicie zachodzi w~niej równość?
Wydaje~się bardzo wątpliwe, żeby~dla każdego $( \alpha, A )$ norma zdefiniowana
przez \eqref{eq:Bratteli-Robinson-Operator-Algebras-ETC-Vol-I-s01-03} była
równa $| \alpha | + \Vert A \Vert$. Gdyby tak jednak było, dowód nierówności iloczynowej
nie wymagałby sięgnięcia to teorii operatorów na przestrzeni Banacha, lecz
wystarczyłyby elementarne rachunki. Ponieważ w~tej chwili nie jesteśmy
w~stanie powiedzieć nic o~tym kiedy
w~\eqref{eq:Bratteli-Robinson-Operator-Algebras-ETC-Vol-I-s01-05} można
postawić znak równa~się, nie będziemy~się zagłębiać w~ten temat.

Doprecyzujemy teraz pewne elementy dowodu propozycji 2.1.5, zaczynając
od~wykazania tego, że~warunek $\Vert ( \alpha, A ) \Vert = 0$ pociąga za sobą $\alpha = 0$
i~$A = 0$. Rozpatrzmy najpierw przypadek $\alpha = 0$, mamy wtedy
$\Vert ( 0, A ) \Vert = \Vert A \Vert = 0$, więc $A = 0$. Pozostaje rozpatrzyć przypadek
$\alpha \neq 0$ i~pokazać, że~wówczas musi być $\Vert ( \alpha, A ) \Vert \neq 0$. Ponieważ w~takiej
sytuacji mamy
\begin{equation}
  \label{eq:Bratteli-Robinson-Operator-Algebras-ETC-Vol-I-s01-11}
  \Vert ( \alpha, A ) \Vert = | \alpha | \, \Vert ( 1, \tfrac{ 1 }{ \alpha } A ) \Vert,
\end{equation}
to oznaczając $A / \alpha$ symbolem $A'$, możemy ograniczyć nasze badania
do przypadku $( 1, A' )$. Wygodnie będzie dokonać jeszcze jednej zmiany, to
znaczy, to znaczy przyjąć, że~$A'' = -A'$. Ponieważ duża ilość primów jest
uciążliwa w~czytaniu, operator $A''$ będziemy oznaczać ponownie
symbolem~$A$. W~dalszym ciągu notatek będziemy zwykle unikali jawnego
mówienia, że~nową wielkość będziemy oznaczać tym samym symbolem co starą.

Tym samy problem nasz został zredukowany do pokazania, iż~przyjęcie
$\Vert ( 1, -A ) \Vert = 0$ prowadzi do sprzeczności. Niech $B$ będzie dowolnym
elementem $\Ufrak$, mamy
\begin{equation}
  \label{eq:Bratteli-Robinson-Operator-Algebras-ETC-Vol-I-s01-12}
  ( 1, -A ) ( 0, B ) = B - AB,
\end{equation}
a~ponadto
\begin{equation}
  \label{eq:Bratteli-Robinson-Operator-Algebras-ETC-Vol-I-s01-13}
  \Vert B - A B \Vert =
  \Vert ( 1, -A ) ( 0, B ) \Vert \leq
  \Vert ( 1, -A ) \Vert \, \Vert B \Vert = 0.
\end{equation}
Stąd $B - A B = 0$, czyli $B = A B$. Powtarzając to samo rozumowanie dla
$B^{ * }$ dostajemy $B^{ * } = A B^{ * }$, co po wzięciu inwolucji daje nam
$B = B A^{ * }$ dla każdego $B \in \Ufrak$. W~szczególności zachodzi
$A = A A^{ * } = A^{ * }$, co wraz z~dwoma poprzednimi zależnościami daje nam
\begin{equation}
  \label{eq:Bratteli-Robinson-Operator-Algebras-ETC-Vol-I-s01-14}
  A B = B A = B,
\end{equation}
co oznacza, że~$A$ jest jednością w~algebrze $\Ufrak$ wbrew założeniu.
Otrzymana sprzeczność dowodzi tego, że~$\Vert ( \alpha, A ) \Vert = 0$ implikuje
$\alpha = 0$ i~$A = 0$.

\VerSpaceFour





\noindent
\Str{25} W~tym miejscu warto przypomnieć w~jaki sposób zdefiniowane są
potęgi elementu danej algebry. Zacznijmy od przypadku, gdy $A$ jest
elementem dowolnej algebry~$\Acal$. Potęga $A^{ n }$, gdzie
$n \in \Nbb_{ + }$ jest określona prze dwa proste związki

\negVerSpaceFour


\begin{subequations}

  \begin{align}
    \label{eq:Bratteli-Robinson-Operator-Algebras-ETC-Vol-I-s01-15-A}
    A^{ 1 } &= A, \\
    \label{eq:Bratteli-Robinson-Operator-Algebras-ETC-Vol-I-s01-15-B}
    A^{ n + 1 } &= A ( A^{ n } ).
  \end{align}

\end{subequations}


\noindent
Jeśli algebra $\Acal$ posiada element neutralny $\UnitAlg$ to definiujemy
$A^{ 0 }$ jako
\begin{equation}
  \label{eq:Bratteli-Robinson-Operator-Algebras-ETC-Vol-I-s01-16}
  A^{ 0 } = \UnitAlg.
\end{equation}
Jeśli ponadto $A$ posiada element odwrotny $A_{ \textrm{inv} }$, czyli taki
element, że~zachodzi
\begin{equation}
  \label{eq:Bratteli-Robinson-Operator-Algebras-ETC-Vol-I-s01-17}
  A \, A_{ \textrm{inv} } = A_{ \textrm{inv} } \, A = \UnitAlg,
\end{equation}
to określamy $A^{ -n }$, $n \in \Nbb_{ + }$, wzorem
\begin{equation}
  \label{eq:Bratteli-Robinson-Operator-Algebras-ETC-Vol-I-s01-18}
  A^{ -n } = ( A_{ \textrm{inv} } )^{ n }.
\end{equation}
W~szczególności $A^{ -1 } = A_{ \textrm{inv} }$. Tym samym dla elementu
odwracalnego określiliśmy potęgę $A^{ m }$, dla dowolnego $m \in \Zbb$.

\VerSpaceFour





\noindent
\Str{25} Opiszemy tutaj pewne podstawowe własności jakie posiadają
elementy odwracalne algebry, a~które są wykorzystywane w~różnych miejscach
w~książce.

Przyjmijmy, że~$\Ufrak$ jest algebrą z~jednością i~$A$ jest jej elementem
odwracalnym. Wówczas odwracalny jest dowolny element $\alpha A$, gdzie $\alpha \neq 0$
jest liczbą zespoloną, zachodzi bowiem
\begin{equation}
  \label{eq:Bratteli-Robinson-Operator-Algebras-ETC-Vol-I-s01-19}
  ( \alpha A )^{ -1 } = \frac{ 1 }{ \alpha } A^{ -1 }.
\end{equation}

Niech teraz $A$ i~$B$ będą odwracalnymi elementami algebry, które są
przemienne: $A B = B A$. Wówczas $A$ i~$B^{ -1 }$ również są
przemienne. Dowód tej własności jest bardzo prosty:
\begin{equation}
  \label{eq:Bratteli-Robinson-Operator-Algebras-ETC-Vol-I-s01-20}
  A B^{ -1 } = B^{ -1 } B A B^{ -1 } = B^{ -1 } A B B^{ -1 } = B^{ -1 } A.
\end{equation}
Prostym wnioskiem z~tego, jest to, że~jeśli $A$ i~$B$ są przemienne,
to~przemienne są $A^{ -1 }$ i~$B^{ -1 }$. Z~przemienności $A$ i~$B$ wynika
bowiem przemienność $A$ i~$B^{ -1 }$. Stosując tą samą własność ponownie
dostajemy, że~$A^{ -1 }$ i~$B^{ -1 }$~są przemienne.

Przejdźmy teraz do trudniejszego problemu. Mamy dwa elementy algebry
$A$,~$B$, takie że
\begin{equation}
  \label{eq:Bratteli-Robinson-Operator-Algebras-ETC-Vol-I-s01-21}
  C = A B
\end{equation}
i~$C$ jest odwracalny. Co można powiedzieć o~odwracalności elementów $A$
i~$B$? Jeden prosty przykład dotyczy sytuacji, gdy element $B C^{ -1 }$ jest
przemienny z~elementem~$A$:
\begin{equation}
  \label{eq:Bratteli-Robinson-Operator-Algebras-ETC-Vol-I-s01-22}
  A ( B C^{ -1 } ) = ( B C^{ -1 } ) A.
\end{equation}
Wówczas bowiem możemy pomnożyć
\eqref{eq:Bratteli-Robinson-Operator-Algebras-ETC-Vol-I-s01-21} przez
$C^{ -1 }$ dostając
\begin{equation}
  \label{eq:Bratteli-Robinson-Operator-Algebras-ETC-Vol-I-s01-23}
  \UnitAlg = A ( B C^{ -1 } ).
\end{equation}
Z~przemienności mnożenia $A$ i~$B C^{ -1 }$ dostajemy
\begin{equation}
  \label{eq:Bratteli-Robinson-Operator-Algebras-ETC-Vol-I-s01-24}
  \UnitAlg = A ( B C^{ -1 } ) = ( B C^{ -1 } ) A.
\end{equation}
Czyli $B C^{ -1 }$ jest elementem odwrotnym do~$A$.

W~szczególności jeśli $A B = C$, $C$ jest odwracalny i~wszystkie trzy
elementy są między sobą przemienne, to $A$ i~$B$ są odwracalne. Jak
wykazaliśmy powyżej, wówczas $C^{ -1 }$ jest przemienny z~$A$ i~$B$ mamy więc
\begin{equation}
  \label{eq:Bratteli-Robinson-Operator-Algebras-ETC-Vol-I-s01-25}
  \UnitAlg = A ( B C^{ -1 } ) = B A C^{ -1 } = ( B C^{ -1 } ) A.
\end{equation}
Zamieniając rolami $A$ i~$B$ dostajemy, że~$A C^{ -1 }$ jest elementem
odwrotnym do~$B$. Oprócz tego wiemy, że~$C$ jest zawsze przemienny z~$A B$,
bowiem
\begin{equation}
  \label{eq:Bratteli-Robinson-Operator-Algebras-ETC-Vol-I-s01-26}
  \UnitAlg = ( A B ) C^{ -1 } = C^{ -1 } ( A B ).
\end{equation}
Tym samym $C^{ -1 }$, a~więc też~$C$, jest przemienny z~$A B$. Co~jednak
można powiedzieć o~przemienności $C$ z~$A$ lub $B$?

\VerSpaceFour





\noindent
\Str{25} Dobrze jest zwrócić tu uwagę, na~pewną własność normy elementu
odwrotnego, która zachodzi dla każdej algebry unormowanej. Ponieważ mamy
\begin{equation}
  \label{eq:Bratteli-Robinson-Operator-Algebras-ETC-Vol-I-s01-27}
  1 = \Vert \UnitAlg \Vert = \Vert A \, A^{ -1 } \Vert \leq \Vert A \Vert \, \Vert A^{ -1 } \Vert,
\end{equation}
co prowadzi do nierówności
\begin{equation}
  \label{eq:Bratteli-Robinson-Operator-Algebras-ETC-Vol-I-s01-28}
  \frac{ 1 }{ \Vert A \Vert } \leq \Vert A^{ -1 } \Vert.
\end{equation}

\VerSpaceFour





\noindent
\Str{25--26} Przekształcając równość
\begin{equation}
  \label{eq:Bratteli-Robinson-Operator-Algebras-ETC-Vol-I-s01-29}
  ( \lambda \HorSpaceOne \UnitAlg - A )^{ -1 } =
  \lambda^{ -1 } \sum_{ m \geq 0 } \frac{ A^{ n } }{ \lambda^{ n } }, \qquad
  | \lambda | > \Vert A \Vert,
\end{equation}
można otrzymać ciekawe rezultaty wynik. Zacznijmy od przyjęcia,
że~$\lambda = 1$. Wówczas operator odwrotny do $\UnitAlg - A$, gdzie $\Vert A \Vert < 1$
wyraża~się wzorem
\begin{equation}
  \label{eq:Bratteli-Robinson-Operator-Algebras-ETC-Vol-I-s01-30}
  ( \UnitAlg - A )^{ -1 } = \sum_{ m \geq 0 } A^{ m }.
\end{equation}
Rozpoznajemy tu wzór analogiczny do wyrażeni na sumę szeregu geometrycznego
liczb rzeczywistych:
\begin{equation}
  \label{eq:Bratteli-Robinson-Operator-Algebras-ETC-Vol-I-s01-31}
  \frac{ \UnitAlg }{ \UnitAlg - A } \equiv ( \UnitAlg - A )^{ -1 } =
  \sum_{ m \geq 0 } A^{ m }.
\end{equation}
Używając standardowej metody na szacowanie normy szeregu od~razu dostajemy
\begin{equation}
  \label{eq:Bratteli-Robinson-Operator-Algebras-ETC-Vol-I-s01-32}
  \Vert ( \UnitAlg - A )^{ -1 } \Vert \leq
  \frac{ 1 }{ 1 - \Vert A \Vert }.
\end{equation}

Rozpatrzymy teraz element $B$ taki, że~$\Vert \UnitAlg - B \Vert < 1$. Ponieważ
$B = \UnitAlg - ( \UnitAlg - B )$, a~na~mocy tego co zostało
powiedziane wcześniej operator $\UnitAlg - ( \UnitAlg - B )$ jest
odwracalny, tym samym
\begin{equation}
  \label{eq:Bratteli-Robinson-Operator-Algebras-ETC-Vol-I-s01-33}
  B^{ -1 } =
  \big( \UnitAlg - ( \UnitAlg - B ) \big)^{ -1 } =
  \sum_{ m \geq 0 } ( \UnitAlg - B )^{ m }.
\end{equation}
Oszacowanie \eqref{eq:Bratteli-Robinson-Operator-Algebras-ETC-Vol-I-s01-32}
przyjmuje teraz formę
\begin{equation}
  \label{eq:Bratteli-Robinson-Operator-Algebras-ETC-Vol-I-s01-34}
  \Vert B^{ -1 } \Vert =
  \big( \UnitAlg - ( \UnitAlg - B ) \big)^{ -1 } \leq
  \frac{ 1 }{ 1 - \Vert \UnitAlg - B \Vert }.
\end{equation}

Wróćmy jeszcze na~chwilę do~wyrażenia $( \lambda \UnitAlg - A )^{ -1 }$, gdzie
$| \lambda | > \Vert A \Vert$. Korzystając z~notacji $B = A / \lambda$ możemy zapisać
\begin{equation}
  \label{eq:Bratteli-Robinson-Operator-Algebras-ETC-Vol-I-s01-35}
  ( \lambda \HorSpaceOne \UnitAlg - A )^{ -1 } =
  \lambda^{ -1 } ( \UnitAlg - B )^{ -1 }.
\end{equation}
Teraz $\Vert B \Vert < 1$, więc dostajemy
\begin{equation}
  \label{eq:Bratteli-Robinson-Operator-Algebras-ETC-Vol-I-s01-36}
  ( \lambda \HorSpaceOne \UnitAlg - A )^{ -1 } =
  \lambda^{ -1 } \sum_{ m \geq 0 } B^{ m } =
  \lambda^{ -1 } \sum_{ m \geq 0 } \frac{ A^{ m } }{ \lambda^{ m } }.
\end{equation}
Widzimy więc kolejną analogię między rozważanymi tu wzorami a~szeregiem
geometrycznym liczb rzeczywistych.

\VerSpaceFour





\noindent
\Str{26} W~propozycji 2.2.2 czytamy, że~jeśli zdefiniujemy promień spektralny
jako
\begin{equation}
  \label{eq:Bratteli-Robinson-Operator-Algebras-ETC-Vol-I-s01-37}
  \rho( A ) =
  \sup\big\{ | \lambda |; \, \lambda \in \sigma_{ \Ufrak }( A ) \big\},
\end{equation}
to zachodzi
\begin{equation}
  \label{eq:Bratteli-Robinson-Operator-Algebras-ETC-Vol-I-s01-38}
  \rho( A ) =
  \lim_{ n \to \infty } \Vert A^{ n } \Vert^{ 1 / n }.
\end{equation}
Co więcej granica ta istniej i~na mocy tego widmo jest niepuste. Jeśli
dobrze pamiętam dla zbioru pustego $\sup \emptyset = -\infty$, więc fakt,
że~$\rho( A ) \geq 0$ dowodziłby, że~istotnie widmo nie może być puste. Mam jednak
obawy, czy nie istnieje jakaś luka, między definicją promienia spektralnego
jako supremum po~wartościach bezwzględnych z~widma
\eqref{eq:Bratteli-Robinson-Operator-Algebras-ETC-Vol-I-s01-27},
a~wnioskowaniem z~tego, że~skoro jest dodatni, to widmo musi być niepuste.

Niezależnie od tego, z~dowodu wiemy, że~widmo zawsze jest niepuste. Jeśli
$\rho( A ) > 0$ to musi istnieć $\lambda \in \sigma_{ \Ufrak }( A )$, taki że $\lambda \neq 0$. Jeśli
zaś $\rho( A ) = 0$, to wówczas $0 \in \sigma_{ \Ufrak }( A )$.

\VerSpaceFour





\noindent
\Str{26} W~dowodzie propozycji 2.2.2 jest pokazane, że~jeśli
$0 \in r_{ \Ufrak }( A )$ to $r_{ A } > 0$. Później zostanie udowodnione,
że~$\rho( A ) = r_{ A }$, ale na razie będziemy~się trzymać tych oznaczeń.
Kiedy jednak udowodnimy, że~$r_{ A }$ jest równy promieniowi spektralnemu,
i~że widmo zawsze jest niepuste ta implikacja zyska elegancką interpretację.
Mianowicie, $0 \in r_{ \Ufrak }( A )$ oznacza, że~operator
\begin{equation}
  \label{eq:Bratteli-Robinson-Operator-Algebras-ETC-Vol-I-s01-39}
  0 \, \UnitAlg - A = -A,
\end{equation}
jest odwracalny, a~tym samy odwracalny jest element $-( -A ) = A$. Ponieważ
zaś, jeśli element $A$ jest odwracalny to odwracalne są też wszystkie
elementy $\lambda \UnitAlg - A$, gdzie $| \lambda | < \Vert A^{ -1 } \Vert$. Tym samym w~kole
wyznaczonym przez ostatnią nierówność nie elementów widma, więc promień
spektralny musi być większy od~zera.

Analogicznie możemy wyjaśnić, czemu $r_{ A } = 0$ oznacza,
że~$0 \in \sigma_{ \Ufrak }( A )$. Aby promień spektralny był równy $0$, dla
dowolnego otoczenia punktu $\lambda = 0$ musi istnieć $\lambda_{ 1 }$ taki,
że~element $\lambda_{ 1 } \UnitAlg - A$ jest nieodwracalny. Ale na mocy tego co
zostało powiedziane wyżej, jest to możliwe tylko wtedy, gdy~$A$ sam jest
nieodwracalny.

\VerSpaceFour





\noindent
\Str{26--27} Dowód bardzo ważnej propozycji 2.2.2 można w~wielu miejsca
uczynić w~jaśniejszy sposób, co postaram~się tutaj zrobić. Zacznijmy od
udowodnienia tego, że~z~wzoru
\begin{equation}
  \label{eq:Bratteli-Robinson-Operator-Algebras-ETC-Vol-I-s01-40}
  \rho( A ) = \lim_{ n \to \infty } \Vert A^{ n } \Vert^{ 1 / n },
\end{equation}
wynika, iż~$\rho( A ) \leq \Vert A \Vert$. Z~nierówności iloczynowej dostajemy w~prosty
sposób $\Vert A^{ n } \Vert \leq \Vert A \Vert^{ n }$ i~tym samym
\begin{equation}
  \label{eq:Bratteli-Robinson-Operator-Algebras-ETC-Vol-I-s01-41}
  \rho( A ) =
  \lim_{ n \to \infty } \Vert A^{ n } \Vert^{ 1 / n } \leq
  \lim_{ n \to \infty } \Vert A \Vert = \Vert A \Vert.
\end{equation}

Przejdźmy teraz do obliczeń na elementach $A_{ n }$
i~$R_{ n } = ( \UnitAlg - A_{ n } )^{ -1 }$. Przyjmujemy, że~$R_{ n }$ istniej
dla każdej wartości $n$, a~tym samym
$R_{ n }^{ \; -1 } = ( \UnitAlg - A_{ n } )$ też istnieje dla każdego~$n$. Tym
samym możemy napisać
\begin{equation}
  \label{eq:Bratteli-Robinson-Operator-Algebras-ETC-Vol-I-s01-42}
  \UnitAlg - R_{ n } = R_{ n }^{ \; -1 } R_{ n } - R_{ n } =
  ( R_{ n }^{ \; -1 } - \UnitAlg ) R_{ n } =
  -A_{ n } R_{ n }.
\end{equation}
Ta równość inaczej zapisana daje nam
$\UnitAlg - R_{ n } = -A_{ n } ( \UnitAlg - A_{ n } )^{ -1 }$. Jeśli
$\Vert A_{ n } \Vert \to 0$ to dla dostatecznie dużych $n$ mamy $\Vert A_{ n } \Vert < 1$,
czyli możemy użyć oszacowania
\eqref{eq:Bratteli-Robinson-Operator-Algebras-ETC-Vol-I-s01-32}, tym samym
dostajemy
\begin{equation}
  \label{eq:Bratteli-Robinson-Operator-Algebras-ETC-Vol-I-s01-43}
  \Vert \UnitAlg - R_{ n } \Vert \leq
  \frac{ \Vert A_{ n } \Vert }{ 1 - \Vert A_{ n } \Vert }.
\end{equation}
W~konsekwencji jeśli $\Vert A_{ n } \Vert \to 0$ to również
$\Vert \UnitAlg - R_{ n } \Vert \to 0$.

Przekształcając
\eqref{eq:Bratteli-Robinson-Operator-Algebras-ETC-Vol-I-s01-42} w~inny
sposób dostajemy $A_{ n } = -( \UnitAlg - R_{ n } ) R_{ n }^{ \; -1 }$. Będziemy
chcieli teraz zbadać zachowanie $\Vert A_{ n } \Vert$ w~sytuacji gdy
$\Vert \UnitAlg - R_{ n } \Vert \to 0$, dlatego potrzebujemy wyrazić w~jakiś sposób
$R_{ n }^{ \; - 1 }$ za~pomocą $\UnitAlg - R_{ n }$. Ponieważ dla
odpowiednio dużych $n$ zachodzi
$\Vert \UnitAlg - R_{ n } \Vert < 1$, więc możemy skorzystać ze wzoru
\eqref{eq:Bratteli-Robinson-Operator-Algebras-ETC-Vol-I-s01-25}:
\begin{equation}
  \label{eq:Bratteli-Robinson-Operator-Algebras-ETC-Vol-I-s01-44}
  A_{ n } =
  -( \UnitAlg - R_{ n } )
  \big( \UnitAlg - ( \UnitAlg - R_{ n } ) \big)^{ -1 }.
\end{equation}
Korzystając ze wzoru
\eqref{eq:Bratteli-Robinson-Operator-Algebras-ETC-Vol-I-s01-26} dostajemy
\begin{equation}
  \label{eq:Bratteli-Robinson-Operator-Algebras-ETC-Vol-I-s01-45}
  \Vert A_{ n } \Vert \leq
  \frac{ \Vert \UnitAlg - R_{ n } \Vert }{ 1 - \Vert \UnitAlg - R_{ n } \Vert }.
\end{equation}
Tym samym $\Vert \UnitAlg - R_{ n } \Vert \to 0$ pociąga za sobą $\Vert A_{ n } \Vert \to 0$, co
wraz z~poprzednio udowodnionym wnioskiem oznacza, że~jedna z~tych granic
implikuje drugą.

Kolejna część dowodu opiera~się na następującej pomyśle. Jak wykazano
wcześniej w~dowodzie możemy przyjąć, iż~$r_{ A } > 0$. Rozważmy zbiór
\begin{equation}
  \label{eq:Bratteli-Robinson-Operator-Algebras-ETC-Vol-I-s01-46}
  S_{ A } =
  \{ \lambda; \; \lambda \in \Cbb, \, | \lambda | \geq r_{ A } \}.
\end{equation}
Przyjmiemy, że~$r_{ A } > \rho( A )$, co oznacza, że~$S_{ A } \subseteq r_{ \Ufrak }( A )$
i~wykażemy, iż~prowadzi to do sprzeczności. Tym samym musi być
być~$r_{ A } \leq \rho( A )$, co wraz z~poprzednio dowiedzionymi wcześniej
nierównościami kończy dowód twierdzenia.

Dowód zaczniemy od wprowadzenia podstawowego zespolonego pierwiastka
$n$-tego stopnia z~$1$, będziemy go oznaczać symbolem~$\omega$:
\begin{equation}
  \label{eq:Bratteli-Robinson-Operator-Algebras-ETC-Vol-I-s01-47}
  \omega = e^{ i \frac{ 2 \pi }{ n } }.
\end{equation}
Ponieważ $| \omega | = 1$ więc, jeśli $\lambda \in S_{ A }$ to również
$\lambda / \omega^{ k }$, $k = 1, 2, \ldots, n$, tym samym element
\begin{equation}
  \label{eq:Bratteli-Robinson-Operator-Algebras-ETC-Vol-I-s01-48}
  \frac{ \lambda }{ \omega^{ k } } \UnitAlg - A,
\end{equation}
jest odwracalny. Na mocy zależności
\eqref{eq:Bratteli-Robinson-Operator-Algebras-ETC-Vol-I-s01-19} również
element
\begin{equation}
  \label{eq:Bratteli-Robinson-Operator-Algebras-ETC-Vol-I-s01-49}
  \UnitAlg - \frac{ \omega^{ k } A }{ \lambda }
\end{equation}
jest odwracalny. W~następnym kroku definiujemy
\begin{equation}
  \label{eq:Bratteli-Robinson-Operator-Algebras-ETC-Vol-I-s01-50}
  R_{ n }( A; \lambda ) =
  n^{ -1 } \sum_{ k = 1 }^{ n } \left( \UnitAlg -
    \frac{ \omega^{ k } A }{ \lambda } \right)^{ -1 }.
\end{equation}
Funkcja ta jest dobrze określona dla wszystkich $\lambda \in S_{ A }$. Wedle książki
„elementarny rachunek” pozwala pokazać, że
\begin{equation}
  \label{eq:Bratteli-Robinson-Operator-Algebras-ETC-Vol-I-s01-51}
  R_{ n }( A; \lambda ) =
  \left( \UnitAlg - \frac{ A^{ n } }{ \lambda^{ n } } \right)^{ -1 }.
\end{equation}
Pierwsze pytanie na które należy odpowiedzieć w~kontekście tego wzoru, jest
to, czy możemy zapisać wyraz $( \UnitAlg - \omega^{ k } A / \lambda )^{ -1 }$,
$\lambda \in S_{ A }$ za pomocą szeregu
\eqref{eq:Bratteli-Robinson-Operator-Algebras-ETC-Vol-I-s01-30}?
W~ogólnym przypadku odpowiedź jest negatywna.

Przypomnijmy, że~$r_{ A }$ jest zdefiniowany jako limes supremum ciągu
$\Vert A^{ n }\Vert^{ 1 / n }$, jeśli więc $| \lambda | > r_{ A }$,
to~$| \lambda | > \Vert A^{ n } \Vert^{ 1 / n }$ dla pewnego~$n$. Tym samym mamy
$| \lambda |^{ n } > \Vert A^{ n } \Vert$ i~można powtórzyć rozumowanie z~początku dowodu,
by pokazać, że~element $( \lambda - A )^{ -1 }$ można wyrazić za pomocą szeregu
potęgowego. Jednak dla $| \lambda | = r_{ A }$ takiego rozumowania nie można
powtórzyć. W~tym momencie nie możemy wykluczyć, że~dla pewnego $A$ zachodzi
przykładowo
\begin{equation}
  \label{eq:Bratteli-Robinson-Operator-Algebras-ETC-Vol-I-s01-52}
  \Vert A^{ n } \Vert^{ 1 / n } = 1 + \frac{ 1 }{ n },
\end{equation}
więc nie istnieje takie $n$, że~$| \lambda | = r_{ A } > \Vert A^{ n } \Vert^{ 1 / n }$.
Wobec tego musimy znaleźć metodę dowodu, która działa dla wszystkich
$\lambda \in S_{ A }$.

Nie znalazłem żadnego prawdziwie elementarnego sposobu pokazania,
że~wzory \eqref{eq:Bratteli-Robinson-Operator-Algebras-ETC-Vol-I-s01-50}
i~\eqref{eq:Bratteli-Robinson-Operator-Algebras-ETC-Vol-I-s01-51} są
równoważne. Jedyną ideą jaką mam jest to, że~ponieważ wszystkie elementy
algebry jakie występują w~tych wzorach wyrażają~się przez $\UnitAlg$ i~$A$,
więc są między sobą przemienne. Wiemy już bowiem, że~przemienność $A$ i~$B$
oznacza przemienność $A^{ -1 }$ i~$B^{ -1 }$. W~takiej sytuacji powinno
dać~się przenieść dowody twierdzeń o~funkcjach wymiernych zmiennej
zespolonej.

Niech teraz $\alpha > 0$. Z~zasadniczego twierdzenia algebry wiemy, że~zachodzi
\begin{equation}
  \label{eq:Bratteli-Robinson-Operator-Algebras-ETC-Vol-I-s01-53}
  ( z^{ n } - \alpha^{ n } ) =
  ( z - \omega \alpha ) ( z - \omega^{ 2 } \alpha ) ( z - \omega^{ 3 } \alpha ) \ldots
  ( z - \omega^{ n } \alpha ).
\end{equation}
Jeśli wszystkie elementy $z - \omega \alpha$, $z - \omega^{ 2 } \alpha$, \ldots, $z - \omega^{ n } \alpha$,
$z^{ n } - \alpha^{ n }$ są odwracalne, co~dla liczb zespolonych jest równoważne
warunkowi $z \neq \omega^{ k } \alpha$,  $k = 1, 2, \ldots, n$, to~powinien zachodzić rozkład
na ułamki prost, postaci
\begin{equation}
  \label{eq:Bratteli-Robinson-Operator-Algebras-ETC-Vol-I-s01-54}
  \frac{ 1 }{ z^{ n } - \alpha^{ n } } =
  \frac{ 1 }{ n } \frac{ 1 }{ z - \omega \alpha } +
  \frac{ 1 }{ n } \frac{ 1 }{ z - \omega^{ 2 } \alpha } + \ldots +
  \frac{ 1 }{ n } \frac{ 1 }{ z - \omega^{ n } \alpha }.
\end{equation}
Podstawiając $z = 1$ dostajemy
\begin{equation}
  \label{eq:Bratteli-Robinson-Operator-Algebras-ETC-Vol-I-s01-55}
  \frac{ 1 }{ 1 - \alpha^{ n } } =
  \frac{ 1 }{ n } \frac{ 1 }{ 1 - \omega \alpha } +
  \frac{ 1 }{ n } \frac{ 1 }{ 1 - \omega^{ 2 } \alpha } + \ldots +
  \frac{ 1 }{ n } \frac{ 1 }{ 1 - \omega^{ n } \alpha },
\end{equation}
przy czym musi zachodzić $\omega^{ k } \alpha \neq 1$, $k = 1, 2, \ldots, n$.
Potrzebowalibyśmy jakoś uogólnić to na przypadek przemiennych elementów
algebry~$C^{ * }$. W~przypadku wzorów
\eqref{eq:Bratteli-Robinson-Operator-Algebras-ETC-Vol-I-s01-50}
i~\eqref{eq:Bratteli-Robinson-Operator-Algebras-ETC-Vol-I-s01-51} to,
że~występujące w~nich odwrotności wyrazów algebry istnieją, wynika z~założeń
które przyjęliśmy, więc analog rozkładu
\eqref{eq:Bratteli-Robinson-Operator-Algebras-ETC-Vol-I-s01-55} powinien
zawsze istnieć. Co nie zmienia faktu, że~w~końcu dojdziemy do sprzeczności.

\VerSpaceFour





\noindent
\Str{27} Krótko omówimy tutaj wzór
\begin{equation}
  \label{eq:Bratteli-Robinson-Operator-Algebras-ETC-Vol-I-s01-56}
  ( \lambda^{ n } \HorSpaceOne \UnitAlg - A^{ n } ) =
  ( \lambda \HorSpaceOne \UnitAlg - A )
  ( \lambda^{ n - 1 } \HorSpaceOne \UnitAlg + \lambda^{ n - 2 } A + \ldots +
  A^{ n - 1 } )
\end{equation}
i~jego konsekwencje. Dla wygody wprowadzamy oznaczenia. z~oznaczeń
$A_{ 1 } = \lambda^{ n } \HorSpaceOne \UnitAlg - A^{ n }$,
$A_{ 2 } = \lambda \HorSpaceOne \UnitAlg - A$
i~$A_{ 3 } = \lambda^{ n - 1 } \HorSpaceOne \UnitAlg + \lambda^{ n - 2 } A + \ldots + A^{ n - 1 }$.

Zakładamy, że~$\lambda^{ n } \in r_{ \Ufrak }( A^{ n } )$, czyli $A_{ 1 }$ jest
elementem odwracalnym. Ponieważ $A_{ 1 }$, $A_{ 2 }$ i~$A_{ 3 }$ są
przemienne, więc jako pokazano wcześniej w~tych notatkach, wszystkie trzy są
odwracalne. Skoro więc $A_{ 2 } = \lambda \HorSpaceOne \UnitAlg - A$ jest
odwracalny, więc $\lambda \in r_{ \Ufrak }( A )$. Skoro więc fakt,
że~$\lambda^{ n } \in r_{ \Ufrak }( A^{ n } )$ implikuje $\lambda \in r_{ \Ufrak }( A )$, więc na
mocy definicji i~zasady kontrapozycji, jeśli $\lambda \in \sigma_{ \Ufrak }( A )$ to
$\lambda^{ n } \in \sigma_{ \Ufrak }( A^{ n } )$. Tę ostatnią własność Bratteli i~Robinson
zapisują jako
\begin{equation}
  \label{eq:Bratteli-Robinson-Operator-Algebras-ETC-Vol-I-s01-57}
  \sigma_{ \Ufrak }( A )^{ n } \subset \sigma_{ \Ufrak }( A^{ n } ),
\end{equation}
co jest dość niestandardowym użyciem symbolu $S^{ n }$, gdzie $S$ jest
podzbiorem pewnej struktury algebraicznej, w~której istnieje działanie
zwane mnożeniem. Zwykle bowiem przez $S^{ 2 }$ (analogicznie dla $n > 2$)
rozumiemy zbiór
\begin{equation}
  \label{eq:Bratteli-Robinson-Operator-Algebras-ETC-Vol-I-s01-58}
  S^{ 2 } = \{ \alpha \beta; \alpha, \beta \in S \},
\end{equation}
podczas gdy Bratteli i~Robinson oznaczają tym symbolem zbiór
\begin{equation}
  \label{eq:Bratteli-Robinson-Operator-Algebras-ETC-Vol-I-s01-59}
  S^{ 2 } = \{ \alpha^{ 2 }; \alpha \in S \}.
\end{equation}

\VerSpaceFour






% ##################
\newpage

\CenterBoldFont{Błędy}


\begin{center}

  \begin{tabular}{|c|c|c|c|c|}
    \hline
    Strona & \multicolumn{2}{c|}{Wiersz} & Jest
                              & Powinno być \\ \cline{2-3}
    & Od góry & Od dołu & & \\
    \hline
    21  & & 10 & $f f^{ * }$ & $f^{ * } f$ \\
    22  & 17 & & $( \alpha, A^{ * } )$ & $( \alpha, A )^{ * }$ \\
    24  & & 15 & two-sided & nontrivial two-sided \\
    % & & & & \\
    % & & & & \\
    % & & & & \\
    % & & & & \\
    % & & & & \\
    % & & & & \\
    % & & & & \\
    % & & & & \\
    % & & & & \\
    % & & & & \\
    % & & & & \\
    \hline
  \end{tabular}

\end{center}

\VerSpaceSix


Str. 31. \ldots the resolvent $R( \lambda ) = ( A - \lambda I )$\ldots

\noindent
\StrWierszG{480}{13} \\
\Jest  Henrichs, R. W. Decomposition \\
\Powin Henrichs, R. W. \\
Decomposition \\
\StrWierszG{486}{20} \\
\Jest  Schflitzel, R. Maximal \\
\Powin Schflitzel, R. \\
Maximal \\
\StrWierszD{487}{11} \\
\Jest  Taylor, J. L. The Tomita \\
\Powin Taylor, J. L. \\
The Tomita \\
\StrWierszG{488}{16} \\
\Jest  Wils, W. Direct \\
\Powin Wils, W. \\
Direct \\






% ############################










% #####################################################################
% #####################################################################
% Bibliography

\bibliographystyle{plalpha}

\bibliography{MathematicsBooks}{}





% ############################

% End of the document
\end{document}
