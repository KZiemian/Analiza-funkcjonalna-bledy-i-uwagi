% ---------------------------------------------------------------------
% Podstawowe ustawienia i pakiety
% ---------------------------------------------------------------------
\RequirePackage[l2tabu, orthodox]{nag} % Wykrywa przestarzałe i niewłaściwe
% sposoby używania LaTeXa. Więcej jest w l2tabu English version.
\documentclass[a4paper,11pt]{article}
% {rozmiar papieru, rozmiar fontu}[klasa dokumentu]
\usepackage[MeX]{polski} % Polonizacja LaTeXa, bez niej będzie pracował
% w języku angielskim.
\usepackage[utf8]{inputenc} % Włączenie kodowania UTF-8, co daje dostęp
% do polskich znaków.
\usepackage{lmodern} % Wprowadza fonty Latin Modern.
\usepackage[T1]{fontenc} % Potrzebne do używania fontów Latin Modern.



% ------------------------------
% Podstawowe pakiety (niezwiązane z ustawieniami języka)
% ------------------------------
\usepackage{microtype} % Twierdzi, że poprawi rozmiar odstępów w tekście.
\usepackage{graphicx} % Wprowadza bardzo potrzebne komendy do wstawiania
% grafiki.
\usepackage{verbatim} % Poprawia otoczenie VERBATIME.
\usepackage{textcomp} % Dodaje takie symbole jak stopnie Celsiusa,
% wprowadzane bezpośrednio w tekście.
\usepackage{vmargin} % Pozwala na prostą kontrolę rozmiaru marginesów,
% za pomocą komend poniżej. Rozmiar odstępów jest mierzony w calach.
% ------------------------------
% MARGINS
% ------------------------------
\setmarginsrb
{ 0.7in} % left margin
{ 0.6in} % top margin
{ 0.7in} % right margin
{ 0.8in} % bottom margin
{  20pt} % head height
{0.25in} % head sep
{   9pt} % foot height
{ 0.3in} % foot sep



% ------------------------------
% Często przydatne pakiety
% ------------------------------
\usepackage{csquotes} % Pozwala w prosty sposób wstawiać cytaty do tekstu.
\usepackage{xcolor} % Pozwala używać kolorowych czcionek (zapewne dużo
% więcej, ale ja nie potrafię nic o tym powiedzieć).



% ------------------------------
% Pakiety do tekstów z nauk przyrodniczych
% ------------------------------
\let\lll\undefined % Amsmath gryzie się z językiem pakietami do języka
% polskiego, bo oba definiują komendę \lll. Aby rozwiązać ten problem
% oddefiniowuję tę komendę, ale może tym samym pozbywam się dużego Ł.
\usepackage[intlimits]{amsmath} % Podstawowe wsparcie od American
% Mathematical Society (w skrócie AMS)
\usepackage{amsfonts, amssymb, amscd, amsthm} % Dalsze wsparcie od AMS
% \usepackage{siunitx} % Dla prostszego pisania jednostek fizycznych
\usepackage{upgreek} % Ładniejsze greckie litery
% Przykładowa składnia: pi = \uppi
\usepackage{slashed} % Pozwala w prosty sposób pisać slash Feynmana.
\usepackage{calrsfs} % Zmienia czcionkę kaligraficzną w \mathcal
% na ładniejszą. Może w innych miejscach robi to samo, ale o tym nic
% nie wiem.



% ##########
% Tworzenie otoczeń "Twierdzenie", "Definicja", "Lemat", etc.
\newtheorem{theorem}{Twierdzenie}  % Komenda wprowadzająca otoczenie
% „theorem” do pisania twierdzeń matematycznych
\newtheorem{definition}{Definicja}  % Analogicznie jak powyżej
\newtheorem{corollary}{Wniosek}



% ------------------------------
% Pakiety napisane przez użytkownika.
% Mają być w tym samym katalogu to ten plik .tex
% ------------------------------
\usepackage{latexgeneralcommands}
\usepackage{mathcommands}

\usepackage{functionalanalysiscommands}



% ---------------------------------------------------------------------
% Dodatkowe ustawienia dla języka polskiego
% ---------------------------------------------------------------------
\renewcommand{\thesection}{\arabic{section}.}
% Kropki po numerach rozdziału (polski zwyczaj topograficzny)
\renewcommand{\thesubsection}{\thesection\arabic{subsection}}
% Brak kropki po numerach podrozdziału



% ------------------------------
% Ustawienia różnych parametrów tekstu
% ------------------------------
\renewcommand{\baselinestretch}{1.1}

% Ustawienie szerokości odstępów między wierszami w tabelach.
\renewcommand{\arraystretch}{1.4}





% ------------------------------
% Pakiet „hyperref”
% Polecano by umieszczać go na końcu preambuły.
% ------------------------------
\usepackage{hyperref} % Pozwala tworzyć hiperlinki i zamienia odwołania
% do bibliografii na hiperlinki.










% ---------------------------------------------------------------------
% Tytuł i tytuł tekstu
\title{Analiza funkcjonalna \\
  {\Large Błędy i~uwagi}}

\author{Kamil Ziemian, korekta Wojciech Dyba}


% \date{}
% ---------------------------------------------------------------------










% ####################################################################
\begin{document}
% ####################################################################





% ######################################
\maketitle % Tytuł całego tekstu
% ######################################





% ######################################
\section{Algebry operatorów i~algebry topologiczne}

\vspace{\spaceTwo}
% ######################################



% ############################
\Work{ % Autor i tytuł dzieła
  A. V. Balakrishnan \\
  \textit{Analiza funkcjonalna stosowana},
  \cite{BalakrishnanAnalizaFunkcjonalnaStosowana1992}}



% ##################
\CenterBoldFont{Błędy}


\begin{center}

  \begin{tabular}{|c|c|c|c|c|}
    \hline
    & \multicolumn{2}{c|}{} & & \\
    Strona & \multicolumn{2}{c|}{Wiersz} & Jest
                              & Powinno być \\ \cline{2-3}
    & Od góry & Od dołu & & \\
    \hline
    13  & & 10 & $\int^{ 1 }_{ -1 }$ & $2 \int^{ 1 }_{ -1 }$ \\
    14  & 15 & & $<$ & $\leq$ \\
    15  & 18 & & otrzymywaliśmy & otrzymalibyśmy \\
    % & & & & \\
    \hline
  \end{tabular}

\end{center}

\vspace{\spaceTwo}
% ############################










% ############################
\newpage

\Work{ % Autorzy i tytuł dzieła
  Anton Deitmar, Siegrfied Echterhoff \\
  \textit{Principles~of Harmonic Analysis},
  \cite{DeitmarEcherhoffPrinciplesOfHarmonicAnalysis2009}}


% ##################
\CenterBoldFont{Uwagi}


\start \textbf{Str. 42. Twierdzenie 2.2.6. Wzór na~promień spektralny.}
Aby~wykazać, że~$\lambda^{ n } \in \sigma_{ \Acal }( a^{ n })$
należy pokazać, iż~element
\begin{equation}
  \label{eq:DE-1}
  ( \lambda 1 - a ) \sum_{ j = 0 }^{ n - 1 } \lambda^{ j } a^{ n - 1 - j }
\end{equation}
jest nieodwracalny. Nie jest to jednak dla~mnie oczywiste. Arkadiusz
Bochniak powiedział, żeby zobaczyć na~dowód twierdzenia widmie
wielomian operatora (ang. \textit{polynomial spectral mapping
  theorem}). Należy przy tym zauważyć, że~z~tego twierdzenie wynika
od~razu poszukiwana własność.

\vspace{\spaceFour}





\start \textbf{Str. 46. Lemat 2.4.2.} W~dowodzie tego twierdzenie trzeba
chyba rozumować następująco. Chcemy pokazać,
że~$m( a ) \leq \norm{ a }$ dla każdego $a \in \Acal$.

Rozpatrujemy dwa przypadki. 1) Gdy $m( a ) = 0$, wtedy oczywiście
$m( a ) = 0 \leq \norm{ a }$. 2). Gdy $m( a ) \neq 0$. Wówczas
$m( a - m( a ) 1 ) = 0$. Gdyby istniał element $b$~odwrotny
do~$a - m( a ) 1$, to~zachodziłoby
\begin{equation}
  \label{eq:DE-2}
  1 = m( 1 ) = m( b ( a - m( a ) 1 ) ) = m( b ) m( a - m( a ) ) =
  m( b ) 0 = 0,
\end{equation}
co jest niemożliwe. Dalej można już postępować jak w~zaprezentowanym
dowodzie.





% ##################
\newpage

\CenterBoldFont{Błędy}


\begin{center}

  \begin{tabular}{|c|c|c|c|c|}
    \hline
    & \multicolumn{2}{c|}{} & & \\
    Strona & \multicolumn{2}{c|}{Wiersz} & Jest
                              & Powinno być \\ \cline{2-3}
    & Od góry & Od dołu & & \\
    \hline
    42  & 10 & & $s\: u\: p$ & $\sup$ \\
    44  &  9 & & Banach-Algebras & Banach algebras \\
    % & & & & \\
    % & & & & \\
    \hline
  \end{tabular}

\end{center}


\vspace{\spaceTwo}
% ############################










% ############################
\newpage

\Work{ % Autorzy i tytuł dziełą
  I. M. Gelfand, G. E. Shilov \\
  \textit{Generalized Functions: Volume~I, Properties and~Operations},
  \cite{GelfandShilovGeneralizedFunctionsVolI1964}}


% ##################
\CenterBoldFont{Błędy}


\begin{center}

  \begin{tabular}{|c|c|c|c|c|}
    \hline
    & \multicolumn{2}{c|}{} & & \\
    Strona & \multicolumn{2}{c|}{Wiersz} & Jest
                              & Powinno być \\ \cline{2-3}
    & Od góry & Od dołu & & \\
    \hline
    12  &  4 & & $f_{ 0 } = 0$ & $x_{ 0 } = 0$ \\
    28  & 16 & & $r \leq a$ & $r \geq a$ \\
    % & & & & \\
    \hline
  \end{tabular}

\end{center}

\vspace{\spaceTwo}
% ############################










% ############################
\newpage

\Work{ % Autor i tytuł dzieła
  Walter Rudin \\
  \textit{Analiza funkcjonalna},
  \cite{RudinAnalizaFunkcjonalna2012}}


% ##################
\CenterBoldFont{Uwagi}


\start W~tych notatkach, tak jak i~w~samej książce, będziemy domyślnie
zakładać, że~wszystkie przestrzenie wektorowe są albo nad ciałem liczb
rzeczywistych~$\Rbb$ albo liczby zespolonych $\Cbb$. Zagadnienie które
wyniki można uogólnić na ciała takie jak $\Zbb_{ p }$, gdzie $p$~jest liczbą
pierwszą, jest bardzo ciekawe, lecz w~tym momencie nie jest na tyle ważny
byśmy mogli się nim zajmować.

\vspace{\spaceFour}





% ##################
\CenterBoldFont{Uwagi do konkretnych stron}


\start \Str{}

\vspace{\spaceFour}





\start \Str{}

\vspace{\spaceFour}





\start \Str{}

\vspace{\spaceFour}





\start \Str{}

\vspace{\spaceFour}





\start \Str{}

\vspace{\spaceFour}





\start \Str{}

\vspace{\spaceFour}





\start \Str{}

\vspace{\spaceFour}





\start \Str{}

\vspace{\spaceFour}





\start \Str{}

\vspace{\spaceFour}





\start \Str{}

\vspace{\spaceFour}





\start \Str{}

\vspace{\spaceFour}





\start \Str{}

\vspace{\spaceFour}





\start \Str{}

\vspace{\spaceFour}





\start \Str{}

\vspace{\spaceFour}





\start \Str{}

\vspace{\spaceFour}



\start \Str{}

\vspace{\spaceFour}





\start \Str{}

\vspace{\spaceFour}





\start \Str{}

\vspace{\spaceFour}





\start \Str{}

\vspace{\spaceFour}





\start \Str{}





% ##################
\newpage

\CenterBoldFont{Błędy}


\begin{center}

  \begin{tabular}{|c|c|c|c|c|}
    \hline
    Strona & \multicolumn{2}{c|}{Wiersz} & Jest
                              & Powinno być \\ \cline{2-3}
    & Od góry & Od dołu & & \\
    \hline
    % & &   &  &  \\
    % &   & &  &  \\
    % & &   &  &  \\
    % & &   &  &  \\
    % & &   &  &  \\
    % & &   &  &  \\
    % &   & & &  \\
    % & &  &  &  \\
    % &   & &  &  \\
    % & &   &  &  \\
    % &   & &  &  \\
    % &   & &  &  \\
    % & &   &  &  \\
    % & &   & &  \\
    % & &  &  &  \\
    % & &   &  &  \\
    % & &   &  &  \\
    % &   & &  &  \\
    % &   & & &  \\
    % & &  &  &  \\
    % & &   &  &  \\
    % &  & &  &  \\
    % & &  & &  \\
    % & &  & &  \\
    % & &  &  &  \\
    % &   & &  &  \\
    % & &   &  &  \\
    % & &  &  &  \\
    % &  & &  &  \\
    %  &  & &  &  \\
    %  & &   &  &  \\
    %  &   & &  &  \\
    %  &   & &  &  \\
    %  &   & &  &  \\
    % & &  &  &  \\
    % & &  &  &  \\
    % & &  &  &  \\
    % & & & & \\
    % & & & & \\
    % & & & & \\
    % & & & & \\
    % & & & & \\
    % & & & & \\
    % & & & & \\
    % & & & & \\
    % & & & & \\
    \hline
  \end{tabular}

\end{center}

\vspace{\spaceTwo}


\noindent
% \StrWd{}{} \\
% \Jest   \\
% \Powin  \\
% \StrWg{}{} \\
% \Jest   \\
% \Powin  \\
% \StrWd{}{} \\
% \Jest   \\
% \Powin  \\
% \StrWd{}{} \\
% \Jest   \\
% \Powin  \\
% \StrWg{}{} \\
% \Jest   \\
% \Powin  \\
% \StrWd{}{} \\
% \Jest   \\
% \Powin  \\
% \StrWd{}{} \\
% \Jest   \\
% \Powin  \\


\vspace{\spaceTwo}
% ############################










% ############################
\newpage

\Work{ % Autor i tytuł dzieła
  Armen H.~Zemanian \\
  \textit{Teoria dystrybucji i~analiza transformat},
  \cite{ZemanianTeoriaDystrybucji1969}}


% ##################
\CenterBoldFont{Uwagi do konkretnych stron}


\start \Str{16} Rozważania o~zamkniętości ze~względu na zbieżność
przestrzeni funkcji próbnych $\Dcal$~są w~mojej ocenie trochę
chaotyczne, spróbuję więc je jakoś rozjaśnić. W~istocie chodzi o~to,
że~definiujemy zbieżność ciągu funkcji próbnych
$\{ \varphi_{ \nu }( t ) \}$, żądając od~niego dwóch własności:
1)~Dla każdego $k$, ciąg $\{ \varphi_{ \nu }^{ ( k ) }( t ) \}$ jest
zbieżny jednostajnie do jakiejś funkcji
$f_{ k }( t ), \; k = 0, 1, \ldots$ \\
2)~Nośniki wszystkich funkcji $\{ \varphi_{ \nu }( t ) \}$ zawarte są
w~wspólnym zbiorze zwartym. \\
Zwróćmy uwagę, że~na~razie nie możemy mówić, iż~ciąg
$\{ \varphi_{ \nu }( t ) \}$ jest zbieżny do funkcji
$f( t ) = f_{ 0 }( t )$, bo~nie wiemy, czy znajomość funkcji $f( t )$
pozwala nam odtworzyć funkcje $f_{ 1 }( t ), f_{ 2 }( t ), \ldots$ W~tym
momencie właściwsze byłoby powiedzenie, że~ciąg
$\{ \varphi_{ \nu }( t ) \}$ jest zbieżny do~rodziny funkcji
$f_{ k }( t )$.

Jednak na mocy twierdzeń z~analizy matematycznej, które można znaleźć
np.~w~książce Schwartza, str.~649--652,~\cite{SchwartzKursAnalizyMatematycznejVolI1979}, przy tych
warunkach funkcja $f( t )$ jest klasy $\Ccal^{ \infty }( \Rbb )$ i~zachodzi
$f^{ ( k ) }( t ) = f_{ k }( t )$. Tym samy pokazaliśmy, że~zbieżność
zdefiniowana wyżej, jest rzeczywiście zbieżnością do~jakiejś funkcji.
Z~drugiego warunku zbieżności ciągu funkcji próbnych wynika,
że~funkcja $f( t )$ również ma~zwarty nośnik i~tym samym należy
do~$\Dcal$.

Tym samym możemy stwierdzić, że~jeśli istnieje funkcja zespolona,
do~której ciąg funkcji próbnych jest zbieżny w~podanym wyżej sensie,
to~funkcja graniczna również jest funkcją próbną.

\vspace{\spaceFour}





\start \Str{19} Fakt, że~funkcja dana wzorem (4) jest klasy
$\Ccal^{ \infty }( \Rbb )$ został, chyba przyjęty w~domyśle. Należy jednak~się
nad nim zatrzymać i~dowieść tego faktu. \Dok

\vspace{\spaceFour}





\start \Str{20} Aby wywód był pełny należy jeszcze dowieść,
że~$[ \varphi( t ) ]^{ \frac{ 1 }{ n } } = \sqrt[ n ]{ \varphi( t ) }$ jest
klasy $\Ccal^{ \infty }( \Rbb )$.

\vspace{\spaceFour}





\start \Str{21} Warto zatrzymać~się tu na~chwilę, nad faktem który
bardzo słusznie jest stale przypominany w~kontekście teorii
dystrybucji: nie czegoś takiego jak wartość dystrybucji w~punkcie.
Dystrybucja $f$ z~przestrzeni $\Dcal'( \Rbb ??? )$ jest określona
na~funkcjach z~$\Dcal( ??? )$, a~nie na $\Rbb ???$ i~nawet dla dystrybucji
regularnej wartość $f( t )$ w~konkretnym punkcie nie ma zwykle sensu.
Punkt ma bowiem miarę Lebesgue’a równą~0, więc wartość funkcji która
reprezentuje tę~dystrybucję można w~nim dowolnie zmienić.

Jedynie jeśli w~klasie abstrakcji funkcji reprezentujących daną
dystrybucję istnieje jedna funkcja wyróżniona, to~wartość tej funkcji
jest też wartością dystrybucji w~konkretnym punkcie. Jest tak
np.~jeśli jedna z~tych funkcji jest ciągła.

\vspace{\spaceFour}





\start \Str{29} Użycie we wzorze~(3) symbolu
\begin{equation}
  \label{eq:Zemanian-01}
  \lim_{ \epsilon \to 0^{ + } } \int_{ \epsilon }^{ b } t^{ -3 / 2 } \varphi( t ) \, dt,
\end{equation}
jest trochę mylące. Zdaje~się bowiem sugerować, że~całka
$\int_{ 0 }^{ b } t^{ -3 / 2 } \varphi( t ) \, dt$ istniej i~jest równa tej
granicy, jednak całka ta jest rozbieżna jeśli $\varphi( 0 ) \neq 0$.
Poprawniejsze byłoby następujące rozumowanie.

Najpierw rozpatrzmy całkę
\begin{equation}
  \label{eq:Zemanian-02}
  \int_{ \epsilon }^{ b } t^{ -3/2 } \varphi( t ) \, dt =
  \frac{ 2 \varphi( 0 ) }{ \sqrt{ \epsilon } } - \frac{ 2 \varphi( 0 ) }{ \sqrt{ b } }
  + \int_{ \epsilon }^{ b } \frac{ \psi( t ) }{ \sqrt{ t } } \, dt.
\end{equation}
Całka ta jest skończona dla każdego $\varepsilon$ i~powyższy wzór pozwala
nam zidentyfikować źródło rozbieżności w~granicy $\varepsilon \searrow 0$.
Dysponując tą wiedzą, możemy zdefiniować dystrybucję
$\Pf \, t^{ -3 / 2 } \HeavisideFunPlus( t )$ wzorem
\begin{equation}
  \label{eq:Zemanian-03}
  \langle \Pf \, t^{ -3/2 } \HeavisideFunPlus( t ), \varphi( t ) \rangle =
  \lim_{ \epsilon \to 0^{ + } } \left[ \int_{ \epsilon }^{ +\infty } t^{ -3/2 } \varphi( t ) \, dt
    - \frac{ 2 \varphi( 0 ) }{ \sqrt{ \epsilon } } \right].
\end{equation}

\vspace{\spaceFour}





\start \Str{32} W~tym miejscu można bez trudności, i~nawet byłoby to
bardziej naturalne, wprowadzić pseudofunkcję
$\Pf \frac{ \HeavisideFunPlus( -t )}{ t }$.

\vspace{\spaceFour}





\start \Str{33--34} Przyjmijmy najpierw, że~jeżeli dana jest krzywa
$A$, to przez $A( t )$ będziemy oznaczać taką funkcję, że~krzywa
ta~ma~przedstawienie $( t, A( t ) )$. Teraz należy dokonać takiej
zmienny w~linii 2~(od~dołu) na~stronie~34. \\
\Jest  Dlatego przesunięta cześć krzywej $B$ \\
\Powin Dlatego pole pod~krzywą $\varphi( t ) B( t )$ na~przedziale
$\tau \leq t \leq \epsilon$

\vspace{\spaceFour}





\start \Str{36} W~tym miejscu po raz pierwszy chyba w~książce
pojawia~się termin „obszar”. Z~kontekstu wynika, że~należy przez
niego rozumieć dowolny podzbiór $???$.

\vspace{\spaceFour}





\start \Str{41} Obok nazwy \textit{zbiór zer dystrybucji}, proponowałbym
również używać dla tego pojęcia nazwy \textit{zbiór zerowy dystrybucji}.

\vspace{\spaceFour}




\start \Str{42} Logiczniej byłoby zaraz po~paragrafie \S 1.6 umieścić
paragraf \S 1.8. Pojęcie zbioru zerowego i~nośnika dystrybucji nie ma
wielkiego sensu 1.8.1 i~wynikających z~niego konsekwencji. Paragraf \S
1.7 \textit{Kilka operacji na~dystrybucjach} najlepiej byłby umieścić
jako \S 1.6.

\vspace{\spaceFour}





\start \Str{47--51} Przedstawiony tu dowód twierdzenia 1.8.1, które
jest niezmiernie ważne, zawiera wiele luk, które postaram~się
w~dalszych podpunktach uzupełnić, tu zaś zbiorę potrzebne do~tego
informacje.

Zacznijmy od~tego, że~aby nie obciążać Czytelnika nowymi pojęciami,
nie używa pojęcia zbioru zwartego, zamiast tego mówi o~zbiorach
domkniętych i~ograniczonych\footnote{Jak wiadomo z~twierdzenia
  Heinego-Borela, zbiór w~$\Rbb^{ n }$ jest zwarty wtedy i~tylko
  wtedy, gdy jest domknięty i~ograniczony.} w~$\Rbb^{ n }$, ponieważ jednak
jest to niewygodne, będę mówił o~zbiorach zwartych. Będziemy jeszcze
potrzebowali dwóch twierdzeń odnośnie tych zbiorów.





% #############
\begin{theorem}[Walter Rudin, twr. 2.7,
  str.~45,~\cite{RudinAnalizaRzeczywistaIZespolona1998}]
  \label{thm:Zemanian-01}

  Niech $X$ będzie lokalnie zwartą przestrzenią Hausdorffa,
  $K$~zbiorem zwartym, $U$~zbiorem otwartym i~$K \subset U$. Istnieje
  zbiór otwarty o~zwartym domknięciu taki,~że
  \begin{equation}
    \label{eq:Zemanian-04}
    K \subset V \subset \overline{ V } \subset U.
  \end{equation}

\end{theorem}
% #############





Sens tego twierdzenia jest następujący. Jeśli $K \notin \{ \emptyset, X \}$
i~przestrzeń $X$ jest spójna, to wewnątrz zbioru otwartego $U$, można
powiększyć zbiór zwarty $K$ do~zbioru zwartego $\overline{ V }$, przy czym
$K \neq \overline{ V }$. Jest tak dlatego, że~$V$ jest otwarty,
a~w~przestrzeni Hausdorffa zbiór zwarty jest domknięty. Jeśli więc
przestrzeń jest spójna to nie może zajść równość $K = V$, chyba,
że~$K = \emptyset$ lub~$K = X$.

Jeśli przestrzeń nie jest spójna, to~może~się zdarzyć, że~$K$ jest
maksymalną składową spójną i tym samym jest otwarto-domkniętym
zbiorem. Wtedy jak najbardziej może~się zdarzyć, iż~$K = V$.





% #############
\begin{theorem}[Laurent Schwartz,
  str??\cite{SchwartzKursAnalizyMatematycznejVolI1979}]
  \label{thm:Zemanian-02}

  Niech $( X, d )$ będzie przestrzenią metryczną, w~której każdy zbiór
  domknięty i~ograniczony jest zwarty. Jeśli $K$ jest zbiorem zwartym,
  $D$ zbiorem domkniętym i~$K \cup D = \emptyset$, to odległość zbiorów $K$
  i~$D$
  \begin{equation}
    \label{eq:Zem-s01-05}
    d( K, D ) > 0,
  \end{equation}
  do~tego odległość ta przyjmuje w~pewnym punkcie minimum.

\end{theorem}
% #############





\noindent
Schwartz formułuje równoważne założenie twierdzenia, że~każda
kula domknięta jest zwarta. Tą~równoważność łatwo pokazać.

Będę również używał oznaczenia $\supp \varphi$ na~oznaczenie nośnika
funkcji $\varphi( t )$ (ang.~\textit{support}).

\vspace{\spaceFour}





\start \Str{48--49} \textbf{Lemat 1, uzupełnienie.} Na~podstawie twierdzenia
\eqref{thm:Zemanian-01} istniej zbiór $\Psi$ o~żądanych własnościach. Teraz
na~mocy twierdzenia \eqref{thm:Zemanian-02} mamy
$d( \Theta, \complement \Psi ) = d_{ 1 } > 0$, jeśli więc $\alpha_{ 1 } < d_{ 1 }$, to~nośnik funkcji
$\gamma_{ \alpha_{ 1 } }( t - \tau )$ zawiera~się w~$\Psi$ dla~każdego $t \in \Theta$, tym samym
$\varphi( t ) = 1$. Gwarantuje to~też, iż~funkcja podcałkowa jest klasy
$\Ccal^{ \infty }$. Dowiedliśmy więc, że~$\varphi( t ) = 1$ na~$\Theta$.

Jeśli teraz oznaczymy odległość $d( \Psi, \complement \Omega ) = d_{ 2 } > 0$ i~przyjmiemy
$\alpha_{ 2 } < d_{ 2 }$, to dla każdego $t \in \complement \Omega$ nośnik funkcji
$\gamma_{ \alpha_{ 1 } }( t - \tau )$ nie przecina~się z~$\Psi$, więc $\varphi( t ) = 0$.

\vspace{\spaceFour}





\start \Str{49} W~języku polskim znaczniej lepiej od~nazwy
\textit{lokalnie skończonego pokrycia} brzmi określenie \textit{pokrycie
  lokalnie skończone}. Jego też będę dalej używał.

\vspace{\spaceFour}





\start \Str{49} \textit{Lemat 2, uzupełnienie.} Zauważmy, że~każdy na~mocy
założeń, każdy domknięty podzbiór $\Omega_{ k }$ jest zwarty, więc
istnienie zbiorów $\Lcal_{ k }$ o~zadanych własnościach wynika ponownie
z~twierdzenia \eqref{thm:Zemanian-01}.

\vspace{\spaceFour}





\start \Str{50} \textbf{Lemat 4, uzupełnienie.} Niech $A$ będzie zbiorem
ograniczonym w~$\Rbb^{ n }$, czyli zawiera~się on w~kuli $K( 0, r )$. Trzeba
teraz pokazać, że~tylko skończona ilość zbiorów $\Ocal_{ \alpha }$ z~nowego
pokrycia może przecinać~się z~tą kulą. Oznaczymy majorantę średnicy
zbiorów przez $\Diam_{ \Cfrak }$, a~przez $M$ minimalną
odległość punktu prostopadłościanu reprezentowanego układem liczb
$\{ m_{ 1 }, \ldots, m_{ n } \}$ od~punktu $0$, czyli
\begin{equation}
  \label{eq:Zemanian-06}
  M = \left( \sum_{ i = 0 }^{ n } m_{ i } \right)^{ \frac{ 1 }{ 2 } }.
\end{equation}
Zbiór $\Ocal_{ \alpha }$ może przeciąć zadaną kulę, tylko jeśli
$M - \Diam_{ \Cfrak } < r$, czyli mówiąc prościej,
jeśli~prostopadłościan jest na tyle blisko, że~zbiór $\Ocal_{ \alpha }$
może dosięgnąć kuli $K( 0, r )$. Ponieważ istnieje tylko skończona
ilość układów liczb $\{ m_{ 1 }, \ldots, m_{ n } \}$ spełniających tę
nierówność, więc tylko skończoną ilość razy będziemy wybierali
skończoną rodzinę zbiorów $\Ocal_{ \alpha }$ takich, że~jest możliwe
by~$K( 0, r ) \cap \Ocal_{ \alpha } \neq \emptyset$.

\vspace{\spaceFour}





\start \Str{51} Cel przedstawionej tu konstrukcji jest następujący.
Potrzebujemy pokrycia lokalnie skończonego~$\Omega_{ \nu }$, którego
elementy albo nie przecinają~się z~$\Xi$, albo
$\Omega_{ \nu } \subset \Theta$, gdy $\Omega_{ \nu } \cap \Xi \neq \emptyset$.
Rozkład jedności dla tego pokrycia pozwoli pokazać, że~funkcję~$\psi$
można rozłożyć na~funkcje o~nośnikach w~zbiorach otwartych na~których
dystrybucja $f$~jest równa $0$.

Wydaje mi~się, że~do tej konstrukcji zbiór $\Xi_{ 1 }$ jest
wprowadzony niepotrzebnie.

\vspace{\spaceFour}



\start \Str{59}

\vspace{\spaceFour}



\start \Str{62}

\vspace{\spaceFour}



\start \Str{71} Wzór
\begin{equation}
  \label{eq:Zemanian-07}
  f( t ) =
  f_{ c }( t )
  - \sum_{ \nu = -1 }^{ -\infty } \Delta f_{ \nu } \HeavisideFunPlus( t_{ \nu } - t )
  + \sum_{ \nu = 0 }^{ \infty } \Delta f_{ \nu } \HeavisideFunPlus( t - t_{ \nu } ),
\end{equation}
zawiera pewną subtelność, którą warto wyjaśnić. \Dok

\vspace{\spaceFour}





\start \Str{74} Fakt, że~iloraz różnicowy
\begin{equation}
  \label{eq:Zemanian-08}
  \frac{ \varphi( t - \Delta ) - \varphi( t ) }{ \Delta }
\end{equation}
dąży do $-\varphi^{ ( 1 ) }( t )$ przy $\Delta \to 0$ wynika od~razu z~definicji
pochodnej i~twierdzenia o~składaniu granic\footnote{Udowodnij i~zapisz
  gdzieś to twierdzenie.}, problem jest tylko taki, że~jest to
zbieżność punktowa. Aby zaś skorzystać z~ciągłości dystrybucji
potrzebujemy udowodnić zbieżność jednostajną tego ilorazy
do~$-\varphi^{ ( 1 ) }( t )$ oraz jednostajną zbieżność jego $k$-pochodnej
do~$-\varphi^{ ( k + 1 ) }( t )$. Inaczej mówiąc dla $k = 0, 1, 2, \ldots$
musi zachodzić
\begin{equation}
  \label{eq:Zemanian-09}
  \lim_{ \Delta \to 0 }
  \frac{ \varphi^{ ( k ) }( t - \Delta ) - \varphi^{ ( k ) }( t ) }{ \Delta } =
  -\varphi^{ ( k + 1 ) }( t ),
\end{equation}
gdzie zbieżność jest rozumiana jako zbieżność jednostajna.

Dowód podany przez Zemaniana nie jest jedynym możliwym, ale~jest
bezpośredni rachunkowo, przez co stosunkowo łatwy do~zrozumienia.





% ##################
\newpage

\CenterBoldFont{Błędy}


\begin{center}

  \begin{tabular}{|c|c|c|c|c|}
    \hline
    & \multicolumn{2}{c|}{} & & \\
    Strona & \multicolumn{2}{c|}{Wiersz} & Jest
                              & Powinno być \\ \cline{2-3}
    & Od góry & Od dołu & & \\
    \hline
    12  & &  6 & wartości & tylko rzeczywiste \\
    14  &  6 & & całka & całka przedstawiająca $\varphi_{ \alpha }'( t )$ \\
    16  & &  6 & ciągi & funkcje \\
    17  & &  2 & \textit{przestrzeni} & \textit{funkcjonału} \\
    17  & &  3 & która ma & które mają \\
    17  & &  4 & z przestrzeni & na przestrzeni \\
    18  &  6 & & $\{ \varphi_{ \nu } \}\nu \to \infty$
           & $\{ \varphi_{ \nu } \}$, przy $\nu \to \infty$ \\
    18  & & 10 & dystrybuantę & dystrybucję \\
    21  &  9 & & $\Dcal$ & $\Dcal'$ \\
    25  & &  1 & dąży & dąży punktowo \\
    26  &  5 & & pochodnym od & pochodną \\
    27  &  7 & & $\delta^{ ( 1 ) }\boldsymbol{ ( t } )$ & $\delta^{ ( 1 ) }( t )$ \\
    27  & &  3 & $\varphi^{ ( k \boldsymbol{ ) } }\boldsymbol{ ( t ) } $
           & $\varphi^{ ( k ) }( t )$ \\
    27  & &  9 & $< f,\, \varphi >$ & $\langle f, \, \varphi \rangle$ \\
    28  & &  4 & $\delta^{ ( 2 ) }( 0 )$ & $\varphi^{ ( 2 ) }( 0 )$ \\
    28  & & 17 & dążą & jednocześnie dążą \\
    29  & &  8 & $t = 0$ & $t \neq 0$ \\
    31  &  6 & & $\varphi( )$ & $\varphi( t )$ \\
    31  &  8 & & $\Pf \, | t |^{ \beta } \; 1_{ + }( t )$
           & $\Pf \, | t |^{ \beta } \; 1_{ + }( -t )$ \\
    33  & &  1 & skończoną & nieskończoną \\
    34  & &  1 & $\varphi( 0 ) \log \epsilon$ & $\varphi( 0 ) \log \epsilon$. \\
    35 & 1 & & tj.~$\langle B, \varphi \rangle$. Wartość & Wartość \\
    36 & & 7 & $\partial t_{ 2 }^{ \: k_{ 2 } } \ldots \partial t_{ n }^{ \: k_{ n } }$
           & $\partial t_{ 2 }^{ k_{ 2 } } \ldots \partial t_{ n }^{ k_{ n } }$ \\
    37 & & 1 & wyboru układu & układu \\
    37 & & 16 & \textit{mamy $\varphi_{ \alpha }( t )$ równe}
           & \textit{$\varphi_{ \alpha }( t )$ jest równa} \\
    38  &  4 & & $\{ 0, 0, \ldots, 0 ) \}$ & $\{ 0, 0, \ldots, 0 \}$) \\
    41  & &  2 & dystrybucji & dystrybucji regularnej \\
    45  & & 10 & $\boldsymbol{f}$ & $f$ \\
    50  & 13 & & \textit{jedną} & \textit{stałą} \\
    % 50 & 13 & & \textit{jedną} & \textit{jedną z~nich} \\
    50  & &  3 & $C$ & $\Cfrak$ \\
    55  &  9 & & zbieżny w~$\Dcal'$ & zbieżny \\
    59  &  3 & & $\theta_{ \nu }$ & $\theta$ \\
    64  &  6 & & $\nu f( \nu^{ n } t )$ & $\nu^{ n } f( \nu t )$ \\
    71 & & 1 & $\leq M | \nu |^{ -2 }$ & $\leq M | \nu |^{ -1 }$ \\
    71 & & 4 & gdzie $p$ & gdzie $p = k - 1$, a $g( t )$ \\
    71 & & 5 & $M$ i~$k$ stałych & $M$ rzeczywistego i~$k$ naturalnego \\
    % & & & & \\
    % & & & & \\
    % & & & & \\
    % & & & & \\
    % & & & & \\
    % & & & & \\
    % & & & & \\
    % & & & & \\
    % & & & & \\
    \hline
  \end{tabular}

\end{center}


\noindent
\StrWd{21}{10} \\
\Jest  opisem niewykończonego \\
\Powin niepełnym opisem pewnego \\
\StrWg{34}{2} \\
\Jest  Odpowiednie nachylenie wskazuje \\
\Powin Odpowiednią funkcje przedstawia \\
\StrWd{38}{4} \\
\Jest  choć może nie być sprecyzowana jego wartość \\
\Powin choć jego wartość może nie być podana jawnie \\
\StrWd{40}{6} \\
\Jest  wartości liczbowych \\
\Powin wartości liczbowej w~punkcie \\
\StrWg{50}{14} \\
\Jest  \textit{pokrycie $\Rbb^{ n }$.} \\
\Powin \textit{pokrycie $\Rbb^{ n }$, przy czym średnice wszystkich jego
  zbiorów, są~ograniczone przez tą samą stałą, co~dla rodziny
  $\Cfrak_{ T }$.} \\
\StrWd{71}{5} \\
\Jest  \textit{$M$ i $k$ stałych rzeczywistych} \\
\Powin \textit{$M > 0$ i~$k$ naturalnego} \\
\StrWd{71}{4} \\
\Jest  \textit{gdzie $p$} \\
\Powin \textit{gdzie $p = k + 2$, a~$g( t )$} \\


\vspace{\spaceTwo}
% ############################










% ######################################
\newpage

\section{Algebry operatorów i~algebry topologiczne}

\vspace{\spaceTwo}
% ######################################



% ############################
\Work{ % Autorzy i tytuł dzieła
  Ola Bratteli, Derek W.~Robinson \\
  \textit{Operator Algebras and~Quantum Statistical Mechanics, Vol I.} \\
  \textit{$C^{ * }$- and $W^{ * }$-Algebras, Symmetry Groups. Decoposition~of
  State}, \cite{BratteliRobinsonOperatorAlgebrasVolI2002}}


% ##################
\CenterBoldFont{Uwagi do konkretnych stron}


\start \Str{19} Autorzy zapomnieli o~jednym warunku, który jest
potrzebny w~definicji algebry
\begin{equation}
  \label{eq:BR-VolI-s01-01}
  ( A + B ) C = A C + B C.
\end{equation}

\vspace{\spaceFour}



\start \Str{27} Transformacja podana dla $\lambda^{ n } - A^{ n }$ i
wynikające z niej konsekwencje, nie są wystarczająco omówione.

\vspace{\spaceFour}





% ##################
\CenterBoldFont{Błędy}


\begin{center}
  \begin{tabular}{|c|c|c|c|c|}
    \hline
    & \multicolumn{2}{c|}{} & & \\
    Strona & \multicolumn{2}{c|}{Wiersz} & Jest
                              & Powinno być \\ \cline{2-3}
    & Od góry & Od dołu & & \\
    \hline
    21  & & 10 & $ff^{ * }$ & $f^{ * }\! f$ \\
    22  & 12 & & $( \alpha, A^{ * } )$ & $( \alpha, A )^{ * }$ \\
    % & & & & \\
    % & & & & \\
    % & & & & \\
    % & & & & \\
    % & & & & \\
    % & & & & \\
    % & & & & \\
    % & & & & \\
    % & & & & \\
    % & & & & \\
    % & & & & \\
    % & & & & \\
    \hline
  \end{tabular}

\end{center}

\vspace{\spaceTwo}


Str. 31. \ldots the resolvent $R( \lambda ) = ( A - \lambda I )$\ldots



% ############################










% ############################
\newpage

\Work{ % Autor i tytuł dzieła
  Masamichi Takesaki \\
  \textit{Theory of Operator Algebras. Volume I}, cite\{MTTOAI\}}



% ##################
\CenterBoldFont{Błędy}


\begin{center}

  \begin{tabular}{|c|c|c|c|c|}
    \hline
    & \multicolumn{2}{c|}{} & & \\
    Strona & \multicolumn{2}{c|}{Wiersz}& Jest
                              & Powinno być \\ \cline{2-3}
    & Od góry & Od dołu & & \\
    \hline
    5 & & 6 & $x_{ 0 }^{ -1 } + \sum\limits_{ n = 0 }^{ \infty }
              [ x_{ 0 }^{ -1 } ( x_{ 0 } - x ) ]^{ n } x_{ 0 }^{ -1 }$
           & $\sum\limits_{ n = 0 }^{ \infty } [ x_{ 0 }^{ -1 } ( x_{ 0 } - x ) ]^{ n }
             x_{ 0 }^{ -1 }$ \\
    7 & & 9 & $[ ( 1 / \lambda ) x - 1 ]^{ -1 }$
           & $[ 1 - ( 1 / \lambda ) x ]^{ -1 }$ \\
    7 & & 8 & $\frac{ 1 }{ \lambda } \big( \frac{ 1 }{ \lambda } - x \big)^{ -1 }$
           & $-\frac{ 1 }{ \lambda } \big( 1 - \frac{ 1 }{ \lambda } x \big)^{ -1 }$ \\
    7 & & 3 & $\sum\limits_{ n = 0 }^{ \infty } ( 1 / \lambda^{ n + 1 } ) x^{ n }$
           & $-\sum\limits_{ n = 0 }^{ \infty } ( 1 / \lambda^{ n + 1 } ) x^{ n }$ \\
    8 & 3 & & $( \lambda - \lambda_{ 0 } )^{ n } f( \lambda_{ 0 } )^{ n + 1 }$
           & $-( \lambda_{ 0 } - \lambda )^{ n } f( \lambda_{ 0 } )^{ n + 1 }$ \\
    & & & & \\
    \hline
  \end{tabular}

\end{center}

\vspace{\spaceTwo}



% ############################










% #####################################################################
% #####################################################################
% Bibliografia

\bibliographystyle{plalpha}

\bibliography{MathComScienceBooks}{}





% ############################

% Koniec dokumentu
\end{document}
