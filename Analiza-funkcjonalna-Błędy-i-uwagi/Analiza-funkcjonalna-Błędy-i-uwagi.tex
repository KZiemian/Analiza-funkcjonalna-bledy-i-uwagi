% ---------------------------------------------------------------------
% Basic configuration and packages
% ---------------------------------------------------------------------
% Package for discovering wrong and outdated usage of LaTeX.
% More information to be found in l2tabu English version.
\RequirePackage[l2tabu, orthodox]{nag}
% Class of LaTeX document: {size of paper, size of font}[document class]
\documentclass[a4paper,11pt]{article}



% ---------------------------------------
% Packages not tied to particular normal language
% ---------------------------------------
% This package should improved spaces in the text.
\usepackage{microtype}
% Add few important symbols, like text Celcius degree
\usepackage{textcomp}



% ---------------------------------------
% Polonization of LaTeX document
% ---------------------------------------
% Basic polonization of the text
\usepackage[MeX]{polski}
% Switching on UTF-8 encoding
\usepackage[utf8]{inputenc}
% Adding font Latin Modern
\usepackage{lmodern}
% Package is need for fonts Latin Modern
\usepackage[T1]{fontenc}



% ---------------------------------------
% Setting margines
% ---------------------------------------
% Package for easy settings of margins. Unit of measurement is inch.
\usepackage{vmargin}
\setmarginsrb
{ 0.7in}  % left margin
{ 0.6in}  % top margin
{ 0.7in}  % right margin
{ 0.8in}  % bottom margin
{  20pt}  % head height
{0.25in}  % head sep
{   9pt}  % foot height
{ 0.3in}  % foot sep



% ---------------------------------------
% Setting vertical spaces in the text
% ---------------------------------------
% Setting space between lines
\renewcommand{\baselinestretch}{1.1}

% Setting space between lines in tables
\renewcommand{\arraystretch}{1.4}



% ---------------------------------------
% Packages for scientific papers
% ---------------------------------------
% Switching off \lll symbol, that I guess is representing letter ``Ł''.
% It collide with `amsmath' package's command with the same name
\let\lll\undefined
% Basic package from American Mathematical Society (AMS)
\usepackage[intlimits]{amsmath}
% Other very useful packages from AMS
\usepackage{amsfonts, amssymb, amscd, amsthm}
% Better looking calligraphy fonts
\usepackage{calrsfs}

% Better looking greek letters
% Example of use: pi -> \uppi
\usepackage{upgreek}
% Improving look of lambda letter
\let\oldlambda\Lambda
\renewcommand{\lambda}{\uplambda}





% ---------------------------------------
% Defining new environments (?)
% ---------------------------------------
% Defining enviroment ``Wniosek''
\newtheorem{corollary}{Wniosek}
\newtheorem{definition}{Definicja}
\newtheorem{theorem}{Twierdzenie}





% ------------------------------
% Private packages
% You need to put them in the same directory as .tex file
% ------------------------------
% Contains various command useful for working with a text
\usepackage{latexgeneralcommands}
% Contains definitions useful for working with mathematical text
\usepackage{mathcommands}

% Package for use in text about functional analysis
\usepackage{functionalanalysiscommands}





% ------------------------------
% Package ``hyperref''
% They advised to put it on the end of preambule
% ------------------------------
% It allows you to use hyperlinks in the text
\usepackage{hyperref}










% ---------------------------------------------------------------------
% Defining title and author of the text
\title{Analiza funkcjonalna \\
  {\Large Błędy i~uwagi}}

\author{Kamil Ziemian, korekta Wojciech Dyba}


% \date{}
% ---------------------------------------------------------------------










% ####################################################################
\begin{document}
% ####################################################################





% ######################################
% Title of the text
\maketitle
% ######################################





% ######################################
\section{A. V. Balakrishnan
  \textit{Analiza funkcjonalna stosowana},
  \cite{BalakrishnanAnalizaFunkcjonalnaStosowana1992}}
% ######################################




% ##################
\CenterBoldFont{Błędy}


\begin{center}

  \begin{tabular}{|c|c|c|c|c|}
    \hline
    & \multicolumn{2}{c|}{} & & \\
    Strona & \multicolumn{2}{c|}{Wiersz} & Jest
                              & Powinno być \\ \cline{2-3}
    & Od góry & Od dołu & & \\
    \hline
    13  & & 10 & $\int^{ 1 }_{ -1 }$ & $2 \int^{ 1 }_{ -1 }$ \\
    14  & 15 & & $<$ & $\leq$ \\
    15  & 18 & & otrzymywaliśmy & otrzymalibyśmy \\
    % & & & & \\
    \hline
  \end{tabular}

\end{center}

\VerSpaceSix



% ############################










% ######################################
\section{Anton Deitmar, Siegrfied Echterhoff
  \textit{Principles~of Harmonic Analysis},
  \cite{DeitmarEcherhoffPrinciplesOfHarmonicAnalysis2009}}
% ######################################


% ##################
\CenterBoldFont{Uwagi}


\noindent
\textbf{Str. 42. Twierdzenie 2.2.6. Wzór na~promień spektralny.}
Aby~wykazać, że~$\lambda^{ n } \in \sigma_{ \Acal }( a^{ n })$
należy pokazać, iż~element
\begin{equation}
  \label{eq:DE-1}
  ( \lambda 1 - a ) \sum_{ j = 0 }^{ n - 1 } \lambda^{ j } a^{ n - 1 - j }
\end{equation}
jest nieodwracalny. Nie jest to jednak dla~mnie oczywiste. Arkadiusz
Bochniak powiedział, żeby zobaczyć na~dowód twierdzenia widmie
wielomian operatora (ang. \textit{polynomial spectral mapping
  theorem}). Należy przy tym zauważyć, że~z~tego twierdzenie wynika
od~razu poszukiwana własność.

\VerSpaceFour





\noindent
\textbf{Str. 46. Lemat 2.4.2.} W~dowodzie tego twierdzenie trzeba
chyba rozumować następująco. Chcemy pokazać,
że~$m( a ) \leq \norm{ a }$ dla każdego $a \in \Acal$.

Rozpatrujemy dwa przypadki. 1) Gdy $m( a ) = 0$, wtedy oczywiście
$m( a ) = 0 \leq \norm{ a }$. 2). Gdy $m( a ) \neq 0$. Wówczas
$m( a - m( a ) 1 ) = 0$. Gdyby istniał element $b$~odwrotny
do~$a - m( a ) 1$, to~zachodziłoby
\begin{equation}
  \label{eq:DE-2}
  1 = m( 1 ) = m( b ( a - m( a ) 1 ) ) = m( b ) m( a - m( a ) ) =
  m( b ) 0 = 0,
\end{equation}
co jest niemożliwe. Dalej można już postępować jak w~zaprezentowanym
dowodzie.





% ##################
\CenterBoldFont{Błędy}


\begin{center}

  \begin{tabular}{|c|c|c|c|c|}
    \hline
    & \multicolumn{2}{c|}{} & & \\
    Strona & \multicolumn{2}{c|}{Wiersz} & Jest
                              & Powinno być \\ \cline{2-3}
    & Od góry & Od dołu & & \\
    \hline
    42  & 10 & & $s\: u\: p$ & $\sup$ \\
    44  &  9 & & Banach-Algebras & Banach algebras \\
    % & & & & \\
    % & & & & \\
    \hline
  \end{tabular}

\end{center}

\VerSpaceSix


% ############################










% ######################################
\section{I.M. Gelfand, G.E. Shilov \\
  \textit{Generalized Functions: Volume~I, Properties
    and~Operations},
  \cite{GelfandShilovGeneralizedFunctionsVolI1964}}
% ######################################


% ##################
\CenterBoldFont{Błędy}


\begin{center}

  \begin{tabular}{|c|c|c|c|c|}
    \hline
    & \multicolumn{2}{c|}{} & & \\
    Strona & \multicolumn{2}{c|}{Wiersz} & Jest
                              & Powinno być \\ \cline{2-3}
    & Od góry & Od dołu & & \\
    \hline
    12  &  4 & & $f_{ 0 } = 0$ & $x_{ 0 } = 0$ \\
    28  & 16 & & $r \leq a$ & $r \geq a$ \\
    % & & & & \\
    \hline
  \end{tabular}

\end{center}

\VerSpaceSix



% ############################










% ######################################
\section{Walter Rudin \textit{Analiza funkcjonalna},
  \cite{RudinAnalizaFunkcjonalna2012}}
% ######################################



% ##################
\CenterBoldFont{Uwagi}


\noindent
W~tych notatkach, tak jak i~w~samej książce, będziemy domyślnie
zakładać, że~wszystkie przestrzenie wektorowe są albo nad ciałem liczb
rzeczywistych~$\Rbb$ albo liczby zespolonych $\Cbb$. Zagadnienie które
wyniki można uogólnić na ciała takie jak $\Zbb_{ p }$, gdzie $p$~jest liczbą
pierwszą, jest bardzo ciekawe, lecz w~tym momencie nie jest na tyle ważny
byśmy mogli się nim zajmować.

\VerSpaceFour





% ##################
\CenterBoldFont{Uwagi do konkretnych stron}


\noindent
\Str{17}

\VerSpaceFour





\noindent
\Str{18}

\VerSpaceFour





\noindent
\Str{20}

\VerSpaceFour





\noindent
\Str{22}

\VerSpaceFour





\noindent
\Str{}

\VerSpaceFour





\noindent
\Str{}

\VerSpaceFour





\noindent
\Str{}

\VerSpaceFour





\noindent
\Str{}

\VerSpaceFour





\noindent
\Str{}

\VerSpaceFour





\noindent
\Str{}

\VerSpaceFour





\noindent
\Str{}

\VerSpaceFour





\noindent
\Str{}

\VerSpaceFour





% \start \Str{}

% \vspace{\spaceFour}





% \start \Str{}

% \vspace{\spaceFour}





% \start \Str{}

% \vspace{\spaceFour}



% \start \Str{}

% \vspace{\spaceFour}





% \start \Str{}

% \vspace{\spaceFour}





% \start \Str{}

% \vspace{\spaceFour}





% \start \Str{}

% \vspace{\spaceFour}





% \start \Str{}





% ##################
\newpage

\CenterBoldFont{Błędy}


\begin{center}

  \begin{tabular}{|c|c|c|c|c|}
    \hline
    Strona & \multicolumn{2}{c|}{Wiersz} & Jest
                              & Powinno być \\ \cline{2-3}
    & Od góry & Od dołu & & \\
    \hline
    % & &   &  &  \\
    % &   & &  &  \\
    % & &   &  &  \\
    % & &   &  &  \\
    % & &   &  &  \\
    % & &   &  &  \\
    % &   & & &  \\
    % & &  &  &  \\
    % &   & &  &  \\
    % & &   &  &  \\
    % &   & &  &  \\
    % &   & &  &  \\
    % & &   &  &  \\
    % & &   & &  \\
    % & &  &  &  \\
    % & &   &  &  \\
    % & &   &  &  \\
    % &   & &  &  \\
    % &   & & &  \\
    % & &  &  &  \\
    % & &   &  &  \\
    % &  & &  &  \\
    % & &  & &  \\
    % & &  & &  \\
    % & &  &  &  \\
    % &   & &  &  \\
    % & &   &  &  \\
    % & &  &  &  \\
    % &  & &  &  \\
    %  &  & &  &  \\
    %  & &   &  &  \\
    %  &   & &  &  \\
    %  &   & &  &  \\
    %  &   & &  &  \\
    % & &  &  &  \\
    % & &  &  &  \\
    % & &  &  &  \\
    % & & & & \\
    % & & & & \\
    % & & & & \\
    % & & & & \\
    % & & & & \\
    % & & & & \\
    % & & & & \\
    % & & & & \\
    % & & & & \\
    \hline
  \end{tabular}

\end{center}

\VerSpaceSix


\noindent
% \StrWd{}{} \\
% \Jest   \\
% \Powin  \\
% \StrWg{}{} \\
% \Jest   \\
% \Powin  \\
% \StrWd{}{} \\
% \Jest   \\
% \Powin  \\
% \StrWd{}{} \\
% \Jest   \\
% \Powin  \\
% \StrWg{}{} \\
% \Jest   \\
% \Powin  \\
% \StrWd{}{} \\
% \Jest   \\
% \Powin  \\
% \StrWd{}{} \\
% \Jest   \\
% \Powin  \\



% ############################




















% ######################################
\section{Algebry operatorów i~algebry topologiczne}
% ######################################













% ############################
\newpage

{ % Autor i tytuł dzieła
  Masamichi Takesaki \\
  \textit{Theory of Operator Algebras. Volume I}, cite\{MTTOAI\}}



% ##################
\CenterBoldFont{Błędy}


\begin{center}

  \begin{tabular}{|c|c|c|c|c|}
    \hline
    & \multicolumn{2}{c|}{} & & \\
    Strona & \multicolumn{2}{c|}{Wiersz}& Jest
                              & Powinno być \\ \cline{2-3}
    & Od góry & Od dołu & & \\
    \hline
    5 & & 6 & $x_{ 0 }^{ -1 } + \sum\limits_{ n = 0 }^{ \infty }
              [ x_{ 0 }^{ -1 } ( x_{ 0 } - x ) ]^{ n } x_{ 0 }^{ -1 }$
           & $\sum\limits_{ n = 0 }^{ \infty } [ x_{ 0 }^{ -1 } ( x_{ 0 } - x ) ]^{ n }
             x_{ 0 }^{ -1 }$ \\
    7 & & 9 & $[ ( 1 / \lambda ) x - 1 ]^{ -1 }$
           & $[ 1 - ( 1 / \lambda ) x ]^{ -1 }$ \\
    7 & & 8 & $\frac{ 1 }{ \lambda } \big( \frac{ 1 }{ \lambda } - x \big)^{ -1 }$
           & $-\frac{ 1 }{ \lambda } \big( 1 - \frac{ 1 }{ \lambda } x \big)^{ -1 }$ \\
    7 & & 3 & $\sum\limits_{ n = 0 }^{ \infty } ( 1 / \lambda^{ n + 1 } ) x^{ n }$
           & $-\sum\limits_{ n = 0 }^{ \infty } ( 1 / \lambda^{ n + 1 } ) x^{ n }$ \\
    8 & 3 & & $( \lambda - \lambda_{ 0 } )^{ n } f( \lambda_{ 0 } )^{ n + 1 }$
           & $-( \lambda_{ 0 } - \lambda )^{ n } f( \lambda_{ 0 } )^{ n + 1 }$ \\
    & & & & \\
    \hline
  \end{tabular}

\end{center}

\VerSpaceSix



% ############################










% #####################################################################
% #####################################################################
% Bibliography

\bibliographystyle{plalpha}

\bibliography{MathematicsBooks}{}





% ############################

% End of the document
\end{document}
