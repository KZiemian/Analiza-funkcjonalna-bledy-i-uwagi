% ------------------------------------------------------------------------------------------------------------------
% Basic configuration and packages
% ------------------------------------------------------------------------------------------------------------------
% Package for discovering wrong and outdated usage of LaTeX.
% More information to be found in l2tabu English version.
\RequirePackage[l2tabu, orthodox]{nag}
% Class of LaTeX document: {size of paper, size of font}[document class]
\documentclass[a4paper,11pt]{article}



% ------------------------------------------------------
% Packages not tied to particular normal language
% ------------------------------------------------------
% This package should improved spaces in the text.
\usepackage{microtype}
% Add few important symbols, like text Celcius degree
\usepackage{textcomp}



% ------------------------------------------------------
% Polonization of LaTeX document
% ------------------------------------------------------
% Basic polonization of the text
\usepackage[MeX]{polski}
% Switching on UTF-8 encoding
\usepackage[utf8]{inputenc}
% Adding font Latin Modern
\usepackage{lmodern}
% Package is need for fonts Latin Modern
\usepackage[T1]{fontenc}



% ------------------------------------------------------
% Setting margins
% ------------------------------------------------------
\usepackage[a4paper, total={14cm, 25cm}]{geometry}



% ------------------------------------------------------
% Setting vertical spaces in the text
% ------------------------------------------------------
% Setting space between lines
\renewcommand{\baselinestretch}{1.1}

% Setting space between lines in tables
\renewcommand{\arraystretch}{1.4}



% ------------------------------------------------------
% Packages for scientific papers
% ------------------------------------------------------
% Switching off \lll symbol, that I guess is representing letter ``Ł''.
% It collide with `amsmath' package's command with the same name
\let\lll\undefined
% Basic package from American Mathematical Society (AMS)
\usepackage[intlimits]{amsmath}
% Equations are numbered separately in every section.
\numberwithin{equation}{section}

% Other very useful packages from AMS
\usepackage{amsfonts}
\usepackage{amssymb}
\usepackage{amscd}
\usepackage{amsthm}

% Package with better looking calligraphy fonts
\usepackage{calrsfs}

% Package with better looking greek letters
% Example of use: pi -> \uppi
\usepackage{upgreek}
% Improving look of lambda letter
\let\oldlambda\Lambda
\renewcommand{\lambda}{\uplambda}




% ------------------------------------------------------
% BibLaTeX
% ------------------------------------------------------
% Package biblatex, with biber as its backend, allow us to handle
% bibliography entries that use Unicode symbols outside ASCII.
\usepackage[
language=polish,
backend=biber,
style=alphabetic,
url=false,
eprint=true,
]{biblatex}

\addbibresource{ReedSimonMethodsOfModernETCVolIBibliography.bib}





% ------------------------------------------------------
% Defining new environments (?)
% ------------------------------------------------------
% Defining enviroment ``Wniosek''
\newtheorem{corollary}{Wniosek}
\newtheorem{definition}{Definicja}
\newtheorem{theorem}{Twierdzenie}





% ------------------------------------------------------
% Private packages
% You need to put them in the same directory as .tex file
% ------------------------------------------------------
% Contains various command useful for working with a text
\usepackage{latexgeneralcommands}
% Contains definitions useful for working with mathematical text
\usepackage{mathcommands}





% ------------------------------------------------------
% Package "hyperref"
% They advised to put it on the end of preambule
% ------------------------------------------------------
% It allows you to use hyperlinks in the text
\usepackage{hyperref}










% ------------------------------------------------------------------------------------------------------------------
% Title and author of the text
\title{Michael Reed, Barry Simon \textit{Methods~of Modern Mathematical
    Physics},
  \parencite{Reed-Simon-Methods-of-modern-mathematical-physics-Vol-I-Pub-1980} \\
  {\Large Błędy i~uwagi}}

\author{Kamil Ziemian, korekta Wojciech Dyba}


% \date{}
% ------------------------------------------------------------------------------------------------------------------










% ####################################################################
% Beginning of the document
\begin{document}
% ####################################################################





% ######################################
\maketitle
% ######################################





% ######################################
\section{Uwagi ogólne}

\label{sec:-ETC-Uwagi-ogolne}
% ######################################





% ######################################
\section{Uwagi do poszczególnych rozdziałów}

\label{sec:-ETC-Uwagi-ogolne}
% ######################################



% ######################################
\section{Uwagi do rozdziału~I}

\label{sec:-ETC-Uwagi-ogolne}
% ######################################


\noindent
Nigdzie nie została podana definicja $\sup$ i~$\inf$ ani
w~przypadku ogólnych relacji porządkujących, ani w~przypadku liczb
rzeczywistych. % Sprawdzone.

\VerSpaceFour





\noindent
Nie~zdefiniowano zbiorów zwartych, ani~nie~wskazano, że~zbiór
domknięty i~ograniczony w $\Rbb^{ n }$ jest zwarty.

\VerSpaceFour





\noindent
Nie podana została definicja sumy miar borelowskich,
ani~nie~zostało udowodnione, że funkcja całkowalna osobno względem
miar $\mu_{ 1 }$ i~$\mu_{ 2 }$ jest całkowalna względem ich sumy.
Uzupełnienie wykładu w~tym miejscu nie jest trudne i~przedstawiamy je
poniżej.

Dla dowolnych dwóch miar dodatnich $\mu_{ 1 }$ i~$\mu_{ 2 }$,
niekoniecznie borelowskich, określonym na wspólnym $\sigma$-pierścieniu $\Rcal$, definiujemy \textbf{sumę miar} jako:
\begin{equation}
  \label{eq:RS-VolI-s0-01}
  \begin{split}
    ( \mu_{ 1 } + \mu_{ 2 } )( A )
    &=
      \mu_{ 1 }( A ) + \mu_{ 2 }( A ), \quad
      \forall A \in \Rcal, \\
    ( \mu_{ 1 } + \mu_{ 2 } )( A )
    &= +\infty, \quad
      \textrm{jeśli } \mu_{ 1 }( A ) = +\infty \textrm{ lub }
      \mu_{ 2 }( A ) = +\infty.
  \end{split}
\end{equation}
Analogicznie określmy sumę $n$ miar dodatnich. Trzeba teraz pokazać,
że~jest to w~istocie miara.





% #############
\begin{theorem}

  Suma skończonej liczby miar dodatnich $\mu_{ i }$, $i = 1, \ldots, n$,
  określonych na wspólnym pierścieniu $\Rcal$, jest miarą dodatnią
  określoną na tym samym pierścieniu.

  Jeżeli funkcja $f$ jest całkowalna względem względem wszystkich miar
  $\mu_{ i }$, to jest też całkowalna względem ich sumy i~zachodzi:
  \begin{equation}
    \label{eq:RS-Vol-I-s01-02}
    \int f( x ) \, \left( \sum\limits_{ i = 1 }^{ n } d\mu_{ i }( x ) \right)
    = \sum\limits_{ i = 1 }^{ n } \int f( x ) \, d\mu_{ i }( x ).
  \end{equation}

\end{theorem}



\begin{proof}

  Jeśli pokażemy to twierdzenie dla $n = 2$, przypadku ogólnego można
  będzie można dowieść przez prostą indukcję. Suma miar jest określona
  na~wspólnym pierścieniu $\Rcal$, zaś poza warunkiem na~przeliczalną
  addytywność miary rozłącznych zbiorów $A_{ i }$, pozostałe własności
  miary dodatniej wychodzą od razu.

  Załóżmy teraz, że~miary wszystkich tych zbiorów względem $\mu_{ 1 }$
  i~$\mu_{ 2 }$ są skończone. Na podstawie podstawowych twierdzeń
  o~sumowaniu szeregów dodatnich, mamy
  \begin{equation}
    \label{eq:RS-Vol-I-s01-03}
    \begin{split}
      ( \mu_{ 1 } + \mu_{ 2 } )\left( \bigcup_{ i = 1 }^{ \infty } A_{ i } \right)
      &=
        \mu_{ 1 }\left( \bigcup_{ i = 1 }^{ \infty } A_{ i } \right)
        + \mu_{ 2 }\left( \bigcup_{ i = 1 }^{ \infty } A_{ i } \right)
        =
        \sum_{ i = 1 }^{ \infty } \mu_{ 1 }( A_{ i } )
        + \sum_{ i = 1 }^{ \infty } \mu_{ 2 }( A_{ i } ) = \\
      &=
        \sum_{ i = 1 }^{ \infty } \left( \mu_{ 1 }( A_{ i } ) + \mu_{ 2 }( A_{ i } )
        \right)
        = \sum_{ i = 1 }^{ \infty } ( \mu_{ 1 } + \mu_{ 2 } )( A_{ i } ).
    \end{split}
  \end{equation}
  Jeśli chodź dla jednego zbioru zachodzi
  $\mu_{ n }( A_{ i } ) = +\infty$, to wówczas obie strony są równe
  $+\infty$. Problem, czy można dodać dwie miary, które nie są
  dodatnie wymaga osobnych rozważań, ten problem jednak nie powinien
  być dla nas istotny.

  Rozpatrzmy teraz problem całkowania względem sumy miar $\mu_{ i }$
  funkcji $f$, która jest całkowalnej względem każdej miar z~osobna,
  ograniczymy~się przy tym dla przypadku funkcji określonej na~$\Rbb$.
  Jeżeli $f \geq 0$, wówczas zachodzi równość:
  \begin{equation}
    \label{eq:RS-Vol-I-s01-04}
    \begin{split}
      \sum_{ n }( f )_{ \mu_{ 1 } + \mu_{ 2 } }
      &=
        \sum_{ m = 0 }^{ \infty } \frac{ m }{ n } ( \mu_{ 1 } + \mu_{ 2 } )
        \left( f^{ -1 }\left[ \left[ \tfrac{ m }{ n },
        \tfrac{ m + 1 }{ n } \right) \right] \right) = \\
      &=
        \sum_{ m = 0 }^{ \infty } \left\{ \frac{ m }{ n } \mu_{ 1 }\left(
        f^{ -1 }\left[ \left[ \tfrac{ m }{ n }, \tfrac{ m + 1 }{ n }
        \right) \right] \right) \right.
        + \left. \frac{ m }{ n } \mu_{ 2 }\left(
        f^{ -1 }\left[ \left[ \tfrac{ m }{ n }, \tfrac{ m + 1 }{ n }
        \right) \right] \right) \right\} = \\
      &=
        \sum_{ n }( f )_{ \mu_{ 1 } } + \sum_{ n }( f )_{ \mu_{ 2 } }.
    \end{split}
  \end{equation}
  Znowu ponieważ wszystkie wyrazy szeregów są dodatnie, więc powyższe
  przekształcenia są poprawne.

  Jako, że~w~podejściu prezentowanym przez Reeda i~Simona, całkowanie
  funkcji zespolonych sprowadza~się do całkowania funkcji dodatnich,
  widzimy więc, że~funkcja całkowalna względem każdej z~miar
  $\mu_{ i }$, $i = 1, \ldots, n$~tzn.
  \begin{equation}
    \label{eq:RS-Vol-I-s01-05}
    \int f \, d\mu_{ 1 } < +\infty \; \textrm{  oraz  }
    \int f \, \mu_{ 2 } < +\infty,
  \end{equation}
  jest też całkowalna względem ich sumy i~na podstawie wzoru
  \eqref{eq:RS-Vol-I-s01-05}, zachodzi równość:
  \begin{equation}
    \label{eq:RS-Vol-I-s01-06}
    \int f \, d( \mu_{ 1 } + \mu_{ 2 } )
    = \int f \, d\mu_{ 1 }  + \int f \, d\mu_{ 2 }.
  \end{equation}

\end{proof}
% #############





\textbf{Uwaga.} Należy tą sytuację odróżnić od~przypadku, gdy mamy dwie
funkcje $f_{ 1 }$ i~$f_{ 2 }$, całkowalne odpowiednio względem miar
$\mu_{ 1 }$ i~$\mu_{ 2 }$, i~chcemy powiedzieć coś o~całkowalności ich
sumy. Wtedy bowiem może okazać się, że~funkcja np.~$f_{ 1 }$ jest
całkowalna względem $\mu_{ 1 }$ bo~miara zbioru na którym przyjmuje
duże wartości jest mała względem $\mu_{ 1 }$. Jeżeli jednak miara
$\mu_{ 2 }$ tego zbioru jest duża, funkcja $f_{ 1 }$ może być względem
niej niecałkowalna.

\textbf{Przykład.} Weźmy funkcję na prostej rzeczywistej
\begin{equation}
  \label{eq:RS-Vol-I-s01-07}
  f( x ) =
  \begin{cases}
    +\infty, \quad x = 0. \\
    \quad \;0, \quad x \neq 0, \\
  \end{cases}
\end{equation}
Jako, że~$f \equiv 0$, jest ona całkowalna względem miary Lebesgue'a
$dx$ i~jej całka wynosi~0. Jeżeli teraz dodamy do miary Lebesgue'a
miarę Diraca skupioną w~0, oznaczmy ją $d\delta_{ 0 }$, to $f$ będzie
niecałkowalna względem miary $dx + d\delta_{ 0 }$. % Sprawdzone.










% ######################################
\section{Uwagi do rozdziału~I}

\label{sec:-ETC-Uwagi-ogolne}
% ######################################


\noindent
\textbf{Ciągłość ciągowa funkcji na~przestrzeniach metrycznych.}
W~tym rozdziale napotykamy kilkakrotnie następującą sytuacje. Mamy
funkcje ciągłą $f( x )$ oraz~ciąg $x_{ n_{ 1 }, \ldots, \, n_{ m } }$
i~chcemy wejść z~granicą pod funkcję, innymi słowy potrzebujemy, aby
zachodziło\footnote{Rozpatrujemy tu tylko przypadek skończonej liczby
  indeksów będących liczbami naturalnymi, bo~na~razie nie znalazłem
  potrzeby rozpatrywać ciągów o~nieskończonej ilości indeksów
  lub~takich które nie są indeksowane liczbami naturalnymi.}
\begin{equation}
  \label{eq:RS-Vol-I-s01-08}
  \lim\limits_{ \substack{ n_{ 1 } \to +\infty \\
      \cdots \\
      n_{ m } \to +\infty } }
  f( x_{ n_{ 1 }, \ldots, \, n_{ m } } ) =
  f( \lim_{ \substack{ n_{ 1 } \to +\infty \\
      \cdots \\
      n_{ m } \to +\infty } }
  x_{ n_{ 1 }, \ldots, \, n_{ m } } ).
\end{equation}
Jeżeli funkcja jest \textit{ciągła topologicznie}, czyli przeciwobraz
każdego zbioru otwartego jest otwarty, to poniżej udowodnimy, że~ta
równość zachodzi. Pytanie czy jeśli funkcja jest \textit{ciągowo
  ciągła}, czyli
\begin{equation}
  \label{eq:RS-Vol-I-s01-09}
  \lim\limits_{ n \to +\infty } f( x_{ n } ) =
  f( \lim\limits_{ n \to +\infty } x_{ n } ),
\end{equation}
dla każdego zbieżnego ciągu $x_{ n }$, to~czy ta własność będzie
zachodzić również dla ciągów wieloindeksowych? W~przypadku ogólnym nie
wiem, czy tak jest, bowiem gdy przestrzeń topologiczna nie~ma
przeliczalnej bazy otoczeń, to~ciągi postaci $x_{ n }$ nie zawierają
pełnej wiedzy o~topologii. Jednak w~najbardziej nas interesującym
przypadku przestrzeni metrycznych, pożądana równość istotnie zachodzi.

Na początku będzie nam potrzebna definicja granicy ciągu
wieloindeksowego. Najprostsza definicja tej granicy jest
następująca\footnote{Granica ciągu wieloindeksowego powinna być
  szczególnym przypadkiem granic indeksowanych zbiorami skierowanymi,
  które~są omawiane w~książce Maurina
  \parencite{Maurin-Analiza-Vol-I-Pub-1974}.}. Niech
$X$~będzie przestrzenią metryczną, a~$x_{ n_{ 1 }, \ldots, \, n_{ m } }$
ciągiem wieloindeksowym w~$X$. $x_{ 0 }$ jest granicą ciągu
$x_{ n_{ 1 }, \ldots, \, n_{ m } }$ wtedy i~tylko wtedy, gdy
\begin{equation}
  \label{eq:RS-Vol-I-s01-10}
  ( \forall \, \varepsilon > 0) ( \exists \, N )
  ( \forall \, n_{ 1 } > N, \ldots, \, n_{ m } > N ) \;\,
  d( x_{ n_{ 1 }, \ldots, \, n_{ m } }, \, x_{ 0 } ) < \varepsilon.
\end{equation}





% #############
\begin{theorem}

  Niech $X$ i~$Y$ będą przestrzeniami metrycznymi,
  $x_{ n_{ 1 }, \ldots, n_{ m } }$ ciągiem elementów przestrzeni $X$.
  Jeśli funkcja $f: X \to Y$ jest ciągła topologicznie i~istnieje
  granica ciągu
  $\lim_{ \, n_{ 1 } \to +\infty, \, \ldots, \, n_{ m } \to +\infty } \, x_{ n_{ 1 }, \, \ldots, \, n_{ m } } = x_{ 0 }$, to

  \begin{equation}
    \label{eq:RS-Vol-I-s01-11}
    \lim\limits_{ \substack{ n_{ 1 } \to +\infty \\
        \cdots \\
        n_{ m } \to +\infty } }
    f( x_{ n_{ 1 }, \ldots, \, n_{ m } } )
    = f( \lim\limits_{ \substack{ n_{ 1 } \to +\infty \\
        \cdots \\
        n_{ m } \to +\infty } } x_{ n_{ 1 }, \ldots, \, n_{ m } } ).
  \end{equation}

\end{theorem}



\begin{proof}

  Dla uproszczenia notacji będziemy rozpatrywać ciąg dwuindeksowy
  $x_{ n, \, m }$. Ustalmy $\varepsilon > 0$, szukamy teraz takiego $N$,
  by~$d( f( x_{ n, \, m } ),\, f( x_{ 0 } ) ) < \varepsilon$. Ponieważ $f$
  jest ciągła istnieje taka $\delta > 0$, \linebreak
  że~$d_{ Y }( f( x ),\, f( x_{ 0 } ) ) < \varepsilon$
  dla~$d_{ X }( x,\, x_{ 0 } ) < \delta$. Wystarczy więc dobrać tak $N$,
  żeby dla $n > N$, $m > N$, $x_{ n, \, m }$ znajdował~się
  w~odległości mniejszej niż~$\delta$ od~$x_{ 0 }$. Takie $N$ istnieją
  na~mocy definicji granicy ciągu dwuindeksowego.

\end{proof}
% #############





% #############
\begin{corollary}

  Niech $X$ i~$Y$ będą przestrzeniami metrycznymi. Jeżeli $f: X \to Y$
  jest ciągowo ciągła, to z~granicą ciągu wieloindeksowego można wejść
  pod funkcje.

\end{corollary}



\begin{proof}

  W~przestrzeni metrycznej funkcja jest ciągowo ciągła wtedy i~tylko wtedy,
  gdy jest ciągła topologicznie, wystarczy więc zastosować poprzednie
  twierdzenie.

\end{proof}
% #############

\VerSpaceFour



% #############
\textbf{Odwzorowania wieloliniowe ciągłe i~szeregi.}

W~tym
rozdziale potrzebujemy kilka razy pokazać równość następującego typu.
Mamy dwa elementy $f = \sum_{ l = 1 }^{ \infty } c_{ l } \varphi_{ l }$,
$g = \sum_{ k = 1 }^{ \infty } d_{ k } \psi_{ k }$,
$c_{ l }, d_{ k } \in \Cbb$, to
\begin{equation}
  \label{eq:RS-Vol-I-s02-01}
  \lim\limits_{ K,\, M \to +\infty } \sum_{ \substack{ l < L \\ k < K } }
  c_{ l } d_{ k } \, ( \varphi_{ l } \circ \psi_{ k } ) = f \circ g,
\end{equation}
gdzie $\circ$ jest jakiegoś rodzaju iloczynem, a~symbol
$K, M \to +\infty$ rozumiem jak przy podanej wcześniej definicji
granicy ciągu wieloindeksowego. Aby~nadać tym operacjom sens, musimy
najpierw zdefiniować odwzorowanie wieloliniowe ograniczone.

Niech $X_{ 1 }, \ldots, X_{ n }, Y$ będą przestrzeniami unormowanymi nad
ciałem $\Rbb$ lub~$\Cbb$, zaś
\begin{equation}
  \label{eq:RS-Vol-I-s02-02}
  B: X_{ 1 } \times \ldots \times X_{ n } \to Y
\end{equation}
odwzorowaniem wieloliniowym. Odwzorowanie takie jest ograniczone jeśli
istnieje taka stała $C \in \Rbb_{ + }$, że~dla wszystkich
$x_{ 1 } \in X_{ 1 }, \ldots, x_{ n } \in X_{ n }$ zachodzi
\begin{equation}
  \label{eq:RS-Vol-I-02-03}
  \norm{ B( x_{ 1 }, \ldots, x_{ n } ) }_{ Y } \leq
  C \norm{ x_{ 1 } }_{ X_{ 1 } } \cdot \ldots \cdot \norm{ x_{ n } }_{ X_{ n } }.
\end{equation}
Podstawowe ich własności przedstawił Schwartz w~swoim „Kursie analizy
matematycznej”, str.~102 i~dalsze,
\parencite{Schwartz-Kurs-analizy-matematycznej-Vol-I-Pub-1979}.
W~szczególności tak jak dla odwzorowania liniowego odwzorowanie
wieloliniowe jest ciągłe wtedy i~tylko wtedy, gdy
jest ograniczone, analogicznie również definiuje~się normę
$\norm{ B }$ odwzorowania wieloliniowego~$B$. Dysponując już tymi
pojęciami możemy sformułować potrzebne nam twierdzenie\footnote{Pewne
  idee tego dowodu pochodzą od~Pawła Ducha.}.





% #############
\begin{theorem}
  \label{thm:OdwzorowanieWielolinioweCiagle}

  Jeśli $B: X_{ 1 } \times \ldots \times X_{ n } \to Y$ jest odwzorowanie
  wieloliniowym ograniczonym przestrzeni unormowanych
  $X_{ 1 }, \ldots, X_{ n }, Y$, zaś szeregi
  $\sum_{ l } c_{ 1, \, l } \, \varphi_{ 1, \, l }, \ldots, \sum_{ l } c_{ n, \, l } \,
  \varphi_{ n, \, l }$, $\varphi_{ i, \, l } \in X_{ i }$,
  $l \in \Nbb$, są~warunkowo zbieżne do~elementów
  $\eta_{ 1 }, \ldots, \eta_{ n }$, to zachodzi równość
  \begin{equation}
    \label{eq:RS-Vol-I-s02-04}
    \sum_{ l_{ 1 }, \ldots, l_{ n } } c_{ 1, \, l_{ 1 } } \cdot \ldots
    \cdot c_{ n, \, l_{ n } } B( \varphi_{ 1, \, l_{ 1 } }, \ldots,
    \varphi_{ n, \, l_{ n } } ) = B( \eta_{ 1 }, \ldots, \eta_{ n } ),
  \end{equation}
  przy czym sumę tą należy rozumieć jako
  \begin{equation}
    \label{eq:RS-Vol-I-s02-05}
    \sum_{ l_{ 1 }, \ldots, l_{ n } } c_{ 1, \, l_{ 1 } } \cdot \ldots
    \cdot c_{ n, \, l_{ n } } \, B( \varphi_{ 1, \, l_{ 1 } }, \ldots,
    \varphi_{ n, \, l_{ n } } )
    =
    \lim_{ N \to +\infty } \sum_{ \substack{ l_{ 1 } < N \\ \cdots \\ l_{ n } < N } }
    c_{ 1, \, l_{ 1 } } \cdot \ldots \cdot c_{ n, \, l_{ n } } \,
    B( \varphi_{ 1, \, l_{ 1 } }, \ldots, \varphi_{ n, \, l_{ n } } ).
  \end{equation}

\end{theorem}



\begin{proof}

  Aby~nie ugrząźć w~„nieskończonej” ilości indeksów dowód
  przeprowadzimy dla przypadku $n = 2$, oznaczmy przy tym
  $\varphi_{ 1,\, l } = \varphi_{ l }$, $\varphi_{ 2, \, l } = \psi_{ l }$,
  $c_{ 1, \, l } = c_{ l }$, $c_{ 1, \, l } = d_{ l }$,
  $\eta_{ 1 } = \varphi$ oraz~$\eta_{ 2 } = \psi$. Ponieważ
  $\varphi = \sum_{ l } c_{ l } \, \varphi_{ l }$, więc istniej takie $K$,
  że~$\varphi = \sum_{ k < K } c_{ k } \, \varphi_{ k } + \varphi_{ R }$, gdzie
  $\norm{ \varphi_{ R } }_{ X_{ 1 } } \leq \varepsilon$. Analogicznie
  $\psi = \sum_{ l < L } d_{ l } \psi_{ l } + \psi_{ R }$,
  $\norm{ \psi_{ R } }_{ X_{ 2 } } \leq \varepsilon$. Ponieważ suma
  $\sum_{ k < K } c_{ k } \varphi_{ k }$ jest zbieżna do~$\varphi$, wynika,
  że~ciąg
  $\norm{ \sum_{ k < K } c_{ k } \varphi_{ k } }_{ X_{ 1 } } \to \norm{
    \varphi }_{ X_{ 1 } }$, jest więc również ograniczony. Dobierzmy
  $M \geq 0$ tak~by było wspólnym ograniczeniem\footnote{To
    ograniczenie to~wszystko czego potrzebujemy w~dowodzie, a~aby je
    otrzymać wystarczy warunkowa zbieżność omawianych szeregów.}
  ciągów $\norm{ \sum_{ k < K } c_{ k } \varphi_{ k } }_{ X_{ 1 } }$
  i~$\norm{ \sum_{ l < L } d_{ l } \psi_{ l } }_{ X_{ 2 } }$, niech
  teraz $N = \max( L, K )$. Rozpatrzmy wyrażenie
  \begin{equation}
    \begin{split}
      \label{eq:RS-Vol-I-s02-06}
      &\Vert B( \varphi, \psi ) - \sum_{ k, \, l < N } c_{ k } d_{ l }
        B( \varphi_{ k }, \psi_{ l } ) \Vert_{ Y } =
        \Vert B( \sum_{ k < N } \varphi_{ k } + \varphi_{ R }, \sum_{ l < N } d_{ l } \psi_{ l } +
        \psi_{ R } ) - \sum_{ k,\, l < N } c_{ k } d_{ l }
        B( \varphi_{ k }, \psi_{ l } ) \Vert_{ Y } = \\
      &=
        \Vert B( \sum_{ l < N } \varphi_{ l }, \psi_{ R } )
        + B( \varphi_{ R }, \sum_{ k < N } \psi_{ k } )
        + B( \varphi_{ R }, \psi_{ R } ) \Vert_{ Y } \leq
        \Vert B( \sum_{ l < N } \varphi_{ l }, \psi_{ R } ) \Vert_{ Y }
        + \Vert B( \varphi_{ R }, \sum_{ k < N } \psi_{ k } ) \Vert_{ Y } + \\
      &+ \Vert B( \varphi_{ R }, \psi_{ R } ) \Vert_{ Y }
        \leq 2 \norm{ B } M \varepsilon + \norm{ B } \varepsilon^{ 2 }.
    \end{split}
  \end{equation}
  Dobierając odpowiednio duże $N$ możemy uczynić tą różnicę dowolnie
  małą co kończy dowód.

\end{proof}
% #############






Mogłoby~się wydawać, iż~powyższe twierdzenie nie może być prawdziwe,
bowiem wiemy, że~jeśli szeregi liczb rzeczywistych, a~liczby
rzeczywiste~są przestrzenią Banacha, są~warunkowo zbieżne, to suma ich
iloczynu zależy od~kolejności sumowania poszczególnych
wyrazów\footnote{Co jest prawdą pod warunkiem, że~iloczyn szeregów
  warunkowo zbieżnych nie może być przestawialnie zbieżny. ?????
  Zobacz książkę Schwartza str.~110 i~dalsze,
  \parencite{Schwartz-Kurs-analizy-matematycznej-Vol-I-Pub-1979}.}.

Problem jest jednak pozorny, bowiem twierdzenie powyższe dotyczy tylko
szczególnego sposobu sumowania, które w~przedstawieniu tablicowym
jakiego używa Fichtenholza (zobacz str.~???
\parencite{Fichtenholz-Rachunek-rozniczkowy-ETC-Vol-I-Pub-2005})
odpowiada „sumowaniu po~coraz większych prostokątach”. Inaczej
mówiąc, sum te~zawsze da~się przedstawić jako
\begin{equation}
  \label{eq:RS-Vol-I-s02-07}
  \sum_{ \substack{ k < K \\
      l < L \\ } } a_{ k } b_{ l }
  = ( \sum_{ k < K } a_{ k } ) ( \sum_{ l < L } b_{ l } ).
\end{equation}
Zbieżność tego typu sum
do~$( \sum_{ k } a_{ k } ) ( \sum_{ l } b_{ l } )$ jest oczywista.
Dopiero przy bardziej skomplikowanym wyborze sposobu sumowania,
przykłady ponownie można znaleźć we~wspomnianym miejscu
u~Fichtenholza, kolejność sumowania ma znaczenie. % Sprawdzone.

\VerSpaceTwo










% ######################################
\section{Uwagi do konkretnych stron}

\label{sec:-ETC-Uwagi-ogolne}
% ######################################


\noindent
\Str{7} Warto w~tym miejscu zauważyć, że~każdy domknięty
podzbiór przestrzeni zupełnej jest zupełny. Zbyt dużo materiały w~książce
brakuje, aby móc z czysty sumieniem wstawić powyższy
komentarz. Aby wiedzieć, co to znaczy, że podzbiór przestrzeni
metrycznej jest zupełny, należy wprowadzić topologię indukowaną.


\noindent
\Str{9} Nie wykazano, jedynie stwierdzono jednym słowem
„thus”, że~obie podane definicje~są równoważne. W~istocie, jak
powszechnie wiadomo z~pracy dotyczących analizy funkcjonalnej:
\begin{equation}
  \label{eq:RS-Vol-I-s01-12}
  \norm{ T } =
  \sup_{ \norm{ x }_{ 1 } \leq 1 } \frac{ \norm{ T x }_{ 2 } }{ \norm{ x }_{ 1 } }
  =
  \sup_{ \norm{ x }_{ 1 } = 1 }
  \frac{ \norm{ T x }_{ 2 } }{ \norm{ x }_{ 1 } }
  = \inf \{ \, C \, | \, \forall x, \norm{ T x }_{ 2 } \leq C \norm{ x }_{ 1 } \}.
\end{equation}
Elegancki dowód tego faktu, można znaleźć np.
w~\parencite{Chmielinski-Analiza-funkcjonalna-Pub-2004},
str.~115--116. % Sprawdzone.

\VerSpaceFour





\noindent
\Str{10} W~dowodzie twierdzenia B.L.T. skorzystanie z~$\limsup$
jest zupełnie niepotrzebne, poza tym utrudnia dowód,
że~$\Vert \widetilde{ T } \Vert = \norm{ T }$. Wystarczy bowiem skorzystać
z~twierdzenia o~zachowaniu nierówności przy~przechodzeniu do~granicy,
by otrzymać pożądaną nierówność
\begin{equation}
  \label{eq:RS-Vol-I-s01-13}
  \lim\limits_{ n \to \infty } \norm{ T x_{ n } }_{ 2 }
  \leq \lim\limits_{ n \to \infty } \norm{ T } \, \norm{ x_{ n } }_{ 1 }
  = \norm{ T } \, \norm{ x }_{ 1 },
\end{equation}
Wynika stąd od razu że~$\Vert \widetilde{ T } \Vert \leq \norm{ T }$. Ponieważ
zaś~na~zbiorze $V_{ 1 }$, zachodzi
$\Vert \widetilde{ T } x \Vert_{ 2 } = \norm{ T x }_{ 2 } \leq \norm{ T } \norm{ x
}_{ 1 }$, przy czym $\norm{ T }$ jest minimalną stałą dla~której jest
to~prawdą, więc ze~względu na~to, iż~$V_{ 1 } \subset \widetilde{ V }_{ 1 }$,
mamy nierówność $\norm{ T } \leq \Vert \widetilde{ T } \Vert$, tym samym
$\Vert \widetilde{ T } \Vert = \norm{ T }$. % Sprawdzone.

\VerSpaceFour





\noindent
\Str{12} Ponieważ nie została podana definicja podciągu, nie
można było użyć następującej eleganckiej definicji. Liczba $b$~jest
punktem skupienia ciągu $\{ a_{ n } \}$, jeśli $\{ a_{ n } \}$ zawiera
podciąg zbieżny do~$b$. % Sprawdzone.

\VerSpaceFour





\noindent
\Str{12} Zaproponowana na tej stronie alternatywna definicja
$\limsup$ ma sens tylko dla zbiorów ograniczonych. W~analogiczny
sposób jak pierwotną definicję należy ją rozszerzyć na przypadek
zbiorów nieograniczonych. % Sprawdzone.

\VerSpaceFour





\noindent
\Str{13} Przelicz jawnie, że~funkcje trapezoidalne tworzą ciąg
Cauchy'ego. ????

\VerSpaceFour





\noindent
\Str{16} Można podać równoważną, prostszą charakteryzację
zbiorów mierzalnych. Zbiór $M$ jest mierzalny wtedy i~tylko wtedy, gdy
$M \cup B' = B$, gdzie $B$, $B'$ są pewnymi zbiorami borelowskimi
i~$\mu( B' ) = 0$. Jeśli $M$ ma taką własność, to oczywiście spełniona
jest definicja pierwotna przy $A_{ 1 } = B_{ 1 } = B'$,
$A_{ 2 } = B_{ 2 } = \emptyset$.

Jeśli zaś $M \cup A_{ 1 } = B \cup A_{ 2 }$, to możemy do obu stron
dodać $B_{ 1 } \cup B_{ 2 }$ i~dostajemy,
że~$M \cup B_{ 1 } \cup B_{ 2 } = B \cup B_{ 1 } \cup B_{ 2 }$. Zbiory
$B \cup B_{ 1 } \cup B_{ 2 }$ i~$B_{ 1 } \cup B_{ 2 }$ to zbiory
borelowskie, przy czym $\mu( B_{ 1 } \cup B_{ 2 } ) = 0$.

Inaczej mówiąc, zbiór mierzalny różni~się o~zbiór miary~0 od pewnego
zbioru borelowskiego. % Sprawdzone.

\VerSpaceFour





\noindent
\Str{16} Choć jest to rozdział „Wiadomości wstępne”
(``Preliminaries''), to dyskusja całkowania funkcji niedodatnich jest
przeprowadzona zbyt szybko i~pobieżnie, by~uznać ją za~satysfakcję.

Po~pierwsze należałoby zauważyć, że~jeżeli mamy dwie funkcje
$0 \leq g \leq f$, to z~tego, że~$\int f \, dx < \infty$, wynika
że~$\int g \, dx < \infty$, co przy przyjętej tu definicji całki
z~funkcji dodatniej nie jest od razu oczywiste. Istotnie, jak
zauważyli Reed i~Simon, przy definicji zwykle spotykanej w~pracach dla
matematyków, tego typu dowody~są prostsze. Wiedząc już jednak o~tym,
widać od~razu, że~jeżeli $\int \absOne{ f } \, dx < \infty$ to
całkowalne~są również $f_{ + }$ i~$f_{ - }$, więc definicja całki
z~funkcji niedodatniej ma sens. Idąc dalej~tą drogą można sformułować
całkiem satysfakcjonującą teorię całki.

Podążając ścieżką wskazaną przez Schwartza, zobacz rozdział o~teorii
całki w~\cite{Schwartz-Kurs-analizy-matematycznej-Vol-I-Pub-1979}, będę
zawsze
przyjmował, że~mierzalna funkcja dodatnia jest zawsze całkowalna, choć
jej całka możne wynosić $+\infty$. Jego podejście wskazuje, że~dla
funkcji dodatniej mierzalność nie jest potrzebna, wprowadza więc
on~dodatkowe pojęcia całki górnej, która istniej dla każdej funkcji
$f \geq 0$. % Sprawdzone.

W~tym kontekście warto mieć w~pamięci rozróżnienie jakie podał
Schwartz między całką górną, która jest zdefiniowana dla wszystkich
funkcji nieujemnych i~zawsze ma sens, a~całką w~sensie właściwym
która wymaga by~$\int | f( x ) | \, d\mu( x )$??? była skończona
\cite{Schwartz-Kurs-analizy-matematycznej-Vol-I-Pub-1979}\footnote{Schwartz
  rozwijał teorię całki w~innym
kontekście niż~rozważany tutaj np. całkował od razu funkcji
o~wartościach w~przestrzeni Banacha. To co tu zostało napisane, jest
więc pewną wariacją podanej przez niego teorii.}.

\VerSpaceFour





\noindent
\Str{17} W~twierdzeniu o~zbieżności monotonicznej jest
powiedziane, że~dla funkcji rozpatrywanych w~tym twierdzeniu zachodzi
$\int | f( p ) - f_{ n }( p ) | \, d^{ k }p \to 0$, co~wymaga pewnego
komentarza. Ponieważ $f( p ) \geq f_{ n }( p )$ stąd
$f( p ) - f_{ n }( p ) \geq 0$ więc zachodzi ciąg równoważności
\begin{equation}
  \label{eq:RS-Vol-I-s01-14}
  \int | f( p ) - f_{ n }( p ) | \, dp
  = \int ( f( p ) - f_{ n }( p ) ) \, dp
  = \int f( p ) \, dp - \int f_{ n }( p ) \, dp
\end{equation}
Dowód, że~przy podanych w~tym twierdzeniu założeniach zachodzi
$f \in \Lcal^{ 1 }( \Rbb )$??? i~$\int f_{ n } \, dp \to \int f \, dp$ można
znaleźć w~każdej książce do~teorii całki, np.~str.~29--30,
\cite{Rudin-Analiza-rzeczywista-i-zespolona-Pub-1986}. % Sprawdzone.

\VerSpaceFour





\noindent
\Str{17} W~twierdzeniu o~zbieżności majoryzowanej należy
założyć, że~funkcje $f_{ n }$~są mierzalne. % Sprawdzone.

\VerSpaceFour





\noindent
\Str{21} W~podanej charakterystyce liczbowej zbioru Cantora
jest pewna nieścisłość. Do tego zbioru należy liczba
$\frac{ 1 }{ 3 } = 0.1_{ 3 } = 0.02222\ldots_{ 3 }$, nie jest więc jasne,
jak należy zastosować tu warunek, że~liczba należy do~zbioru Cantora,
jeśli nie zawiera w~swym rozwinięciu trójkowym cyfry~1. Właściwy
warunek\footnote{Wskazał mi go Paweł Duch.} jest następujący:
rozwinięcie trójkowe tej liczby można zapisać w~taki sposób by~nie
zawierało cyfry~1. % Sprawdzone.

\VerSpaceFour





\noindent
\Str{22} Nie rozumiem dlaczego z~faktu, że~miara każdego zbioru
zwartego jest skończona, wynika iż~jest tylko przeliczalnie wiele
punktów o~niezerowej mierze. ????

\VerSpaceFour





\noindent
\Str{22} Wydaje mi~się, że~pojęcie miary osobliwej względem
miary Lebesgue’a zostało wprowadzony w~sposób, który wprowadza sporo
zamieszania. Po pierwsze jeśli miara~$\mu$ jest osobliwa względem
miary Lebesgue’a to znaczy to jest to równoważne stwierdzeniu,
że~$\mu$ i~miara Lebesgue’a~są wzajemnie osobliwe w~myśl definicji
ze~strony~24.

Istotą samego pojęcia wzajemnej osobliwości miar jest fakt, że~miara
$\nu$ jest niezerowana\footnote{Przyjęło~się mówić, że~miara jest
  \textbf{skupiona}, ale na razie nie widziałem, by~Reed i~Simon tego
  pojęcia używali. Zob.~str.~133, \cite{Rudin-Analiza-rzeczywista-i-zespolona-Pub-1986}} tylko na~zbiorze
o~$\mu$-mierze~0. Ujmując to ściśle, istnieje taki zbiór~$A$,
że~$\mu( A ) = 0$ i~jednocześnie $\nu( M \setminus A ) = 0$ ($M$~oznacza
tu przestrzeń mierzalną). Oznacza to, że~$\nu$ może być różna od~zera
tylko na~$A$, a~ten zbiór jest miary zero dla~$\mu$, jest więc dla tej
ostatniej miary „niezauważalny”.

Jeśli jednak przyglądniemy~się teraz definicji, to wystarczy oznaczyć
$B = M \setminus A$ i~relacje~się odwraca. $\nu( B ) = 0$
i~$\mu( M \setminus B ) = 0$, tym samym widzimy, że~relacja ta jest
symetryczna. Obie przedstawione definicje~są więc równoważne.

\VerSpaceFour





\noindent
\Str{25} Należy zrobić dodatkowy komentarz na~temat równości
z~twierdzenia Radona-Nikodyma
\begin{equation}
  \label{eq:RS-Vol-I-s01-15}
  \nu( A ) = \int_{ A } f( x ) d\mu( x ).
\end{equation}
Dowód tego twierdzenie podał Rudin w~\cite{Rudin-Analiza-rzeczywista-i-zespolona-Pub-1986}, str.~134--136,
jednie nie zamieścił w~treści twierdzenia stwierdzenia, że~$f \geq 0$,
pomimo, że~udowodnił ten fakt. Tym samym, całka ta zawsze ma sens
i~jak zauważył Rudin, dzięki $\sigma$-skończoności można
pokazać, że~jest równa $\nu( A )$. % Sprawdzone.

\VerSpaceFour





\noindent
\Str{25} Jakoś nie mogę siebie przekonać, że~funkcja mierzalna
względem dwóch zmiennych, jest też mierzalna względem każdej
ze~zmiennych z~osobna. Dla miar Radona, dla~których zbiory
borelowskie~są mierzalne, tak zapewne jest, jednak dla ogólnych
$\sigma$-algebr wydaje mi~się to dziwnie mocny twierdzeniem.
Zapewne funkcja jest prawie wszędzie mierzalna, ale nic silniejszego.
????

\VerSpaceFour





\noindent
\Str{30} \textbf{Dowód twierdzenia I.27, uzupełnienie.} Zauważmy,
że~tak jak w~dowodzie twierdzenie I.25, jeśli
$\absOne{ f_{ n }( x ) - f_{ n }( y ) } < \varepsilon$ zachodzi dla
każdego~$n$, to również $\absOne{ f( x ) - f( y ) }$. Będzie nam to
potrzebne do~zastosowania argumentu~$\varepsilon / 3$. % Sprawdzone

\VerSpaceFour





\noindent
\Str{30} W~twierdzeniach I.27 i~I.28, należałby dodać, że~dowód
dotyczy funkcji o~wartościach zespolonych. Jednak twierdzenie I.27
łatwo uogólnić na przypadek funkcji o~wartościach w~dowolnej
przestrzeni metrycznej, a~I.28 o~wartościach w~przestrzeni
Banacha. % Sprawdzone.

\VerSpaceFour




\noindent
\Str{37} Można podać inne sformułowanie twierdzenia Pitagorasa,
które w~pewnych sytuacjach jest bardziej naturalne. Jeśli $x_{ i }$,
$i = 1, \ldots, N$ tworzą układ ortogonalny i~$x = \sum x_{ i }$, to
\begin{equation}
  \label{eq:RS-Vol-I-s01-16}
  \norm{ x }^{ 2 } = \sum_{ i = 1 }^{ N } \norm{ x_{ i } }^{ 2 }.
\end{equation}
% Sprawdzone.

\VerSpaceFour





\noindent
\Str{39} Przy okazji definicji izomorfizmu dwóch przestrzeni
Hilberta, warto zauważyć, że~operator unitarny jest zawsze bijekcją
i~tym samym izomorfizmem liniowym. Surjektyweność zachodzi na mocy
definicji, żeby zaś wykazać inijektywność wystarczy skorzystać
z~jednego z~podstawowych twierdzeń algebry: odwzorowanie liniowe jest
iniektywne wtedy i~tylko wtedy, gdy~jego jądro jest trywialne. Jeżeli więc
$U x = 0$, wtedy:
\begin{equation*}
  \label{eq:RS-Vol-I-s01-17}
  0 = ( U x, \, U x )_{ \Hcal_{ 2 } } = ( x, \, x )_{ \Hcal_{ 1 } }
  \iff x = 0.
\end{equation*}
% Sprawdzone.

\VerSpaceFour





\noindent
\Str{40} W~przykładzie~6 przestrzeń Hilberta~$\Hcal'$ musi być
ośrodkowa, ale~pojęcie przestrzeni ośrodkowej jest wprowadzone dopiero
na~stronie~47. % Sprawdzone.

\VerSpaceFour





\noindent
\Str{44} W~tym miejscu warto dowieść następujących twierdzeń.





% #############
\begin{theorem}
  \label{thm:UkladZupelny}

  W~przestrzeni prehilbertowskiej układ ortogonalny niezerowych
  wektorów~$S$ jest zupełny wtedy i~tylko wtedy, gdy~jedynym wektorem
  ortogonalnym do~niego jest wektor~$0$.

\end{theorem}



\begin{proof}

  Przez kontrapozycję $1) \Rightarrow 2)$. Załóżmy, że~istnieje wektor
  $g \neq 0$ ortogonalny do~$S$. Wówczas $S' = S \cup \{ g \}$ jest
  układem ortogonalnym, który zawiera w~sobie $S$ i~$S' \neq S$. Czyli
  $S$ nie jest układem zupełny.

  Przez kontrapozycję $2) \Rightarrow 1)$. Załóżmy teraz, że~$S$ nie jest
  zupełny. W~takim razie istnieje taki układ ortogonalny niezerowych
  wektorów, iż~$S \subset S'$ oraz $S' \setminus S \neq \emptyset$. Teraz
  $y \in S' \setminus S$ jest niezerowym wektorem ortogonalnym do~$S$.

\end{proof}
% #############





% #############
\begin{theorem}
  \label{thm:OrotgonalnoscDoZbioruGestegoIBazy}

  Niech $\Hcal$ będzie przestrzenią Hilberta.
  \begin{enumerate}
  \item Jeżeli wektor $g$ jest ortogonalny do~wszystkich wektorów
    z~$\Hcal$ to~$g = 0$.

  \item Jeżeli wektor $g$ jest ortogonalny do~zbioru gęstego $\Dcal$, to
    $g = 0$.

  \item Jeżeli wektor $g$ ortogonalny do~zbioru $F \subset \Hcal$
    i~jednocześnie pewien zbiór skończonych kombinacji liniowych
    elementów zbioru~$A$, niekoniecznie wszystkich, jest gęsty
    w~$\Hcal$, to~$g = 0$.

  \end{enumerate}

\end{theorem}



\begin{proof}

  \begin{enumerate}
  \item Dowód tego punktu jest powszechnie znany, ale dla porządku
    przedstawmy go tutaj. Jeśli $g$ jest ortogonalny do~$\Hcal$ to
    w~szczególności $( g, \, g ) = 0$, tym samym na mocy definicji
    przestrzeni Hilberta~$g = 0$.

  \item Jeśli zbiór $\Dcal$ jest gęsty, więc dla każdego $x \in \Hcal$
    można znaleźć ciąg $x_{ n } \in \Dcal$ taki, że~$x_{ n } \to x$. Tym
    samym na mocy ciągłości iloczynu skalarnego
    \begin{equation}
      \label{eq:RS-Vol-I-s01-18}
      ( g, \, x ) = \lim_{ n \to \infty } ( g, \, x_{ n } ) = 0.
    \end{equation}
    Tym samy $g = 0$ na~mocy poprzedniego punktu.

  \item Jeśli $g$ jest ortogonalny do~$F$ to jest też ortogonalny
    do~skończonych kombinacji liniowych elementów tego zbioru. Jeśli
    któryś ze zbiorów tych kombinacji liniowych jest gęsty w~$\Hcal$,
    to~$g = 0$ na mocy poprzedniego punktu.

  \end{enumerate}

\end{proof}
% #############

\VerSpaceFour





\noindent
\Str{45} Nie rozumiem dlaczego z~nierówności
$\sum_{ i = 1 }^{ N } \absOne{ ( x_{ \alpha_{ i } }, \,  y ) } \leq \norm{
  y }^{ 2 }$ wynika, że~tylko dla przeliczalnej ilości indeksów $\alpha$
zachodzi $( x_{ \alpha }, \, y ) \neq 0$. ????

\VerSpaceFour





\noindent
\Str{45} \textbf{Dowód twierdzenia II.6.} Fakt, że~granica
rozważanego szeregu nie zależy od~kolejności wyrazów, można udowodnić
w~następujący sposób. Niech $A'$ będzie przeliczalnym zbiorem indeksów
dla których $( x_{ \alpha }, y ) \neq 0$ i~niech $\alpha_{ i } \in A'$
będzie jakimś jego uporządkowaniem. Wówczas przeprowadzone w~dowodzie
rozumowanie pokazuje, że
\begin{equation}
  \label{eq:RS-Vol-I-s01-19}
  y = \sum_{ j = 1 }^{ \infty } ( x_{ \alpha_{ i } }, \, y ) \, x_{ \alpha_{ i } }.
\end{equation}
Jeżeli teraz weźmiemy jakieś inne jego uporządkowanie $x_{ j }$,
to~możemy powtórzyć całe rozumowanie i~otrzymać, że~suma nowego
szeregu znów wynosi $y$, tym samym jest niezależna od~kolejności
sumowania.

% Pomimo tego, że~jego suma nie zależy od porządku wyrazów, nie oznacza
% to, że~szereg ten musi być absolutnie zbieżny. Rozpatrzmy bowiem
% element przestrzeni $\lcal^{ 2 }$
% \begin{equation}
%   \label{eq:RS-Vol-I-s01-20}
%   y = \left( 1, \tfrac{ 1 }{ 2 }, \tfrac{ 1 }{ 3 }, \ldots \right), \quad
%   \norm{ y }^{ 2 }
%   = \sum_{ i = 1 }^{ \infty } \frac{ 1 }{ n^{ 2 } }
%   = \frac{ \pi^{ 2 } }{ 6 } < +\infty.
% \end{equation}
% Jeśli~bazę ortonormalną przyjmiemy:
% \begin{equation}
%   \label{eq:RS-Vol-I-s01-21}
%   e^{ k } = ( 0, \ldots, 0, \overbrace{ 1 }^{ k }, 0, \ldots, 0),
% \end{equation}
% to samym od~razu otrzymujemy
% \begin{equation}
%   \label{eq:RS-Vol-I-s01-22}
%   y =
%   \sum_{ k = 1 }^{ \infty } \tfrac{ 1 }{ k } e^{ k }, \quad
%   \sum_{ k = 1 }^{ \infty } \norm{ \tfrac{ 1 }{ k } e^{ k } }
%   = \sum_{ k = 1 }^{ \infty } \tfrac{ 1 }{ k } = +\infty.
% \end{equation}

% Paweł Duch znalazł dobre wyjaśnienie dlaczego suma tego konkretnego
% typu szeregów nie zależy od~kolejności. Szereg zbieżny warunkowo jest
% zbieżny bowiem jego wyrazy ,,odpowiednio~się kompensują'', wyrazy
% ujemne kasują poprzedzające jej wyrazy dodatnie w~taki sposób,
% że~granica istnieje. Dlatego też jeśli zmienimy ich kolejność to~mogą
% one~się nie skompensowań tak jak poprzednio i~otrzymamy inną sumę.

% Jednak tutaj każdy wyraz jest w~kierunku prostopadłym, do~wszystkich
% innych, więc żadne dwa z~nich nie mogą~się wzajemnie skompensować.
% Jeżeli więc taki szereg jest zbieżny, to zmiana kolejności nie wpływa
% na sumę.

% Schwartz bada w~swoje książce szeregi w~przestrzeniach Banacha, które
% nie~są absolutnie zbieżne, ale~ich suma nie zależy od~kolejności
% sumowania (str.~110 i~dalsze
% \cite{SchwartzKursAnalizyMatematycznejVolI1979}). Trajdos przetłumaczył ich
% nazwę jako szeregi \textit{zbieżne przestawialnie}. % Sprawdzone.

% \vspace{\spaceFour}



% \start \Str{45} Wzorując~się na Grabowskim i~Ingardenie
% \cite{GrabowskiIngardenMechanikaKwantowa1987}, warto w~kontekście
% twierdzenia II.6 udowodnić jeszcze jedną piękną równość:
% \begin{equation}
%   \label{eq:RS-Vol-I-s01-23}
%   ( y, \, y' ) =
%   \sum_{ \alpha \in A } ( y, \, x_{ \alpha } ) ( x_{ \alpha }, \, y' ),
% \end{equation}
% gdzie $S = \{ x_{ \alpha } \}_{ \alpha \in A }$ jest bazą ortonormalną,
% niekoniecznie przeliczalną.

% Zacznijmy od~tego, że~istniej przeliczalny zbiór $x_{ \alpha_{ j } }$
% taki,
% że~$y' = \sum_{ j = 1 }^{ \infty } ( x_{ \alpha_{ j } }, \, y' ) x_{ \alpha_{ j } }$.
% Wtedy
% \begin{equation}
%   \label{eq:RS-Vol-I-s01-24}
%   \begin{split}
%     ( y, \, y' )
%     &=
%       ( y, \, \lim_{ n \to \infty } \sum_{ j = i }^{ n }
%       ( x_{ \alpha_{ j } }, \,  y' ) \, x_{ \alpha_{ j } } )
%       =
%       \lim_{ n \to \infty } ( y, \, \sum_{ j = 1 }^{ n }
%       ( x_{ \alpha_{ j } }, \, y' ) x_{ \alpha_{ j } } ) = \\
%     &=
%       \lim_{ n \to \infty } \sum_{ j = 1 }^{ n } ( y, \, ( x_{ \alpha_{ j } }, \, y' )
%       x_{ \alpha_{ j } } )
%       =
%       \sum_{ j = 1 }^{ \infty } ( y, x_{ \alpha_{ j } } ) ( x_{ \alpha_{ j } }, y' )
%       =
%       \sum_{ \alpha \in A } ( y, \, x_{ \alpha } ) ( x_{ \alpha }, \,  y' ).
%   \end{split}
% \end{equation}
% Zastąpienie w~ostatniej równości sumy po zbiorze
% $\alpha_{ j }$, $j \in \Nbb$ sumą po~$A$ jest możliwe na mocy tego,
% że~jeżeli $\alpha \neq \alpha_{ j }$ dla każdego $j$, to wyrażenie pod
% sumą jest równe zeru. Zauważmy też, że~skoro zbieżność szeregu
% $\sum_{ \alpha \in A } ( x_{ \alpha }, \, y ) x_{ \alpha }$ nie zależy
% od~kolejności, to~na~mocy przeprowadzone rozumowania również suma
% szeregu
% $\sum_{ \alpha \in A } ( y, \, x_{ \alpha } ) ( x_{ \alpha }, \, y' )$, nie
% zależy od~kolejności wyrazów.

% Ta tożsamość pozwala nam wykazać, w~sposób bardziej elegancki niż
% zrobili to Reed i~Simon, równanie (II.2), mianowicie podstawiając
% $y' = y$, mamy
% \begin{equation}
%   \label{eq:RS-Vol-I-s01-25}
%   \norm{ y }^{ 2 } = ( y, \, y ) =
%   \sum_{ \alpha \in A } ( y, \, x_{ \alpha } ) ( x_{ \alpha }, \, y )
%   = \sum_{ \alpha \in A } \absOne{ ( x_{ \alpha }, \,  y ) }^{ 2 }.
% \end{equation}
% % Sprawdzone.

% \vspace{\spaceFour}



% \start \Str{45} Dysponując twierdzeniem II.6, można teraz podać ogólną
% postać nierówności Bessela.




% % #############
% \begin{theorem}[Ogólna postać nierówności Bessela]

%   Niech $S = \{ x_{ \alpha } \}_{ \alpha \in A }$ będzie układem
%   ortonormalny i~niech $B \subset A$. Wówczas:
%   \begin{equation}
%     \label{eq:RS-Vol-I-s01-26}
%     \sum_{ \beta \in B } \absOne{ ( x_{ \beta }, \, x ) }^{ 2 } \leq \norm{ x }^{ 2 }.
%   \end{equation}

% \end{theorem}



% \begin{proof}

%   Prawie nie ma czego dowodzić, bowiem
%   \begin{equation*}
%     \label{eq:RS-Vol-I-s01-27}
%     \norm{ x }^{ 2 } = \sum_{ \alpha \in A } \absOne{ ( x_{ \alpha }, \, x ) }^{ 2 }
%     \geq \sum_{ \beta \in B } \absOne{ ( x_{ \beta }, \, x ) }^{ 2 }.
%   \end{equation*}

% \end{proof}
% % #############
% % Sprawdzone.

% \vspace{\spaceFour}



% % ##################
% \start \Str{49} Konstrukcja iloczynu tensorowego jest przeprowadzona
% bardzo szkicowo, przez co~sens tego co~robimy jest często ukryty,
% tutaj postaram~się uzupełnić brakujące elementy.

% Po~pierwsze zauważmy, że~aby mówić o~sprzężonej biliniowej formie
% $\mu$, przez co rozumiemy, że~jest ona addytywna przy ustalonym
% jednym z~argumentów oraz~że~zachodzi
% \begin{equation}
%   \label{eq:RS-Vol-I-s01-28}
%   \mu\langle \alpha \psi, \, \varphi \rangle = \bar{ \alpha } \, \mu\langle \psi, \, \varphi \rangle, \quad
%   \mu\langle \psi, \, \alpha \varphi \rangle =
%   \bar{ \alpha } \, \mu\langle \psi, \, \varphi \rangle.
% \end{equation}

% Jedną z~przyczyn zamieszania jest używanie tego samego symbolu na
% iloczyn skalarny w~$\Hcal_{ 1 }$, $\Hcal_{ 2 }$ i~$\Ecal$, tu będę je
% oznaczał odpowiednio jako $( \, \cdot \,, \cdot \, )_{ \Hcal_{ 1 } }$,
% $( \, \cdot \,, \cdot \, )_{ \Hcal_{ 2 } }$, $( \, \cdot \,, \cdot \, )_{ \Ecal }$.
% Działanie formy $\varphi_{ 1 } \otimes \varphi_{ 2 }$ można więc zapisać jako
% \begin{equation}
%   \label{eq:RS-Vol-I-s01-29}
%   ( \varphi_{ 1 } \otimes \varphi_{ 2 } ) \langle \psi_{ 1 }, \, \psi_{ 2 } \rangle
%   =
%   ( \psi_{ 1 }, \, \varphi_{ 1 } )_{ \Hcal_{ 1 } }
%   ( \psi_{ 2 }, \, \varphi_{ 2 } )_{ \Hcal_{ 2 } }.
% \end{equation}
% Iloczyn skalarny na~$\Ecal$ ma więc postać:
% \begin{equation}
%   \label{eq:RS-Vol-I-s01-30}
%   ( \varphi \otimes  \psi, \, \eta \otimes \mu )_{ \Ecal }
%   = ( \varphi, \, \eta )_{ \Hcal_{ 1 } } ( \psi, \,  \mu )_{ \Hcal_{ 2 } }.
% \end{equation}
% Formę postaci $\varphi \otimes \psi$ będziemy nazywać \textbf{formą prostą}.
% Dowolna forma jest sumą form prostych.

% Tu~pojawia~się drugie źródło komplikacji. Ta sama forma może być
% przedstawiona za~pomocą różnych sum form prostych. Na~przykład
% \begin{equation}
%   \label{eq:RS-Vol-I-s01-31}
%   \mu_{ 1 } = \varphi \otimes ( \psi_{ 1 } + \psi_{ 2 } ) = \mu_{ 2 }
%   = \varphi \otimes \psi_{ 1 } + \varphi \otimes \psi_{ 2 }.
% \end{equation}
% Z~tego względu należałoby zdefiniować $\Ecal$ jako zbiór klas
% abstrakcji względem relacji równoważności, która utożsamia dwie sumy
% form prostych jeśli przedstawiają tę~samą dwuantyliniową formę.
% Następnie dowiedlibyśmy, że~iloczyn w tej przestrzeni ilorazowej nie
% zależy od~reprezentanta, co jest równoważne postępowaniu Reeda
% i~Simona, jednak z~jakiegoś powodu nie chcieli pójść tą drogą.

% Powyższy przykład uogólnia~się do~dobrze znanego z~algebry liniowej
% faktu, iż~iloczyn tensorowy jest wieloliniowy. Z~samej definicji
% wynika, że~zachodzi następująca równość form
% \begin{equation}
%   \label{eq:RS-Vol-I-s01-32}
%   \varphi \otimes \ldots \otimes ( \alpha \psi_{ 1 } + \beta \psi_{ 2 } ) \otimes \ldots \otimes \eta =
%   \alpha \, \varphi \otimes \ldots \otimes \psi_{ 1 } \otimes \ldots \otimes \eta
%   + \beta \, \varphi \otimes \ldots \otimes \psi_{ 2 } \otimes \ldots \otimes \eta.
% \end{equation}

% W~dalszym ciągu potrzebna będzie nam następująca obserwacja: jeżeli
% $\mu \in \Ecal$, to zachodzi równość:
% \begin{equation}
%   \label{eq:RS-Vol-I-s01-33}
%   ( \varphi \otimes \psi, \, \mu )_{ \Ecal } ??? = \mu\langle \varphi, \psi \rangle,
% \end{equation}
% gdzie $\mu\langle \varphi,  \psi \rangle$ oznacza wartość formy $\mu$ na~parze
% uporządkowanej $\langle \varphi, \psi \rangle$. Jeżeli $\mu = \eta \otimes \chi$ to
% powyższa równość wynika z~definicji wszystkich działań. Jeśli zaś
% $\mu = \sum_{ i = 1 }^{ N } c_{ i } \, ( \eta_{ i } \otimes \psi_{ i } )$,
% wówczas
% \begin{equation}
%   \label{eq:RS-Vol-I-s01-34}
%   \begin{split}
%     ( \varphi \otimes \psi, \mu )_{ \Ecal }
%     &=
%       \left( \varphi \otimes \psi, \, \sum_{ i = 1 }^{ N } c_{ i } \,
%       ( \eta_{ i } \otimes \chi_{ i } ) \right)_{ \Ecal }
%       =
%       \sum_{ i = 1 }^{ N } c_{ i } \,
%       ( \varphi \otimes \psi, \, \eta_{ i } \otimes \chi_{ i } )_{ \Ecal } = \\
%     &=
%       \sum_{ i = 1 }^{ N } c_{ i } \, ( \eta_{ i } \otimes \chi_{ i } )
%       \langle \varphi, \, \psi \rangle
%       = \mu\langle \varphi, \, \psi \rangle.
%   \end{split}
% \end{equation}
% Dla zupełności przedstawimy uogólnienie tej obserwacji. Jeśli nadamy
% $\Hcal_{ 1 } \times \Hcal_{ 2 }$ strukturę sumy prostej
% \begin{equation}
%   \label{eq:RS-Vol-I-s01-35}
%   \sum_{ i = 1 }^{ N } d_{ i } \, \langle \phi_{ i }, \, \varphi_{ i } \rangle
%   :=
%   \langle \sum_{ i = 1 }^{ N } d_{ i } \, \phi_{ i }, \,
%   \sum_{ i = 1 }^{ N } d_{ i } \, \varphi_{ i } \rangle,
% \end{equation}
% i~$\eta = \sum_{ i = 1 }^{ N } d_{ i } \: \varphi_{ i } \otimes \psi_{ i }$,
% $d_{ i } \in \Cbb$, to~korzystając z~półtoraliniowości mamy
% \begin{equation}
%   \label{eq:RS-Vol-I-s01-36}
%   \begin{split}
%     ( \eta, \, \mu )_{ \Ecal }
%     &=
%       \sum_{ i = 1 }^{ N } \bar{ d }_{ i } \:
%       \big( ( \varphi_{ i } \otimes \psi_{ i } ), \, \mu \big)_{ \Ecal }
%       =
%       \sum_{ i = 1 }^{ N } \bar{ d }_{ i } \:
%     ( \varphi_{ i } \otimes \psi_{ i }, \, \mu )_{ \Ecal } = \\
%     &=
%       \sum_{ i = 1 }^{ N } \bar{ d }_{ i } \: \mu \langle \varphi_{ i }, \, \psi_{ i } \rangle
%       = \mu( \eta ).
%   \end{split}
% \end{equation}
% $\mu( \eta )$ oznacza wartość formy $\mu$ na~$\eta$.

% % W tym fragmencie wielokrotnie powtarza się "zauważyć" <<<<<<<<<<<<
% Teraz możemy przejść do~dyskusji, czy~wzór na iloczyn skalarny nie
% zależy od~przedstawienia jako sumy form prostych. Zauważmy, że~dwie
% formy $\mu$, $\eta$~są równe gdy
% \begin{equation}
%   \label{eq:RS-Vol-I-s01-37}
%   \mu\langle \varphi, \, \psi \rangle = \eta\langle \varphi, \, \psi \rangle, \quad
%   \forall \varphi \in \Hcal_{ 1 }, \psi \in \Hcal_{ 2 }.
% \end{equation}

% Musimy teraz pokazać, że~iloczyn skalarnych dwóch form z~$\Ecal$,
% nie~zależy od ich przedstawienia w~postaci sumy form prostych.
% Kluczowe tu jest wynikający z~definicji fakt, że~dwie różne postaci
% tej samej formy różnią się o~sumę form prostych reprezentującą formę
% zerową. Dla przykładu, jeżeli weźmiemy dwie formy określone wyżej
% $\mu_{ 1 }$ i~$\mu_{ 2 }$, to ich różnica:
% \begin{equation}
%   \label{eq:RS-Vol-I-s01-38}
%   \mu_{ 1 } - \mu_{ 2 } =
%   \varphi \otimes ( \psi_{ 1 } + \psi_{ 2 } ) - \varphi \otimes \psi_{ 1 } - \varphi \otimes \psi_{ 2 }
% \end{equation}
% jest formą zerową. Zwróćmy uwagę, że~rozkład formy zerowej na formy
% proste również nie jest jednoznaczny, np.~każda forma postaci
% $\varphi \otimes 0$ reprezentuje formę zerową. Jeżeli teraz uda~się pokazać,
% że~forma zerowa niezależnie od~swojej postaci, jest ortogonalna
% do~każdej formy w~$\Ecal$, to~będziemy mogli pokazać niezależność
% iloczynu skalarnego w~$\Ecal$ od~konkretnej postaci form.

% Wyniki uzyskane wyżej, wraz z~tym co~napisali Reed i~Simon jest
% wystarczające by~wykazać żądaną ortogonalność formy zerowej.
% Dokładniej Reed i~Simon pokazali, że~wybierając bazę odpowiedniej
% przestrzeni dostajemy dla każdego $\lambda \in \Ecal$
% \begin{equation}
%   \label{eq:RS-Vol-I-s01-39}
%   ( \lambda, \, \lambda ) = \sum_{ j l } \absOne{ c_{ j l } }^{ 2 } \geq 0,
% \end{equation}
% czyli iloczyn skalarny jest dodatnio określony, jeśli zaś jest
% równy~0, to $\lambda$ jest formą zerową.

% Teraz wykażemy, że~rozważany iloczyn skalarny jest niezależny od
% użytego rozkładu. Niech $\psi \sim \psi'$ i~$\varphi' \sim \varphi$, a~$\mu$
% będzie formą zerową. W~takiej sytuacji $\psi' = \psi + \mu$
% i~$\varphi' = \varphi + \mu$, przy czym konkretna postać $\mu$ może być w~obu
% przypadkach różna. Teraz możemy, na~poziomie konkretnych sum form
% prostych, przeprowadzić następujący rachunek, kończący dowód
% \begin{equation}
%   \label{eq:RS-Vol-I-s01-40}
%   ( \varphi', \, \psi' ) = ( \varphi + \mu, \, \psi' )
%   = ( \varphi, \, \psi' ) + ( \mu, \, \psi' ) = ( \varphi, \, \psi' )
%   = ( \varphi, \, \psi + \mu ) = ( \varphi, \, \psi ).
% \end{equation}

% Ta relacja stanowi po~prostu, że~dwa elementy
% $\mu_{ 1 }, \mu_{ 2 } \in \Ecal$ są równoważne, wtedy i~tylko wtedy

% \vspace{\spaceFour}



% % #############
% \start \Str{50} Jako konkluzję konstrukcji iloczynu tensorowego
% przestrzeni Hilberta warto podać następującą piękną równość, która
% wynika wprost z~wprowadzonych definicji.
% \begin{equation}
%   \label{eq:RS-Vol-I-s01-41}
%   \norm{ \varphi \otimes \psi }_{ \Hcal_{ 1 } \otimes \Hcal_{ 2 } }
%   = \norm{ \varphi }_{ \Hcal_{ 1 } } \norm{ \psi }_{ \Hcal_{ 2 } }.
% \end{equation}
% Za jej pomocą można łatwo pokazać, że~iloczyn tensorowy jest ciągły,
% dowód jest taki sam jak w~przypadku iloczynu skalarnego. % Sprawdzone.

% \vspace{\spaceFour}



% \start \Str{50} Część dowodu \textbf{Propostion~2}, że~układ
% $\varphi_{ l } \otimes \psi_{ k }$ stanowi bazę ortonormalną, jest miejscami
% zbyt pobieżna by była zrozumiała dla mnie. W~tym jednak momencie
% kluczowa równość
% \begin{equation}
%   \label{eq:RS-Vol-I-s01-42}
%   \varphi \otimes \phi = \sum_{ l,\, k } c_{ l } d_{ k }\, \varphi_{ l } \otimes \psi_{ k },
% \end{equation}
% wynika z~tego, że~iloczyn tensorowy jest funkcją dwuliniową ciągła
% i~twierdzenia \eqref{thm:OdwzorowanieWielolinioweCiagle}. Ponieważ
% $\varphi_{ l } \otimes \psi_{ k }$ jest układem ortonormalnym
% i~przeliczalnym\footnote{W~przypadku gdy nie jest, dla danego wektora
%   tylko przeliczalna ilość współczynników $c_{ l }$ i~$d_{ k }$ nie
%   znika i~można powtórzyć rozumowanie.} można go ustawić w~ciąg
% jednoindeksowy, którego suma nie zależy od kolejności sumowania. Tym
% samym równość \eqref{eq:RS-Vol-I-s01-42} zachodzi niezależnie od~sposobu
% sumowania.

% \vspace{\spaceFour}



% % ####################
% \start \Str{51} W~tym miejscu będziemy potrzebowali następującego
% twierdzenie, którego dowód znalazł Paweł Duch.





% % #############
% \begin{theorem}
%   \label{thm:TwierdzenieOIloczynieMiarPWZerowym}

%   Niech $( X, \mu )$ i~$( Y, \nu )$ są przestrzeniami z~miarą
%   $\sigma$-skończoną i~niech $A \subset X \times Y$ będzie zbiorem
%   mierzalny względem miary $\mu \otimes \nu$. Jeśli dla $\mu$-prawie
%   każdego $x \in X$, miara $\nu$ zbioru $( x \times Y ) \cap A$ jest
%   równa 0, wówczas $(\mu \otimes \nu)( A ) = 0$.

% \end{theorem}



% \begin{proof}

%   Standardowo będziemy utożsamiać zbiór $( x \times Y ) \cap A$
%   z~odpowiednim podzbiorem $Y$

%   W~przypadku miary dodatniej można tak sformułować teorię całki,
%   że~dla każdego zbioru zachodzi\footnote{Zobacz uwagi o~całce górnej
%     w~ujęciu Schwartza.}
%   \begin{equation}
%     \label{eq:RS-Vol-I-s01-43}
%     \mu( S ) = \int_{ S } \chi_{ S }( x ) \, d\mu( x ),
%   \end{equation}
%   niezależnie czy jego miara jest skończona, czy nie. Skoro zaś obie
%   miary~są $\sigma$-skończone, to zachodzi twierdzenie
%   o~zamianie całki podwójnej na iterowaną, znów niezależnie od~tego
%   czy poszczególne całki są skończone czy nie.

%   Dostajemy więc
%   \begin{equation}
%     \label{eq:RS-Vol-I-s01-44}
%     (\mu \otimes \nu)( A ) =
%     \int_{ X \times Y } \chi( x, y ) \, d\mu( x ) \otimes d\nu( y )
%     =
%     \int_{ X } \left( \int_{ Y } \chi( x, y ) \, d\nu( y ) \right) \,
%     d\mu( x )
%     = 0,
%   \end{equation}
%   bowiem na mocy założeń
%   $\mu\big( ( x \times Y ) \cap A \big) = \int_{ Y } \chi( x, y ) \, d\nu( y )$
%   jest równa zeru dla $\mu$-prawie wszystkich~$x$.
% \end{proof}
% % #############





% Zauważmy, że~to twierdzenie ma prostą interpretację na płaszczyźnie
% $\Rbb^{ 2 }$. Wówczas zbiór $x \times Y$ jest prostą pionową przechodzącą
% przez punkt $x$ i~powyższe twierdzenie mówi, że~jeśli prawie każdy
% przekrój zbioru $A$ prostymi pionowymi ma długość~0, to pole
% powierzchni $A$ również jest zerowe.

% \vspace{\spaceFour}



% % ####################
% \start \Str{51} \textbf{Dowód, że~$\varphi_{ k }( x ) \otimes \psi_{ l }( y )$ tworzą układ
%   ortonormalny.} Sprawdźmy najpierw, że~funkcja
% $\varphi_{ l }( x ) \otimes \psi_{ l }( y )$ jest całkowalna w~kwadracie. Zgodnie z~tym
% co napisano wcześnie prawdziwe są następujące
% \begin{subequations}
%   \begin{equation}
%     \label{eq:RS-Vol-I-s01-45}
%     \iint\limits_{ M_{ 1 } \times M_{ 2 } }
%     \absOne{ \varphi_{ k }( x ) \psil( y ) }^{ 2 } \,
%     d\mu_{ 1 }( x ) d\mu_{ 2 }( y )
%     =
%     \int_{ M_{ 1 } } \absOne{ \varphi_{ k }( x ) }^{ 2 } \, d\mu_{ 1 }( x )
%     \int_{ M_{ 2 } } \absOne{ \psi_{ l }( y ) }^{ 2 } \, d\mu_{ 2 }( y )
%     = 1.
%   \end{equation}



%   \begin{equation}
%     \label{eq:RS-Vol-I-s01-46}
%     \begin{split}
%       \iint\limits_{ M_{ 1 } \times M_{ 2 } } \overline{ \varphi( x ) } \,
%       \overline{ \psi_{ l }( y ) } \varphi_{ m }( x ) \psi_{ n }( y ) \,
%       d\mu_{ 1 }( x ) d\mu_{ 2 }( y )
%       &=
%       \int_{ M_{ 1 } } \overline{ \varphi_{ k }( x ) } \varphi_{ m }( x ) \, d\mu_{ 1 }( x )
%         \int_{ M_{ 2 } } \overline{ \psi_{ l }( y ) } \psi_{ n }( y ) \, d\mu_{ 2 }( y )
%         = \\
%       &= \delta_{ k, m } \; \delta_{ l, n }.
%     \end{split}
%   \end{equation}
% \end{subequations}
% Pierwsza całka w~drugiej równości to iloczyn skalarny funkcji
% $\varphi_{ k }( x ) \psi_{ l }( y )$ i~$\varphi_{ m }( x ) \psi_{ n }( y )$
% w~przestrzeni $\Lcal^{ 2 }( M_{ 1 } \times M_{ 2 }, d\mu_{ 1 } \otimes d\mu_{ 2 } )$,
% funkcja podcałkowa jest więc całkowalna i~wszystkie dalsze przejścia
% są dozwolone.

% Teraz udowodnimy, że~funkcja $f( x, y ) = 0$ prawie wszędzie.
% Po~pierwsze zbiór na którym $f( x, y ) \neq 0$ jest równy
% $f^{ -1 }( \Cbb \setminus 0 )$, jest więc on mierzalny. Teraz fakt,
% że~zbiór ten ma~miarę~$0$ wynika z~tego co pokazali Reed i~Simon
% oraz~twierdzenia powyżej.

% \vspace{\spaceFour}



% \start \StrWg{106}{5} Nie rozumiem dlaczego w~tym miejscu dodatkowo
% zaznacza~się, że~miarę dowolnego zbioru mierzalnego, można
% aproksymować od~wewnątrz przez zbiory zwarte i~\textit{borelowskie}.
% Skoro rozważamy tylko zwarte przestrzenie Hausdorffa, wszystkie zbiory
% zwarte~są domknięte i~tym samym borelowskie.





% % ##################
% \newpage
% \CenterBoldFont{Błędy}


% \begin{center}
%   \begin{tabular}{|c|c|c|c|c|}
%     \hline
%     & \multicolumn{2}{c|}{} & & \\
%     Strona & \multicolumn{2}{c|}{Wiersz} & Jest
%                               & Powinno być \\ \cline{2-3}
%     & Od góry & Od dołu & & \\
%     \hline
%     2   &  8 & & surjective & \textbf{surjective} \\
%     2   &  9 & & \textbf{surjective} & surjective \\
%     14  & 10 & & $\sum_{ 2^{ n } }( f )$ & $\sum_{ 2 n }( f )$ \\
%     14  & 14 & & $\sum_{ 2^{ n } }( f )$ & $\sum_{ 2 n }( f )$ \\
%     17  &  3 & & $\int \;\: f_{ n } \;\: dp$ & $\int f_{ n } \, dp$ \\
%     25  &  2 & & $f$ & $f \geq 0$ \\
%     40  & & 3 & a~Hilbert & a~separable Hilbert \\
%     44  & & 7 & $\frac{ T( y ) }{ T( x_{ 0 } ) }$
%            & $\frac{ ( x_{ 0 }, \, y ) }{ ( x_{ 0 }, \, x_{ 0 } ) }$ \\
%     45  & & 2 & some & all \\
%     46  & & 9 & $( v_{ 1 }, \, u_{ 1 } ) v_{ k }$
%            & $( v_{ 1 }, \, u_{ 2 } ) v_{ 1 }$ \\
%     49  & & 7 & linear & bilinear \\
%     50  & 16 & & $\sum_{ j l } \absOne{ c_{ j l } }^{ 2 }$
%            & $\sum_{ j l } \absOne{ c_{ j l } }^{ 2 } \geq 0$ \\
%     53  & & 12 & $\varphi_{ k_{ \sigma( 1 ) } } \otimes \varphi_{ k_{ \sigma( 2 ) } }
%                  \cdots \otimes \varphi_{ k_{ n( p ) } }$
%            & $\varphi_{ k_{ \sigma( 1 ) } } \otimes \varphi_{ k_{ \sigma( 2 ) } } \otimes
%              \cdots \otimes \varphi_{ k_{ \sigma( p ) } }$ \\
%     70  & 8 & & $\lim\limits_{ n \to +\infty }$ & $\lim\limits_{ m \to +\infty }$ \\
%     % Sprawdź czy dobrze.
%     73  & & 2 & $\sum_{ k = 1 }^{ \infty } \absOne{ \lambda_{ k } } < \infty$
%            & $\sum_{ k = 1 }^{ \infty } \absOne{ \lambda_{ k } } \leq
%              \norm{ \lambda_{ 0 } }_{ c_{ 0 }^{ * } } < \infty$ \\
%     79  & 6 & & $\norm{ [x] }$ & $\norm{ [x] }_{ 1 }$ \\
%     82  & & 15 & since$T[ B_{ r }^{ X } ]$ & since $T[ B_{ r }^{ X } ]$ \\
%     104 &  5 & & $[ 0, 1 ]$ & $[ -1, 1 ]$ \\
%     104 &  6 & & $\Vert P_{ n }( f )$ & $\Vert \, P_{ n }( f )$ \\
%     106 & 11 & & $\mu( \, Y \setminus \widetilde{ Y } )$
%            & $\mu( Y \setminus \widetilde{ Y } )$ \\
%     106 &  5 & & $C \subset O$ & $C \subset Y$ \\
%     107 &  4 & & $\Real[ e^{ -i \varphi } f ]$ & $e^{ -i \varphi } f$ \\
%     107 &  4 & & $\Real( e^{ -i \varphi } f )$ & $e^{ -i \varphi } f$ \\
%     107 & 10 & & $\int\! fd\mu$ & $\int f \, d\mu$ \\
%     107 & & 10 & functions & functionals \\
%     109 & 4 & & $\norm{ l }\!\sup$ & $\norm{ l } \sup$ \\
%     112 & & 8 & $X$ & $F$ \\
%     128 & 2 & & $| \rho_{ \alpha_{ 1 } }( x ) |$ & $\rho_{ \alpha_{ 1 } }( x )$ \\
%     128 & 2 & & $| \rho_{ \alpha_{ n } }( x ) |$ & $\rho_{ \alpha_{ n } }( x )$ \\
%     % & & & & \\
%     % & & & & \\
%     % & & & & \\
%     % & & & & \\
%     % & & & & \\
%     \hline
%   \end{tabular}

% \end{center}


% \noindent
% \textbf{Grzbiet} \\
% \Jest  REVSED \\
% \Powin REVISED \\
% \StrWg{43}{3} \\
% \Jest
% $\displaystyle \sup_{ \substack{ \norm{ x }_{ \Hcal } = 1 } } \;\;\;
% \norm{ T x }_{ \Hcal' }$ \\[0.4em]
% \Powin
% $\displaystyle \sup_{ \substack{ \norm{ x }_{ \Hcal } = 1 } }
% \norm{ T x }_{ \Hcal' }$ \\[0.4em]
% \StrWd{50}{17} \\
% \Jest  is~positive defined \\
% \Powin is positive defined and well defined \\
% \StrWd{51}{13} \\
% \Jest
% $\displaystyle \int_{ M_{ 2 } } \bigg( \int_{ M_{ 1 } }
% \overline{ f( x, y ) } \varphi_{ k }( x ) \,
% d\mu_{ 1 }( x ) \bigg) \quad \psi_{ l }( y ) \, d\mu_{ 2 }( y )$ \\
% \Powin
% $\displaystyle \int_{ M_{ 2 } } \bigg( \, \int_{ M_{ 1 } }
% \overline{ f( x, y ) } \varphi_{ k }( x ) \, d\mu_{ 1 }( x ) \bigg)
% \psi_{ l }( y ) \, d\mu_{ 2 }( y )$


% \vspace{\spaceTwo}
% % ############################









% ####################################################################
% ####################################################################
% Bibliography

\printbibliography






% ############################

% Koniec dokumentu
\end{document}
