\RequirePackage[l2tabu, orthodox]{nag}
% Autor: Kamil Ziemian
% Korekta: Wojciech Dyba
\documentclass[a4paper,11pt]{article}
\usepackage[utf8]{inputenc}
\usepackage[polish]{babel}
\usepackage[MeX]{polski}
\usepackage{microtype}
\let\lll\undefined
\usepackage{amsmath}
\usepackage{amsfonts, amssymb, amscd, amsthm}
\usepackage{upgreek} % Lepsze greckie czcionki. Przyklad skladni: pi = \uppi
% \usepackage{txfonts} % Inne ulepszenie greckich liter. Przyklad skladni:
% pi = \piup.
% \usepackage{latexsym} % Więcej symboli.
% \usepackage{textcomp} % Pakiet z dziwnymi symbolami.
\usepackage{indentfirst}
\usepackage{slashed} % Pozwala pisać slash Feynmana.
\usepackage{graphicx} % Pozwala wstawiać grafikę.
\usepackage{titlesec}
%\usepackage{url} % Pozwala pisać ładnie znak ~.
\usepackage{xcolor}
\usepackage{calrsfs}
\usepackage{vmargin}
%----------------------------------------------------------------------
% MARGINS
%----------------------------------------------------------------------
\setmarginsrb
{ 0.7in}  % left margin
{ 0.6in}  % top margin
{ 0.7in}  % right margin
{ 0.8in}  % bottom margin
{  20pt}  % head height
{0.25in}  % head sep
{   9pt}  % foot height
{ 0.3in}  % foot sep
\usepackage{hyperref}
% ########################################



% ####################
\renewcommand{\arraystretch}{1.2}



% ########################################

\newtheorem{twr}{Twierdzenie}
\newtheorem{defin}{Definicja}
\newtheorem{wni}{Wniosek}% \usepackage{cleveref}


% ##############################
% Dodatkowe ustawienia pod języka polski
\titlelabel{\thetitle.\quad}


% ##############################

\newcommand{\spaceOne}{2em}
\newcommand{\spaceTwo}{1em}
\newcommand{\spaceThree}{0.5em}

% Ładniejszy zbiór pusty
\let\oldemptyset\emptyset
\let\emptyset\varnothing


% ####################
% Uproszczenia w pisaniu tekstu
\newcommand{\ld}{\ldots}

% Slashe w tekście
\newcommand{\tbs}{\textbackslash}

% Podstawowe oznaczenia matematyczne
\newcommand{\fr}{\frac}
\newcommand{\tfr}{\tfrac}
\newcommand{\tr}{\textrm}

% Oznaczenia ,,nad i pod''. Wymyśl lepszą nazwę
\newcommand{\til}{\tilde}
\newcommand{\ul}{\underline}
\newcommand{\ol}{\overline}
\newcommand{\wh}{\widehat}
\newcommand{\wt}{\widetilde}

% Trzcionki matematyczne
\newcommand{\mr}{\mathrm}
\newcommand{\mb}{\mathbb}
\newcommand{\mc}{\mathcal}
\newcommand{\mf}{\mathfrak}
\newcommand{\mbf}{\mathbf}

% Strzałki
\newcommand{\ra}{\rightarrow}
\newcommand{\Ra}{\Rightarrow}
\newcommand{\lra}{\longrightarrow}
\newcommand{\xra}{\xrightarrow}

\newcommand{\wtw}{wtedy i~tylko wtedy}


% ####################
% Litery greckie
\newcommand{\al}{\alpha}
\newcommand{\be}{\beta}
\newcommand{\ga}{\gamma}
\newcommand{\del}{\delta}
\newcommand{\Del}{\Delta}
\newcommand{\la}{\uplambda}
\newcommand{\eps}{\epsilon}
\newcommand{\veps}{\varepsilon}
\newcommand{\vp}{\varphi}
\newcommand{\om}{\omega}
\newcommand{\Om}{\Omega}
\newcommand{\si}{\sigma}
\newcommand{\Si}{\Sigma}
\newcommand{\tet}{\theta}

% Standardowe oznaczenia literowe
\newcommand{\N}{\mb{N}}
\newcommand{\R}{\mb{R}}
\newcommand{\C}{\mb{C}}
\newcommand{\D}{\mc{D}}
\newcommand{\Hc}{\mc{H}}
\newcommand{\Lc}{\mc{L}}
\newcommand{\Rn}{\R^{ n }}
\newcommand{\Rc}{\mc{R}}
\newcommand{\Cc}{\mc{C}}
\newcommand{\lc}{\mc{l}}

% Mniej używane oznaczenia literowe
\newcommand{\B}{\mc{B}}
\newcommand{\Oc}{\mc{O}}
\newcommand{\Rp}{\R_{ + }}

% ####################
% Standardowe matematyczne symbole literowe
% Mathbb
% \newcommand{\C}{\mb{C}}
% \newcommand{\N}{\mb{N}}
% \newcommand{\R}{\mb{R}}
% \newcommand{\Rn}{\R^{ n }}
% \newcommand{\Rp}{\R_{ + }}

% % Mathcal
% \newcommand{\B}{\mc{B}}
% \newcommand{\D}{\mc{D}}
% \newcommand{\Fc}{\mc{F}}
% \newcommand{\M}{\mc{M}}
% \newcommand{\Rc}{\mc{R}}
% \newcommand{\Cc}{\mc{C}}
% \newcommand{\Oc}{\mc{O}}
% \newcommand{\T}{\mc{T}}


% ####################
% Teoria mnogości
\newcommand{\set}[1]{\{ #1 \}}
\newcommand{\es}{\emptyset}
\newcommand{\sset}{\subset}
\newcommand{\setm}{\setminus}
\newcommand{\ti}{\times}


% ####################
% Algebra
\newcommand{\Real}{\mf{Re}}
\newcommand{\Imag}{\mf{Im}}
\newcommand{\ot}{\otimes}
\newcommand{\Tr}{\mr{Tr}}


% ####################
% Analiza matematyczna

% Granice
\newcommand{\Lim}{\lim\limits}
\newcommand{\Liminf}{\ul{\lim}}
\newcommand{\Limsup}{\ol{\lim}}

% Sumy
\newcommand{\Sum}{\sum\limits}

% Różniczkowanie i pochodne
\newcommand{\dk}{\, d} % Różniczka Kończąca całkę
\newcommand{\pr}{\partial}
\newcommand{\de}{\mr{d}}
\newcommand{\dd}[3]{\fr{ \de^{ #1 } { #2 } }{ \de { #3 }^{ #1 } }}
\newcommand{\pd}[3]{\fr{ \pr^{ #1 } { #2 } }{ \pr { #3 }^{ #1 } }}

% Powszechnie używane symbole
\newcommand{\Cinfty}{\Cc^{ \infty }}

% Całki i całkowanie
\newcommand{\dx}{\de x}
\newcommand{\Int}{\int\limits}
\newcommand{\IntSet}[3]{\Int_{ #1 } #2 \, d#3} % Całka po zbiorze.
\newcommand{\IntL}[3]{ \Int_{ { #1 } }^{ { #2 } } \de#3 \; }
\newcommand{\IntCaJ}[2]{ \Int #1 \, \de#2 } % Juz z calkowaniem
% Całka Jeden
\newcommand{\IntCaD}[2] { \Int #1 \, d#2 } % Całka dwa
\newcommand{\IntFi}[2]{ \Int \de#1 \, #2 } % Bardziej Fizyczna notacja;)
\newcommand{\IntWie}[3]{ \Int_{ #1 } \de^{ #2 }#3 \; } % Wielowymiarowa
\newcommand{\IntDo}[5]{ \Int_{ #1 } \de^{ #2 }#3\, \de^{ #4 } #5 \; }

% ####################
% Całki w określonych granicach
\newcommand{\IntA}[1]{\Int_{ -\infty }^{ +\infty } \de #1 \;}
\newcommand{\IntB}[1]{\int_{ \R } \de #1 \;}
\newcommand{\IntC}[2]{\int_{ \R } #1 \, \de #2}
\newcommand{\IntD}[1]{\Int_{ 0 }^{ +\infty } \de #1 \;}


% ####################
% Analiza funkcjonalna
\newcommand{\da}{\dagger}

% Brakety
% W analizie funkcjonalnej też często się je stosuje
\newcommand{\lket}{\langle}
\newcommand{\rket}{\rangle}

% Przestrzenie L^{ p }
\newcommand{\Lj}{L^{ 1 }}
\newcommand{\Ld}{L^{ 2 }}
\newcommand{\Lp}{L^{ p }}
\newcommand{\LIj}{\Lc^{ 1 }}
\newcommand{\LIp}{\Lc^{ p }}
\newcommand{\LdJ}{\Ld( \R, \de \mu )}
\newcommand{\LdT}{\Ld( \R^{ 3 }, \de \mu )}
\newcommand{\Ldlim}{L^{ 2 }-\Lim}


% ####################
% Wartość bezwzględna i normy
\providecommand{\absj}[1]{\lvert #1 \rvert}
\providecommand{\absd}[1]{\left| #1 \right|}
\providecommand{\absd}[1]{\left| \, #1 \, \right|}
\newcommand{\norm}[1]{\left|\left| #1 \right|\right|}


% ####################
% Edycja tekstu
\newcommand{\tb}{\textbf}
\newcommand{\noi}{\noindent}
\newcommand{\start}{\noi \tb{--} {}}
\newcommand{\Str}[1]{\tb{Str. #1.}}
\newcommand{\StrWg}[2]{\tb{Str. #1, wiersz #2.}}
\newcommand{\StrWd}[2]{\tb{Str. #1, wiersz #2 (od dołu).}}
\newcommand{\Dow}{\tb{Dowód}}
\newcommand{\Center}[1]{\begin{center} #1 \end{center}}
\newcommand{\CenterTB}[1]{\Center{\tb{#1}}}
\newcommand{\Jest}{\tb{Jest: }}
\newcommand{\Pow}{\tb{Powinno być: }}
\newcommand{\red}[1]{{\color{red} #1}}
\newcommand{\Prze}{{\color{red} Przemyśl.}}
\newcommand{\Pop}{{\color{red} Popraw.}}
\newcommand{\Prob}{{\color{red} Problem.}}
\newcommand{\Dok}{{\color{red} Dokończ.}}
\newcommand{\Pyt}{{\color{red} Pytanie.}}
\newcommand{\Main}[1]{ \begin{center} {\LARGE \tb{#1} } \end{center} }
\newcommand{\Field}[1]{ \begin{center} {\Large \tb{#1} } \end{center} }
\newcommand{\Work}[1]{ \begin{center} {\large \tb{#1}} \end{center} }


% ##############################
% Oznaczenia dla Reeda, Simona

\newcommand{\ci}{\circ}

% Przestrzenie Hilberta
\newcommand{\lcd}{\ell_{ 2 }} % l Caligraphy Dwa
\newcommand{\SP}[2]{( #1, \, #2 )} % Scalar product.
\newcommand{\dket}[2]{\lket #1, \, #2 \rket} % Dirac ket

% % Sumy
% \newcommand{\Sum}{\sum\limits}

% % Różniczkowanie i pochodne
% \newcommand{\pr}{\partial}
% \newcommand{\de}{\mr{d}}
% \newcommand{\dd}[3]{\fr{ \de^{ #1 } { #2 } }{ \de { #3 }^{ #1 } }}
% \newcommand{\pd}[3]{\fr{ \pr^{ #1 } { #2 } }{ \pr { #3 }^{ #1 } }}

% % Całki
% \newcommand{\Int}{\int\limits}
% \newcommand{\IntA}[1]{\Int_{ -\infty }^{ +\infty } \de #1 \;}
% \newcommand{\IntB}[1]{\int_{ \R } \de #1 \;}
% \newcommand{\IntC}[1]{\Int_{ 0 }^{ +\infty } \de #1 \;}
% \newcommand{\IntCa}[2]{ \Int #1 \, \de#2 } % Juz z calkowaniem
% \newcommand{\IntFi}[2]{ \Int \de#1 \, #2 } % Bardziej Fizyczna notacja;)
% \newcommand{\IntWie}[3]{ \int_{ #1 } \de^{ #2 }#3 \; } % Wielowymiarowa.
% \newcommand{\IntDo}[5]{ \Int_{ #1 } \de^{ #2 }#3\, \de^{ #4 } #5 \; }
% \newcommand{\IntL}[3]{ \int\limits_{ { #1 } }^{ { #2 } } \d#3 \; }



% #####################################################################
\begin{document}
% #####################################################################



% ########################################
\Field{
  M. Reed, B. Simon \\
  ,,Methods~of Modern Mathematical Physics~I:\\
  Functional Analysis'', \cite{ReedSimon80} }
% ########################################


\CenterTB{Uwagi}

\noi \tb{Rozdział I.}

\vspace{\spaceThree}

\start Nigdzie nie została podana definicja $\sup$ i~$\inf$ ani
w~przypadku ogólnych relacji porządkujących, ani w~przypadku liczb
rzeczywistych. % Sprawdzone.

% \start Nie~zdefiniowano zbiorów zwartych, ani~nie~wskazano, że~zbiór
% domknięty i~ograniczony w $\mathbb{R}^{ n }$ jest zwarty.


\start Nie podana została definicja sumy miar borelowskich,
ani~nie~zostało udowodnione, że funkcja całkowalna osobno względem
miar $\mu_{ 1 }$ i~$\mu_{ 2 }$ jest całkowalna względem ich sumy.
Uzupełnienie wykładu w~tym miejscu nie jest trudne i~przedstawiamy je
poniżej.

Dla dowolnych dwóch miar dodatnich $\mu_{ 1 }$ i~$\mu_{ 2 }$,
niekoniecznie borelowskich, określonym na wspólnym $\si$\dywiz
pierścieniu $\Rc$, definiujemy \tb{sumę miar} jako:
\begin{equation}
  \label{eq:RSI1}
  \begin{split}
    (\mu_{ 1 } + \mu_{ 2 })( A ) &= \mu_{ 1 }( A ) + \mu_{ 2 }( A ),
    \quad \forall A \in \mc{R}, \\
    (\mu_{ 1 } + \mu_{ 2 })( A ) &= +\infty, \quad \tr{jeśli } \mu_{ 1
    }( A ) = +\infty \tr{ lub } \mu_{ 2 }( A ) = +\infty.
  \end{split}
\end{equation}
Analogicznie określmy sumę $n$ miar dodatnich. Trzeba teraz pokazać,
że~jest to w~istocie miara.

% ##############################
\begin{twr}
  Suma skończonej liczby miar dodatnich $\mu_{ i }$, $i = 1, \ld, n$,
  określonych na wspólnym pierścieniu $\Rc$, jest miarą dodatnią
  określoną na tym samym pierścieniu.

  Jeżeli funkcja $f$ jest całkowalna względem względem wszystkich miar
  $\mu_{ i }$, to jest też całkowalna względem ich sumy i~zachodzi:
  \begin{equation}
    \label{eq:1}
    \int f( x ) \, \left( \Sum_{ i = 1 }^{ n } d\mu_{ i }( x ) \right)
    = \Sum_{ i = 1 }^{ n } \IntCaD{ f( x ) }{ \mu_{ i }( x ) }.
  \end{equation}
\end{twr}
% ##############################

\begin{proof}
  Jeśli pokażemy to twierdzenie dla $n = 2$, przypadku ogólnego można
  będzie można dowieść przez prostą indukcję. Suma miar jest określona
  na~wspólnym pierścieniu $\Rc$, zaś poza warunkiem na~przeliczalną
  addytywność miary rozłącznych zbiorów $A_{ i }$, pozostałe własności
  miary dodatniej wychodzą od razu.

  Załóżmy teraz, że~miary wszystkich tych zbiorów względem $\mu_{ 1 }$
  i~$\mu_{ 2 }$ są skończone. Na podstawie podstawowych twierdzeń
  o~sumowaniu szeregów dodatnich, mamy
  \begin{equation*}
    \begin{split}
      (\mu_{ 1 } + \mu_{ 2 })\left( \bigcup_{ i = 1 }^{ \infty } A_{ i
        } \right) &= \mu_{ 1 }\left( \bigcup_{ i = 1 }^{ \infty } A_{
          i } \right) + \mu_{ 2 }\left( \bigcup_{ i = 1 }^{ \infty }
        A_{ i } \right) = \sum_{ i = 1 }^{ \infty } \mu_{ 1 }( A_{ i }
      )
      + \sum_{ i = 1 }^{ \infty } \mu_{ 2 }( A_{ i } ) \\
      &= \sum_{ i = 1 }^{ \infty } ( \mu_{ 1 }( A_{ i } ) + \mu_{ 2 }(
      A_{ i } ) ) = \sum_{ i = 1 }^{ \infty } ( \mu_{ 1 } + \mu_{ 2 }
      )( A_{ i } ).
    \end{split}
  \end{equation*}
  Jeśli chodź dla jednego zbioru zachodzi
  $\mu_{ n }( A_{ i } ) = +\infty$, to wówczas obie strony są równe
  $+\infty$. Problem, czy można dodać dwie miary, które nie są
  dodatnie wymaga osobnych rozważań. \Dok

  Rozpatrzmy teraz problem całkowania względem sumy miar $\mu_{ i }$
  funkcji $f$, która jest całkowalnej względem każdej miar z~osobna,
  ograniczymy~się przy tym dla przypadku funkcji określonej na~$\R$.
  Jeżeli $f \geq 0$, wówczas zachodzi równość:
  \begin{equation}
    \label{eq:RSI2}
    \begin{split}
      \sum_{ n }( f )_{ \mu_{ 1 } + \mu_{ 2 } } =& \Sum_{ m = 0 }^{
        \infty } \fr{ m }{ n } ( \mu_{ 1 } + \mu_{ 2 } )\left( f^{ -1
        }\left[ \left[ \fr{ m }{ n }, \fr{ m + 1 }{ n } \right)
        \right] \right) = \Sum_{ m = 0 }^{ \infty } \left\{ \fr{ m }{
          n } \mu_{ 1 }\left( f^{ -1 }\left[ \left[ \fr{ m }{ n },
              \fr{ m + 1 }{ n } \right) \right] \right) \right. \\
      &+ \left. \fr{ m }{ n } \mu_{ 2 }\left( f^{ -1 }\left[ \left[
              \fr{ m }{ n }, \fr{ m + 1 }{ n } \right) \right] \right)
      \right\} = \Sum_{ n }( f )_{ \mu_{ 1 } } + \Sum_{ n }( f )_{
        \mu_{ 2 } }.
    \end{split}
  \end{equation}
  Znowu ponieważ wszystkie wyrazy szeregów są dodatnie, więc powyższe
  przekształcenia są poprawne.

  Jako, że~w~podejściu prezentowanym przez Reeda i~Simona, całkowanie
  funkcji zespolonych sprowadza~się do całkowania funkcji dodatnich,
  widzimy więc, że~funkcja całkowalna względem każdej z~miar
  $\mu_{ i }$, $i = 1, \ld, n$~tzn.
  \begin{align*}
    \IntCaD{ f }{ \mu_{ 1 } } < +\infty \;\tr{  oraz  }
    \IntCaD{ f }{ \mu_{ 2 } } < +\infty,
  \end{align*}
  jest też całkowalna względem ich sumy i~na podstawie wzoru
  \eqref{eq:RSI2}, zachodzi równość:
  \begin{equation}
    \label{eq:RSI3}
    \IntCaD{ f }{ ( \mu_{ 1 } + \mu_{ 2 } ) } = \IntCaD{ f }{ \mu_{ 1 } }
    + \IntCaD{ f }{ \mu_{ 2 } }.
  \end{equation}

\end{proof}


\tb{Uwaga.} Należy tą sytuację odróżnić od~przypadku, gdy mamy dwie
funkcje $f_{ 1 }$ i~$f_{ 2 }$, całkowalne odpowiednio względem miar
$\mu_{ 1 }$ i~$\mu_{ 2 }$, i~chcemy powiedzieć coś o~całkowalności ich
sumy. Wtedy bowiem może okazać się, że~funkcja np.~$f_{ 1 }$ jest
całkowalna względem $\mu_{ 1 }$ bo~miara zbioru na którym przyjmuje
duże wartości jest mała względem $\mu_{ 1 }$. Jeżeli jednak miara
$\mu_{ 2 }$ tego zbioru jest duża, funkcja $f_{ 1 }$ może być względem
niej niecałkowalna.

\tb{Przykład.} Weźmy funkcję na prostej rzeczywistej
\begin{equation*}
  f( x ) =
  \begin{cases}
    +\infty, \quad x = 0. \\
    \quad \;0, \quad x \neq 0, \\
  \end{cases}
\end{equation*}
Jako, że~$f \equiv 0$, jest ona całkowalna względem miary Lebesgue'a
$dx$ i~jej całka wynosi~0. Jeżeli teraz dodamy do miary Lebesgue'a
miarę Diraca skupioną w~0, oznaczmy ją $d\del_{ 0 }$, to $f$ będzie
niecałkowalna względem miary $dx + d\del_{ 0 }$. % Sprawdzone.

\vspace{\spaceTwo}



\noi \tb{Rozdział~II.}

\vspace{\spaceThree}

\start \tb{Ciągłość ciągowa funkcji na~przestrzeniach metrycznych.}
W~tym rozdziale napotykamy kilkakrotnie następującą sytuacje. Mamy
funkcje ciągłą $f( x )$ oraz~ciąg $x_{ n_{ 1 }, \ld, \, n_{ m } }$
i~próbujemy wejść z~granicą pod funkcję, innymi słowy chcemy, aby
zachodziło\footnote{Rozpatrujemy tu tylko przypadek skończonej liczby
  indeksów liczby naturalnych, bo~na~razie nie znalazłem potrzeby
  rozpatrywać ciągów o~nieskończonej ilości indeksów lub~takich które
  nie są indeksowane liczbami naturalnymi.}
\begin{equation*}
  \Lim_{ \substack{ n_{ 1 } \ra +\infty \\ \cdots \\  n_{ m } \ra +\infty } }
  f( x_{ n_{ 1 }, \ld, \, n_{ m } } )
  = f( \Lim_{ \substack{ n_{ 1 } \ra +\infty \\ \cdots \\ n_{ m } \ra +\infty } }
  x_{ n_{ 1 }, \ld, \, n_{ m } } ).
\end{equation*}
Jeżeli funkcja jest \emph{ciągła topologicznie}, czyli przeciwobraz
każdego zbioru otwartego jest otwarty, to poniżej udowodnimy, że~tak
równość zachodzi. Pytanie czy jeśli funkcja jest \emph{ciągowo
  ciągła}, czyli
\begin{equation*}
  \Lim_{ n \ra +\infty } f( x_{ n } ) = f( \Lim_{ n \ra +\infty } x_{ n } ),
\end{equation*}
dla każdego zbieżnego ciągu $x_{ n }$, to~czy twierdzenie będzie
zachodzić również w~tej sytuacji? W~przypadku ogólnym nie wiem, czy
tak jest, bowiem gdy przestrzeń topologiczna nie~ma przeliczalnej bazy
otoczeń, to~ciągi postaci $x_{ n }$ nie zawierają pełnej wiedzy
o~topologii. Udowodnimy jednak, że~w~przypadku przestrzeni metrycznych
żądana własność zachodzi.

Na początku będzie nam potrzebna definicja granicy ciągu
wieloindeksowego. Najprostsza definicja tej granicy jest
następująca\footnote{Granica ciągu wieloindeksowego powinna być
  szczególnym przypadkiem granic indeksowanych zbiorami skierowanymi
  które~są omawiane w~książce Maurina \cite{Maurin74}.}. Niech
$X$~będzie przestrzenią metryczną, a~$x_{ n_{ 1 }, \ld, \, n_{ m } }$
ciągiem wieloindeksowym w~$X$. $x_{ 0 }$ jest granicą ciągu
$x_{ n_{ 1 }, \ld, \, n_{ m } }$ \wtw, gdy
\begin{equation*}
  ( \forall \, \veps > 0) ( \exists \, N )
  ( \forall \, n_{ 1 } > N, \ld, \, n_{ m } > N ) \;\,
  d( x_{ n_{ 1 }, \ld, \, n_{ m } }, \, x_{ 0 } ) < \veps.
\end{equation*}

\begin{twr}
  Niech $X$ i~$Y$ będą przestrzeniami metrycznymi,
  $x_{ n_{ 1 }, \ld, n_{ m } }$ ciągiem elementów przestrzeni $X$.
  Jeśli funkcja $f: X \ra Y$ jest ciągła topologicznie i~istnieje
  granica ciągu
  $\Lim_{ \substack{ n_{ 1 } \ra +\infty \\ \cdots \\ n_{ m } \ra
      +\infty } } x_{ n_{ 1 }, \ld, \, n_{ m } } = x_{ 0 }$, to
  \begin{equation}
    \Lim_{ \substack{ n_{ 1 } \ra +\infty \\ \cdots \\  n_{ m } \ra +\infty } }
    f( x_{ n_{ 1 }, \ld, \, n_{ m } } )
    = f( \Lim_{ \substack{ n_{ 1 } \ra +\infty \\ \cdots \\ n_{ m } \ra
        +\infty } } x_{ n_{ 1 }, \ld, \, n_{ m } } ).
  \end{equation}
\end{twr}

\begin{proof}
  Dla uproszczenia notacji będziemy rozpatrywać ciąg dwuindeksowy
  $x_{ n, \, m }$. Ustalmy $\veps > 0$, szukamy teraz takiego $N$,
  by~$d( f( x_{ n, \, m } ),\, f( x_{ 0 } ) ) < \veps$. Ponieważ $f$
  jest ciągła istnieje taka $\del > 0$, \linebreak
  że~$d_{ Y }( f( x ),\, f( x_{ 0 } ) ) < \veps$
  dla~$d_{ X }( x,\, x_{ 0 } ) < \del$. Wystarczy więc dobrać tak $N$,
  żeby dla $n > N$, $m > N$, $x_{ n, \, m }$ znajdował~się
  w~odległości mniejszej niż~$\del$ od~$x_{ 0 }$. Takie $N$ istnieją
  na~mocy definicji granicy ciągu dwuindeksowego.
\end{proof}

\begin{wni}
  Niech $X$ i~$Y$ będą przestrzeniami metrycznymi. Jeżeli $f: X \ra Y$
  jest ciągowo ciągła, to z~granicą ciągu wieloindeksowego można wejść
  pod funkcje.
\end{wni}

\begin{proof}
  W~przestrzeni metrycznej funkcja jest ciągowo ciągła \wtw, gdy jest
  ciągła topologicznie, wystarczy więc zastosować poprzednie
  twierdzenie.
\end{proof}

\start \tb{Odwzorowania wieloliniowe ciągłe i~szeregi.} W~tym
rozdziale potrzebujemy kilka razy pokazać równość następującego typu.
Jeżeli $f = \sum_{ l = 1 }^{ \infty } c_{ l } \vp_{ l }$,
$g = \sum_{ k = 1 }^{ \infty } d_{ k } \psi_{ k }$, to
\begin{equation*}
  \Lim_{ K,\, M \ra +\infty } \Sum_{ \substack{ l < L \\ k < K } }
  c_{ l } d_{ k } \, (\vp_{ l } \ci \psi_{ k } ) = f \ci g,
\end{equation*}
gdzie $\ci$ jest jakiegoś rodzaju iloczynem, a~symbol
$K, M \ra +\infty$ rozumiem jak przy podanej wcześniej definicji
granicy ciągu wieloindeksowego. Aby~nadać tym operacjom sens, musimy
najpierw zdefiniować odwzorowania wieloliniowe ograniczone.

Niech $X_{ 1 }, \ld, X_{ n }, Y$ będą przestrzeniami unormowanymi nad
ciałem $\R$ albo~$\C$, zaś
\begin{equation*}
  B: X_{ 1 } \ti \ld \ti X_{ n } \ra Y
\end{equation*}
odwzorowaniem wieloliniowym. Odwzorowanie takie jest ograniczone jeśli
istnieje taka stała $C \in \Rp$, że~dla wszystkich
$x_{ 1 } \in X_{ 1 }, \ld, x_{ n } \in X_{ n }$ zachodzi
\begin{equation}
  \norm{ B( x_{ 1 }, \ld, x_{ n } ) }_{ Y } \leq C
  \norm{ x_{ 1 } }_{ X_{ 1 } } \cdot \ld \cdot \norm{ x_{ n } }_{ X_{ n } }.
\end{equation}
Podstawowe ich własności przedstawił Schwartz w~swoim ,,Kursie analizy
matematycznej'', str.~102, \cite{Sch79}. W~szczególności tak jak dla
odwzorowania liniowego odwzorowanie wieloliniowe jest ciągłe \wtw, gdy
jest ograniczone, analogicznie również definiuje~się normę
$\norm{ B }$ odwzorowania wieloliniowego~$B$. Dysponując już tymi
pojęciami możemy sformułować potrzebne nam twierdzenie.

\begin{twr}
  Jeśli $B: X_{ 1 } \ti \ld \ti X_{ n } \ra Y$ jest odwzorowanie
  wieloliniowym ograniczonym przestrzeni unormowanych
  $X_{ 1 }, \ld, X_{ n }, Y$, zaś szeregi
  $\sum_{ l } c_{ 1, \, l } \, \vp_{ 1, \, l }, \ld, \sum_{ l } c_{ n,
    \, l } \, \vp_{ n, \, l }$, $\vp_{ i, \, l } \in X_{ i }$,
  $l \in \N$, są~warunkowo zbieżne do~elementów
  $\eta_{ 1 }, \ld, \eta_{ n }$, to zachodzi równość
  \begin{equation*}
    \Sum_{ l_{ 1 }, \ld, l_{ n } } c_{ 1, \, l_{ 1 } } \cdot \ld \cdot
    c_{ n, \, l_{ n } } B( \vp_{ 1, \, l_{ 1 } }, \ld,
    \vp_{ n, \, l_{ n } } ) = B( \eta_{ 1 }, \ld, \eta_{ n } ),
  \end{equation*}
  przy czym sumę tą należy rozumieć jako
  \begin{equation*}
    \Sum_{ l_{ 1 }, \ld, l_{ n } } c_{ 1, \, l_{ 1 } } \cdot \ld \cdot
    c_{ n, \, l_{ n } } \, B( \vp_{ 1, \, l_{ 1 } }, \ld,
    \vp_{ n, \, l_{ n } } ) = \Lim_{ N \ra +\infty }
    \Sum_{ \substack{ l_{ 1 } < N \\ \cdots \\ l_{ n } < N } }
    c_{ 1, \, l_{ 1 } } \cdot \ld \cdot c_{ n, \, l_{ n } } \,
    B( \vp_{ 1, \, l_{ 1 } }, \ld, \vp_{ n, \, l_{ n } } ).
  \end{equation*}
\end{twr}

\begin{proof}
  Aby~nie ugrząźć w~,,nieskończonych'' ciągach indeksów dowód
  przeprowadzimy dla przypadku $n = 2$, oznaczmy przy tym
  $\vp_{ 1,\, l } = \vp_{ l }$, $\vp_{ 2, \, l } = \psi_{ l }$,
  $c_{ 1, \, l } = c_{ l }$, $c_{ 1, \, l } = d_{ l }$,
  $\eta_{ 1 } = \vp$ oraz~$\eta_{ 2 } = \psi$. Ponieważ
  $\vp = \sum_{ l } c_{ l } \, \vp_{ l }$, więc istniej takie $L$,
  że~$\vp = \sum_{ k < K } c_{ k } \, \vp_{ k } + \vp_{ R }$, gdzie
  $\norm{ \vp_{ R } }_{ X_{ 1 } } \leq \veps$. Analogicznie
  $\psi = \sum_{ l < L } d_{ l } \psi_{ l } + \psi_{ R }$,
  $\norm{ \psi_{ R } }_{ X_{ 2 } } \leq \veps$. Ponieważ suma
  $\sum_{ k < K } c_{ k } \vp_{ k }$ jest zbieżna do~$\vp$, wynika,
  że~ ciąg
  $\norm{ \sum_{ k < K } c_{ k } \vp_{ k } }_{ X_{ 1 } } \ra \norm{
    \vp }_{ X_{ 1 } }$, jest więc ograniczony. Dobierzmy $M$ tak~by
  było wspólnym ograniczeniem ciągów
  $\norm{ \sum_{ k < K } c_{ k } \vp_{ k } }_{ X_{ 1 } }$
  i~$\norm{ \sum_{ l < L } d_{ l } \psi_{ l } }_{ X_{ 2 }
  }$\footnote{To ograniczenie to~wszystko czego potrzebujemy
    w~dowodzie, a~aby je otrzymać wystarczy warunkowa zbieżność
    omawianych szeregów.}, niech teraz $N = \max( L, K )$. Rozpatrzmy
  wyrażenie
  \begin{equation*}
    \begin{split}
      & || B( \vp, \psi ) - \Sum_{ k, \, l < N } c_{ k } d_{ l } B(
      \vp_{ k }, \psi_{ l } ) ||_{ Y } = || B( \Sum_{ k < K } \vp_{ k
      } + \vp_{ R }, \Sum_{ l < L } d_{ l } \psi_{ l } + \psi_{ R }) -
      \Sum_{ k < K }\Sum_{ l < L } c_{ k } d_{ l } B( \vp_{ k },
      \psi_{ l } ) ||_{ Y } \\
      &= || B( \Sum_{ l < L } \vp_{ l }, \psi_{ R } ) + B( \vp_{ R },
      \Sum_{ k < K } \psi_{ k } ) + B( \vp_{ R }, \psi_{ R } ) ||_{ Y
      } \leq || B( \Sum_{ l < L } \vp_{ l }, \psi_{ R } ) ||_{ Y }
      + || B( \vp_{ R }, \Sum_{ k < K } \psi_{ k } ) ||_{ Y } \\
      &+ || B( \vp_{ R }, \psi_{ R } ) ||_{ Y } \leq 2 \norm{ B } M
      \veps + \norm{ B } \veps^{ 2 }.
    \end{split}
  \end{equation*}
  Dobierając odpowiednio duże $N$ możemy uczynić tą różnicę dowolnie
  małą co kończy dowód.
\end{proof}

Mogłoby~się wydawać, że~powyższe twierdzenie nie może być prawdziwe,
bowiem wiemy, iż~jeśli szeregi liczb rzeczywistych, a~liczby
rzeczywiste~są przestrzenią Banacha, nie są bezwzględnie zbieżne, to
suma ich iloczynu zależy od kolejności sumowania poszczególnych
wyrazów\footnote{Co jest prawdą pod warunkiem, że~iloczyn szeregów
  warunkowo zbieżnych nie może być bezwzględnie zbieżny. \Dok}.

Problem jest jednak pozorny, bowiem twierdzenie powyższe dotyczy tylko
szczególnego sposobu sumowania, które w~zapisie tablicowym jakie używa
Fichtenholza str.~??? \cite{Fichtenholz04}) odpowiada ,,sumowaniu
po~coraz większych prostokątach''. Inaczej mówiąc, sum które zawsze
da~się przedstawić jako
\begin{equation*}
  \Sum_{ \substack{ k < K \\ l < L \\ } } a_{ k } b_{ l }
  = ( \Sum_{ k < K } a_{ k } ) ( \Sum_{ l < L } b_{ l } ).
\end{equation*}
Zbieżność tego typu sum do~$( \Sum a_{ l } ) ( \Sum b_{ k } )$ jest
oczywista. Dopiero przy bardziej skomplikowanym wyborze sposobu
sumowania, przykłady ponownie można znaleźć u~Fichtenholza, kolejność
sumowania ma znaczenie. % Sprawdzone.

\vspace{\spaceOne}



\noi \tb{Konkretne strony.}

\vspace{\spaceThree}

% \start \Str{7} Warto w~tym miejscu zauważyć, że~każdy domknięty
% podzbiór przestrzeni zupełnej jest zupełny. Zbyt dużo materiały w
% książce brakuje, aby móc z czysty sumieniem wstawić powyższy
% komentarz. Aby wiedzieć, co to znaczy, że podzbiór przestrzeni
% metrycznej jest zupełny, należy wprowadzić topologię indukowaną.


\start \Str{9} Nie wykazano, jedynie stwierdzono jednym słowem
,,thus'', że~obie podane definicje~są równoważne. W~istocie, jak
powszechnie wiadomo z~pracy dotyczących analizy funkcjonalnej:
\begin{equation*}
  \norm{ T } = \sup_{ \norm{ x }_{ 1 } \leq 1 } \fr{ \norm{ T x }_{ 2 } }
  { \norm{ x }_{ 1 } }
  = \sup_{ \norm{ x }_{ 1 } = 1 } \fr{ \norm{ T x }_{ 2 } }
  { \norm{ x }_{ 1 } } = \inf \{ \, C \, | \, \forall x,
  \norm{ T x }_{ 2 } \leq C \norm{ x }_{ 1 } \}.
\end{equation*}
Elegancki dowód tego faktu, można znaleźć np. w~\cite{Chmielinski04},
str.~115--116. % Sprawdzone.

\vspace{\spaceThree}


\start \Str{10} W~dowodzie twierdzenia B.L.T. skorzystanie z~$\Limsup$
jest zupełnie niepotrzebne, poza tym utrudnia dowód,
że~$|| \tilde{ T } || = \norm{ T }$. Wystarczy bowiem skorzystać
z~twierdzenia o~zachowaniu nierówności przy~przechodzeniu do~granicy,
by otrzymać pożądaną nierówność
\begin{equation*}
  \Lim_{ n \ra \infty } \norm{ T x_{ n } }_{ 2 }
  \leq \Lim_{ n \ra \infty } \norm{ T } \norm{ x_{ n } }_{ 1 }
  = \norm{ T } \norm{ x }_{ 1 },
\end{equation*}
Wynika stąd od razu że~$|| \wt{ T } || \leq \norm{ T }$. Ponieważ
zaś~na~zbiorze $V_{ 1 }$, zachodzi
$|| \wt{ T } x ||_{ 2 } = \norm{ T x }_{ 2 } \leq \norm{ T } \norm{ x
}_{ 1 }$, przy czym $\norm{ T }$ jest minimalną stałą dla~której jest
to~prawdą, więc ze~względu na~to, iż~$V_{ 1 } \subset \wt{ V }_{ 1 }$,
mamy nierówność $\norm{ T } \leq || \wt{ T } ||$, tym samym
$|| \wt{ T } || = \norm{ T }$. % Sprawdzone.

\vspace{\spaceThree}


\start \Str{12} Ponieważ nie została podana definicja podciągu, nie
można było użyć następującej eleganckiej definicji. Liczba $b$~jest
punktem skupienia ciągu $\{ a_{ n } \}$, jeśli $\{ a_{ n } \}$ zawiera
podciąg zbieżny do~$b$. % Sprawdzone.

\vspace{\spaceThree}


\start \Str{12} Zaproponowana na tej stronie alternatywna definicja
$\Limsup$ ma sens tylko dla zbiorów ograniczonych. W~analogiczny
sposób jak pierwotną definicję należy ją rozszerzyć na przypadek
zbiorów nieograniczonych. % Sprawdzone.

\vspace{\spaceThree}


\start \Str{13} Przelicz jawnie, że~funkcje trapezoidalne tworzą ciąg
Cauchy'ego. \Dok

\vspace{\spaceThree}


\start \Str{16} Można podać równoważną, prostszą charakteryzację
zbiorów mierzalnych. Zbiór $M$ jest mierzalny \wtw, gdy
$M \cup B' = B$, gdzie $B, B'$ są pewnymi zbiorami borelowskimi
i~$\mu( B' ) = 0$.

Jeśli $M$ ma taką własność, to oczywiście spełniona jest definicja
pierwotna przy $A_{ 1 } = B_{ 1 } = B'$, $A_{ 2 } = B_{ 2 } = \O$.
Jeśli zaś $M \cup A_{ 1 } = B \cup A_{ 2 }$, to możemy do obu stron
dodać $B_{ 1 } \cup B_{ 2 }$ i~dostajemy,
że~$M \cup B_{ 1 } \cup B_{ 2 } = B \cup B_{ 1 } \cup B_{ 2 }$. Zbiory
$B \cup B_{ 1 } \cup B_{ 2 }$ i~$B_{ 1 } \cup B_{ 2 }$ to zbiory
borelowskie, przy czym $\mu( B_{ 1 } \cup B_{ 2 } ) = 0$.

Inaczej mówiąc, zbiór mierzalny różni~się o~zbiór miary~0 od pewnego
zbioru borelowskiego. % Sprawdzone.

\vspace{\spaceThree}


\start \Str{16} Choć jest to rozdział ,,Wiadomości wstępne''
(``Preliminaries''), to dyskusja całkowania funkcji niedodatnich jest
przeprowadzona zbyt szybko i~pobieżnie, by~uznać ją za~satysfakcję.

Po~pierwsze należałoby zauważyć, że~jeżeli mamy dwie funkcje
$0 \leq g \leq f$, to z~tego, że~$\IntCaD{ f }{ x } < \infty$, wynika
że~$\IntCaD{ g }{ x } < \infty$, co przy przyjętej tu definicji całki
z~funkcji dodatniej nie jest od razu oczywiste. Istotnie, jak
zauważyli Reed i~Simon, przy definicji zwykle spotykanej w~pracach dla
matematyków, tego typu dowody~są prostsze. Wiedząc już jednak o~tym,
widać od~razu, że~jeżeli $\IntCaD{ \absj{ f } }{ x } < \infty$ to
całkowalne~są również $f_{ + }$ i~$f_{ - }$, więc definicja całki
z~funkcji niedodatniej ma sens. Idąc dalej~tą drogą można sformułować
całkiem satysfakcjonującą teorię całki.

Idąc drogą wskazaną przez Schwartza, zobacz \cite{Schwartz79}, będę
zawsze przyjmował, że~mierzalna funkcja dodatnia jest zawsze
całkowalna, choć jej całka możne wynosić $+\infty$. W~podejściu
Schwartza wystarczy tylko dodatniość funkcji, dlatego wprowadza
on~dodatkowe pojęcia całki górnej, która istniej dla każdej funkcji
$f \geq 0$. % Sprawdzone.

% W~tym kontekście warto mieć w~pamięci rozróżnienie jakie podał
% Schwartz między całką górną, która jest zdefiniowana dla wszystkich
% funkcji nieujemnych i~zawsze ma sens, a~całką w~sensie właściwym
% która wymaga by~$\IntCaD{ | f( x ) | }{ x }$ była skończona
% \cite{Sch79}\footnote{Schwartz rozwijał teorię całki w~innym
% kontekście niż~rozważany tutaj np. całkował od razu funkcji
% o~wartościach w~przestrzeni Banacha. To co tu zostało napisane, jest
% więc pewną wariacją podanej przez niego teorii.}.

\vspace{\spaceThree}


\start \Str{17} W~twierdzeniu o~zbieżności monotonicznej jest
powiedziane, że~dla funkcji rozpatrywanych w~tym twierdzeniu zachodzi
$\int | f( p ) - f_{ n }( p ) | \dk p \ra 0$, co~wymaga pewnego
komentarza. Ponieważ $f( p ) \geq f_{ n }( p )$ stąd
$f( p ) - f_{ n }( p ) \geq 0$ więc zachodzi ciąg równoważności
\begin{equation*}
  \IntCaD{ | f( p ) - f_{ n }( p ) | }{ p } = \IntCaD{ ( f( p )
    - f_{ n }( p ) ) }{ p } = \IntCaD{ f( p ) }{ p }
  - \IntCaD{ f_{ n }( p ) }{ p }
\end{equation*}
Dowód, że~przy podanych w~tym twierdzeniu założeniach zachodzi
$f \in \LIj$ i~$\IntCaD{ f_{ n } }{ p } \ra \IntCaD{ f }{ p }$ można
znaleźć w~każdej książce do~teorii całki,
np.~\cite{Rudin98}. % Sprawdzone.

\vspace{\spaceThree}


\start \Str{17} W~twierdzeniu o~zbieżności majoryzowanej należy
założyć, że~funkcje $f_{ n }$~są mierzalne. % Sprawdzone.

\vspace{\spaceThree}


\start \Str{21} W~podanej charakterystyce liczbowej zbioru Cantora
jest pewna nieścisłość. Do tego zbioru należy liczba
$\fr{ 1 }{ 3 } = 0.1_{ 3 } = 0.02222\ld_{ 3 }$, nie jest więc jasne,
jak należy zastosować tu warunek, że~liczba należy do~zbioru Cantora,
jeśli nie zawiera w~swym rozwinięciu trójkowym cyfry~1. Właściwy
warunek\footnote{Wskazał mi go Paweł Duch.} jest następujący:
rozwinięcie trójkowe tej liczby można zapisać w~taki sposób by~nie
zawierało cyfry~1. % Sprawdzone.

\vspace{\spaceThree}


\start \Str{22} Nie rozumiem dlaczego z~faktu, że~miara każdego zbioru
zwartego jest skończona, wynika iż~jest tylko przeliczalnie wiele
punktów o~niezerowej mierze. \Dok

\vspace{\spaceTwo}


\start \Str{25} Należy zrobić dodatkowy komentarz na~temat równości
z~twierdzenia Radona\dywiz Nikodyma
\begin{equation}
  \label{eq:2}
  \nu( A ) = \IntSet{ A }{ f }{ \mu }
\end{equation}
\Dok Czy funkcja ta nie jest tylko lokalnie całkowalna? Ponieważ obie
miary w~tym twierdzeniu~są dodatnie, można więc zagwarantować,
że~$f \geq 0$. Całka z~tezy ma więc zawsze sens, jednak wynik może być
nieskończony. Nie jest to jednak problem \Dok

\vspace{\spaceThree}


\start \Str{25} Jakoś nie mogę siebie przekonać, że~funkcja mierzalna
względem dwóch zmiennych, jest też mierzalna względem każdej
ze~zmiennych z~osobna. Dla miar Radona, dla~których zbiory
borelowskie~są mierzalne, tak zapewne jest, jednak dla ogólnych
$\si$\dywiz algebr wydaje mi~się to dziwnie mocny twierdzeniem.
Zapewne funkcja jest prawie wszędzie mierzalna, ale nic silniejszego.
\Dok

\vspace{\spaceThree}


\start \Str{30} \tb{Dowód twierdzenia I.27} Zauważmy, że~na mocy
dowodu twierdzenie I.25, jeśli
$\absj{ f_{ n }( x ) - f_{ n }( y ) } < \veps$ zachodzi dla
każdego~$n$, to również $\absj{ f( x ) - f( y ) }$.

\vspace{\spaceThree}


\start \Str{30} W~twierdzeniach I.27 i~I.28, należałby dodać, że~dowód
dotyczy funkcji o~wartościach zespolonych. Jednak twierdzenie I.27
łatwo uogólnić na przypadek funkcji o~wartościach w~dowolnej
przestrzeni metrycznej, a~I.28 o~wartościach w~przestrzeni Banacha.

\vspace{\spaceThree}


\start \Str{37} Można podać inne sformułowanie twierdzenia Pitagorasa,
które w~pewnych sytuacjach jest bardziej naturalne. Jeśli $x_{ i }$,
$i = 1, \ld, N$ tworzą układ ortogonalny i~$x = \Sum x_{ i }$, to
\begin{equation*}
  \norm{ x }^{ 2 } = \Sum_{ i = 1 }^{ N } \norm{ x_{ i } }^{ 2 }.
\end{equation*}
% Sprawdzone.

\vspace{\spaceThree}


\start \Str{39} Przy okazji definicji izomorfizmu dwóch przestrzeni
Hilberta, warto zauważyć, że~operator unitarny jest bijekcją.
Surjektyweność zachodzi na mocy definicji, żeby zaś wykazać
inijektywność wystarczy skorzystać z~jednego z~podstawowych twierdzeń
algebry: odwzorowanie liniowe jest injektywne \wtw, gdy~jego jądro
jest trywialne. Jeżeli więc $Ux = 0$, wtedy:
\begin{equation*}
  0 = \SP{ Ux }{ Ux }_{ \Hc_{ 2 } } = \SP{ x }{ x }_{ \Hc_{ 1 } } \iff x = 0.
\end{equation*}
% Sprawdzone.

\vspace{\spaceThree}


\start \Str{40} W~przykładzie~6 przestrzeń Hilberta~$\Hc'$ musi być
ośrodkowa, ale~pojęcie przestrzeni ośrodkowej jest wprowadzone dopiero
na~stronie~47. % Sprawdzone.

\vspace{\spaceThree}


\start \Str{44} W~tym miejscu warto dowieść twierdzenia następującego
twierdzenia.
% ##############################
\begin{twr}
  W~przestrzeni prehilbertowskiej układ ortogonalny niezerowych
  wektorów~$S$ jest zupełny \wtw, gdy~jedynym wektorem ortogonalnym
  do~niego jest wektor~$0$.
\end{twr}

% Poniższy fragment jest niezrozumiały
Przez kontrapozycję $1) \Ra 2)$. Załóżmy, że~istnieje wektor
$g \neq 0$ ortogonalny do~$S$. Wówczas $S' = S \cup \set{ g }$ jest
układem ortogonalnym, który zawiera w~sobie $S$ i~$S' \neq S$. Czyli
$S$ nie jest układem zupełny.

% ######################################################################
% Tu skończyłem
Przez kontrapozycję $2) \Ra 1)$. Załóżmy teraz, że~$S$ nie jest
zupełny. W~takim razie istnieje taki układ ortogonalny niezerowych
wektorów, iż~$S \sset S'$ oraz $S' \setm S \neq \es$. Teraz
$y \in S' \setm S$ jest niezerowym wektorem ortogonalnym do~$S$.
% koniec niezrozumiałego fragmentu

\vspace{\spaceThree}


\start \Str{45} Nie rozumiem dlaczego z~nierówności
$\sum_{ i = 1 }^{ N } \absj{ \SP{ x_{ \al_{ i } } }{ y } } \leq \norm{
  y }^{ 2 }$ wynika, że~tylko dla przeliczalnej ilości indeksów $\al$
zachodzi $\SP{ x_{ \al } }{ y } \neq 0$. \Dok

\vspace{\spaceThree}


\start \Str{45} \tb{Dowód twierdzenia II.6.} Fakt, że~granica
rozważanego szeregu nie zależy od~kolejności wyrazów, można udowodnić
w~następujący sposób. Niech $A'$ będzie przeliczalnym zbiorem indeksów
dla których $( x_{ \al }, y ) \neq 0$ i~niech $\al_{ i } \in A'$
będzie jakimś jego uporządkowaniem. Wówczas przeprowadzone w~dowodzie
rozumowanie pokazuje, że
\begin{equation}
  y = \Sum_{ j = 1 }^{ \infty } \SP{ x_{ \al_{ i } } }{ y } x_{ \al_{ i } }.
\end{equation}
Jeżeli teraz weźmiemy jakieś inne jego uporządkowanie $x_{ j }$,
to~możemy powtórzyć całe rozumowanie i~otrzymać, że~suma nowego
szeregu znów wynosi $y$, tym samym jest niezależna od~kolejności
sumowania.

Pomimo tego, że~jego suma nie zależy od porządku wyrazów, nie oznacza
to, że~szereg ten musi być absolutnie zbieżny. Rozpatrzmy bowiem
element przestrzeni $\lcd$
\begin{equation*}
  y = ( 1, \tfr{ 1 }{ 2 }, \tfr{ 1 }{ 3 },\ld ), \quad
  \norm{ y }^{ 2 } = \Sum_{ i = 1 }^{ \infty } \fr{ 1 }{ n^{ 2 } }
  = \fr{ \pi^{ 2 } }{ 6 } < +\infty.
\end{equation*}
Jeśli~bazę ortonormalną przyjmiemy:
\begin{equation*}
  e^{ k } = ( 0, \ld, 0, \overbrace{ 1 }^{ k }, 0, \ld, 0),
\end{equation*}
to samym od~razu otrzymujemy
\begin{equation*}
  y = \Sum_{ k = 1 }^{ \infty } \tfr{ 1 }{ k } e^{ k }, \quad
  \Sum_{ k = 1 }^{ \infty } \norm{ \tfr{ 1 }{ k } e^{ k } }
  = \Sum_{ k = 1 }^{ \infty } \tfr{ 1 }{ k } = +\infty.
\end{equation*}

Paweł Duch znalazł dobre wyjaśnienie dlaczego suma tego konkretnego
typu szeregów nie zależy od~kolejności. Szereg zbieżny warunkowo jest
zbieżny bowiem jego wyrazy odpowiednio~się kompensują, wyrazy ujemne
kasują poprzedzające jej wyrazy dodatnie. Dlatego też jeśli zmienimy
ich kolejność to~mogą one~się nie skompensowań tak jak poprzednio
i~otrzymamy inną sumę.

Jednak tutaj każdy wyraz jest w~kierunku prostopadłym, do~wszystkich
innych, więc żadne dwa z~nich nie mogą~się wzajemnie skompensować.
Jeżeli więc taki szereg jest zbieżny, to zmiana kolejności nie wpływa
na sumę.

Schwartz bada w~swoje książce szeregi w~przestrzeniach Banacha, które
nie~są absolutnie zbieżne, ale~ich suma nie zależy od~kolejności
sumowania (str.~110 i~dalsze \cite{Schwartz79}). Trajdos przetłumaczył
ich nazwę jako szeregi \emph{zbieżne
  przestawialnie}. % Do przejrzenia.

\vspace{\spaceThree}


\start \Str{45} Wzorując~się na Grabowskim i~Ingardenie
\cite{GrabowskiIngarden87}, warto w~kontekście twierdzenia II.6
udowodnić jeszcze jedną piękną równość:
\begin{equation*}
  \SP{ y }{ y' } = \Sum_{ \al \in A } \SP{ y }{ x_{ \al } }
  \SP{ x_{ \al } }{ y' },
\end{equation*}
gdzie $S = \{ x_{ \al } \}_{ \al \in A }$ jest bazą ortonormalną,
niekoniecznie przeliczalną. Istniej przeliczalny zbiór
$x_{ \al_{ j } }$ taki,
że~$y' = \Sum_{ j = 1 }^{ \infty } \SP{ x_{ \al_{ j } } }{ y' } x_{
  \al_{ j } }$. Wtedy
\begin{equation*}
  \begin{split}
    \SP{ y }{ y' } &= \SP{ y }{ \Lim_{ n \ra \infty } \Sum_{ j = i }^{
        n } \SP{ x_{ \al_{ j } } }{ y' } x_{ \al_{ j } } } = \Lim_{ n
      \ra \infty } \SP{ y }{ \Sum_{ j = 1 }^{ n } \SP{ x_{ \al_{ j } }
      }
      { y' } x_{ \al_{ j } } } \\
    &= \Lim_{ n \ra \infty } \Sum_{ j = 1 }^{ n } \SP{ y }{ \SP{ x_{
          \al_{ j } } }{ y' } x_{ \al_{ j } } }
    % = \sum_{ j = 1 }^{ \infty }
    % ( y, x_{ \alpha } ) ( x_{ \alpha }, y' )
    = \Sum_{ \al \in A } \SP{ y }{ x_{ \al } } \SP{ x_{ \al } }{ y' }.
  \end{split}
\end{equation*}
Zastąpienie w~ostatniej równości sumy po zbiorze
$\alpha_{ j }, j \in \N$ sumą po~$A$ jest możliwe na mocy tego,
że~jeżeli $\alpha \neq \alpha_{ j }$ dla każdego $j$, to wyrażenie pod
sumą jest równe zeru. Zauważmy też, że~skoro zbieżność szeregu
$\sum_{ \al \in A } \SP{ x_{ \al } }{ y } x_{ \al }$ nie zależy
od~kolejności, to~na~mocy przeprowadzone rozumowania również suma
szeregu
$\sum_{ \al \in A } \SP{ y }{ x_{ \al } } \SP{ x_{ \al } }{ y' }$, nie
zależy od~kolejności wyrazów.

Ta tożsamość pozwala nam wykazać, w~sposób bardziej elegancki niż
zrobili to Reed i~Simon, równanie (II.2), mianowicie podstawiając
$y' = y$, mamy
\begin{equation*}
  \norm{ y }^{ 2 } = \SP{ y }{ y } = \Sum_{ \al \in A } \SP{ y }{ x_{ \al } }
  \SP{ x_{ \al } }{ y } = \Sum_{ \al \in A } \absj{ \SP{ x_{ \al } }{ y } }^{ 2 }.
\end{equation*}

\vspace{\spaceThree}


\start \Str{45} Dysponując twierdzeniem II.6, można teraz podać ogólną
postać nierówności Bessela.

\begin{twr}[Ogólna postać nierówności Bessela]
  Niech $S = \{ x_{ \al } \}_{ \al \in A }$ będzie układem
  ortonormalny i~niech $B \sset A$. Wówczas:
  \begin{equation*}
    \Sum_{ \be \in B } \absd{ \SP{ x_{ \be }}{ x } }^{ 2 } \leq \norm{ x }^{ 2 }.
  \end{equation*}
\end{twr}
\begin{proof}
  Prawie nie ma czego dowodzić, mamy bowiem
  \begin{equation*}
    \norm{ x }^{ 2 } = \Sum_{ \al \in A } \absd{ \SP{ x_{ \al } }{ x } }^{ 2 }
    \geq \Sum_{ \be \in B } \absj{ \SP{ x_{ \be }}{ x } }^{ 2 }.
  \end{equation*}
\end{proof}

\vspace{\spaceThree}


\start \Str{49} Konstrukcja iloczynu tensorowego jest przeprowadzona
bardzo szkicowo, przez co~sens tego co~robimy jest często ukryty,
tutaj postaram~się uzupełnić brakujące kroki.

Jednym z~przyczyn zamieszania jest używanie tego samego symbolu na
iloczyn skalarny w~$\Hc_{ 1 }$, $\Hc_{ 2 }$ i~$\mc{E}$, tu będę je
oznaczał odpowiednio jako $( \cdot, \cdot )_{ \Hc_{ 1 } }$,
$( \cdot, \cdot )_{ \Hc_{ 2 } }$, $( \cdot, \cdot )_{ \mc{ E } }$.
Działanie formy $\vp_{ 1 } \ot \vp_{ 2 }$ można więc zapisać jako
\begin{equation*}
  ( \vp_{ 1 } \ot \vp_{ 2 } )\dket{ \psi_{ 1 }}{ \psi_{ 2 } }
  = \SP{ \psi_{ 1 } }{ \vp_{ 1 } }_{ \Hc_{ 1 } } \SP{ \psi_{ 2 } }
  { \vp_{ 2 } }_{ \Hc_{ 2 } }.
\end{equation*}
Iloczyn skalarny na~$\mc{E}$ ma więc postać:
\begin{equation*}
  \SP{ \vp \ot \psi }{ \eta \ot \mu }_{ \mc{E} }
  = \SP{ \vp }{ \eta }_{ \Hc_{ 1 } } \SP{ \psi }{ \mu }_{ \Hc_{ 2 } }.
\end{equation*}
Formę postaci $\vp \ot \psi$ będziemy nazywać prostą.

W~dalszym ciągu potrzebna będzie nam następująca obserwacja: jeżeli
$\mu \in \mc{E}$, to zachodzi równość:
\begin{equation*}
  \SP{ \vp \ot \psi }{ \mu }_{ \mc{E} } = \mu\dket{ \vp }{ \psi }.
\end{equation*}
Jeżeli $\mu = \eta \ot \chi$ to powyższa równość wynika z~definicji
wszystkich działań. Jeśli zaś
$\mu = \Sum_{ i = 1 }^{ N } c_{ i } ( \eta_{ i } \ot \psi_{ i } )$,
wówczas
\begin{equation*}
  \begin{split}
    \SP{ \vp \ot \psi }{ \mu }_{ \mc{E} } &= \left( \vp \ot \psi, \,
      \Sum_{ i = 1 }^{ N } c_{ i } ( \eta_{ i } \ot \chi_{ i } )
    \right)_{ \mc{E} } = \Sum_{ i = 1 }^{ N } c_{ i } \SP{ \vp \ot
      \psi }
    { \eta_{ i } \ot \chi_{ i } }_{ \mc{E} } \\
    &= \Sum_{ i = 1 }^{ N } c_{ i } ( \eta_{ i } \ot \chi_{ i } )
    \dket{ \vp }{ \psi } = \mu\dket{ \vp }{ \psi }.
  \end{split}
\end{equation*}
% Dla zupełności przedstawimy uogólnienie tej obserwacji. Niech
% $\eta = \Sum_{ i = 1 }^{ N } d_{ i } ( \vp_{ i } \ot \psi_{ i } )$,
% korzystając z~półtoraliniowości mamy
% \begin{equation*}
%   \begin{split}
%     ( \eta , \mu )_{ \mc{E} } &= \Sum_{ i = 1 }^{ N } \bar{ d }_{ i
%   } ( ( \vp_{ i } \ot \psi_{ i } ), \mu )_{ \mc{E} } = \Sum_{ i = 1
%   }^{ N } \bar{ d }_{ i } ( \vp \ot \psi,
%     \eta_{ i } \ot \chi_{ i } )_{ \mc{E} } \\
%     &= \Sum_{ i = 1 }^{ N } c_{ i } ( \eta_{ i } \ot \chi_{ i } )
%     \lket \vp, \psi \rket = \mu\lket \vp, \psi \rket.
%   \end{split}
% \end{equation*}

% wtym fragmencie wielokrotnie powtarza się "zauważyć" <<<<<<<<<<<<
Teraz możemy przejść do~dyskusje, %do dyskusji<<<<<<<
czy~wzór na iloczyn skalarny nie zależy od~przedstawienia jako sumy
form prostych. Zauważmy, że~dwie formy $\mu$, $\eta$~są równe gdy
\begin{equation*}
  \mu\dket{ \vp }{ \psi } = \eta\dket{ \vp }{ \psi },
  \quad \forall \vp \in \Hc_{ 1 }, \psi \in \Hc_{ 2 }.
\end{equation*}
Jak łatwo jednak zauważy % zauważyć <<<<<<<<<<<<
, formy $\mu_{ 1 } = \vp \ot ( \psi_{ 1 } + \psi_{ 2 } )$
oraz~$\mu_{ 2 } = \vp \ot \psi_{ 1 } + \vp \ot \psi_{ 2 }$
przedstawiają tę samą formę, choć~są sumą innych form prostych. Z~tego
względu należałoby napisać, że~$\mc{E}$ jest zbiorem klas abstrakcji
względem relacji poddanej wyżej, niż~po~prostu zbiorem sum
odpowiednich form prostych, z~jakiegoś jednak względu Reed i~Simon
uznali, że~nie warto zagłębiać~się w~tę kwestię.

Zauważmy, że~powyższy przykład uogólnia~się do~dobrze znanego
z~algebry liniowej faktu, iż~iloczyn tensorowy jest wieloliniowy.
Z~samej definicji wynika, że~zachodzi następująca równość form
\begin{equation*}
  \vp \ot \ld \ot ( \al \psi_{ 1 } + \beta \psi_{ 2 } ) \ot \ld \ot \eta
  = \al \, \vp \ot \ld \ot \psi_{ 1 } \ot \ld \ot \eta
  + \beta \, \vp \ot \ld \ot \psi_{ 2 } \ot \ld \ot \eta.
\end{equation*}

Musimy teraz pokazać, że~iloczyn skalarnych dwóch form z~$\mc{E}$,
nie~zależy od ich przedstawienia w~postaci sumy form prostych.
Zauważmy, że~dwie różne postaci tej samej formy różnią się o~formę
zerową. Jeżeli weźmiemy dwie formy określone wyżej $\mu_{ 1 }$
i~$\mu_{ 2 }$, to ich różnica:
\begin{equation*}
  \mu_{ 1 } - \mu_{ 2 } = \vp \ot ( \psi_{ 1 } + \psi_{ 2 } )
  - \vp \ot \psi_{ 1 } - \vp \ot \psi_{ 2 }
\end{equation*}
jest formą zerową. Zwróćmy uwagę, że~rozkład formy zerowej na formy
proste również nie jest jednoznaczny, np.~każda forma postaci
$\vp \ot 0$ jest formą zerową. Jeżeli teraz uda~się pokazać, że~forma
zerowa niezależnie od~swojej postaci, jest ortogonalna do~każdej formy
w~$\mc{E}$, to~można będzie pokazać, iż~tak rzeczywiście jest.

To~co pokazano powyżej, wraz z~tym co~napisali Reed i~Simon jest
wystarczające by~wykazać żądaną ortogonalność formy zerowej. Teraz
wykażemy, że~rozważany iloczyn skalarny jest niezależny od użytego
rozkładu. Niech $\psi \sim \psi'$ i~$\vp' \sim \vp$, a~$\mu$ będzie
formą zerową. W~takiej sytuacji $\psi' = \psi + \mu$
i~$\vp' = \vp + \mu$, przy czym konkretna postać $\mu$ może być w~obu
przypadkach różna. Teraz możemy, na~poziomie konkretnych
reprezentantów, czyli sum form prostych, przeprowadzić następujący
rachunek, kończący dowód
\begin{equation*}
  \SP{ \vp' }{ \psi' } = \SP{ \vp + \mu }{ \psi' }
  = \SP{ \vp }{ \psi' } + \SP{ \mu }{ \psi' } = \SP{ \vp }{ \psi' }
  = \SP{ \vp }{ \psi + \mu } = \SP{ \vp }{ \psi }.
\end{equation*}

\vspace{\spaceThree}


\start \Str{50} Jako konkluzję konstrukcji iloczynu tensorowego
przestrzeni Hilberta warto podać następującą piękną równość, która
wynika wprost z~wprowadzonych definicji.
\begin{equation*}
  \norm{ \vp \ot \psi }_{ \Hc_{ 1 } \ot \Hc_{ 2 } }
  = \norm{ \vp }_{ \Hc_{ 1 } } \norm{ \psi }_{ \Hc_{ 2 } }.
\end{equation*}
Za jej pomocą można łatwo pokazać, że~iloczyn tensorowy jest ciągły,
dowód jest taki sam jak w~przypadku iloczynu skalarnego.

\vspace{\spaceThree}


\start \Str{50} Również tu~część dowodu, że~układ
$\vp_{ l } \ot \psi_{ k }$ stanowi bazę ortonormalną, jest bardzo
szkicowa\footnote{Pewne idee tu przedstawione pochodzą od~Pawła
  Ducha.}. Aby~ją uzupełnić zauważmy najpierw, że~jeśli rozważane
przestrzeń są metryczne, to~odwzorowanie jest topologiczne
ciągła % ciągłe?<<<<<<
\wtw%jw<<<<
gdy jest ciągowo ciągła\footnote{Nie wiem czy tak jest w~ogólności,
  może bowiem powstać problem gdy nie istnieje przeliczalna baza
  otoczeń.} (tzn.
$\lim_{ n \ra +\infty } f( x_{ n } ) = f( \lim_{ n \ra +\infty } x_{ n
} )$). Ponieważ możemy udowodnić, że~również z~wieleindeksowymi
ciągami liczb naturalnych\footnote{Zapewne nie tylko, ale~większa
  ogólność nie jest nam tu potrzeba.} można wejść do~funkcji
topologiczne ciągłej, aby~uzasadnić tą operację w~przypadku
przestrzeni metrycznych, wystarczy pokazać, że~funkcja jest ciągowo
ciągła. Nie wiem jak jest w~ogólniejszych sytuacjach, lecz na razie ta
wiedza nie jest potrzebna.

Ponieważ norma jest funkcją ciągłą, zachodzi więc równość
\begin{equation*}
  \Lim_{ \substack{ K \ra \infty \\ L \ra \infty } } || \vp \ot \psi
  - \Sum_{ k < K }\Sum_{ l < L } \vp_{ k } \ot \psi_{ l } ||
  = || \vp \ot \psi - \Lim_{ \substack{ K \ra \infty \\ L \ra \infty } } \Sum_{ k < K }
  \Sum_{ l < L } \vp_{ k } \ot \psi_{ l } ||.
\end{equation*}
Aby zakończyć dowód o~bazie ortonormalnej trzeba teraz wykonać
,,bezpośredni rachunek'' i~pokazać, że~pierwsza z~tych granic jest
równa~0. $\vp = \sum_{ l } \vp_{ l }$ możemy więc dobrać $L$ tak~by
$\vp = \sum_{ l < L } \vp_{ l } + \vp_{ R }$, gdzie
$\norm{ \vp_{ R } } < \veps$, analogicznie
$\psi = \sum_{ k < K } \psi_{ k } + \psi_{ R }$,
$\norm{ \psi_{ R } } < \veps$.

Potrzebujemy jeszcze zauważyć, że~z~faktu, iż~suma
$\sum_{ l < L } \vp_{ l }$ jest zbieżna do~$\vp$, wynika,
że~$\norm{ \sum_{ l < L } \vp_{ l } } \ra \norm{ \psi }$, tworzy więc
ona ciąg Cauchy'ego i~tym samym jest ograniczony.
Dobierz % znów proponuje zmianę na >>Dobierzmy<<<
$M$ tak~by było wspólnym ograniczeniem ciągów
$\sum_{ l < L } \vp_{ l }$ i~$\sum_{ k < K } \psi_{ k }$. Tym samym
\begin{equation*}
  \begin{split}
    & || \vp \ot \psi - \Sum_{ k < K }\Sum_{ l < L } \vp_{ k } \ot
    \psi_{ l } || = || ( \Sum_{ l < L } \vp_{ l } + \vp_{ R } ) \ot (
    \Sum_{ k < K } \psi_{ k } + \psi_{ R } )
    - \Sum_{ k < K }\Sum_{ l < L } \vp_{ k } \ot \psi_{ l } || \\
    &= || ( \Sum_{ l < L } \vp_{ l } ) \ot \psi_{ R } + \vp_{ R } \ot
    ( \Sum_{ k < K } \psi_{ k } ) + \vp_{ R } \ot \psi_{ R } || \leq
    || ( \Sum_{ l < L } \vp_{ l } ) \ot \psi_{ R } ||
    + || \vp_{ R } \ot ( \Sum_{ k < K } \psi_{ k } ) || \\
    &+ || \vp_{ R } \ot \psi_{ R } || \leq 2 M \veps + \veps^{ 2 }.
  \end{split}
\end{equation*}
Odpowiednio dobierając $L$ i~$K$ możemy uczynić tą różnicę dowolnie
małą co kończy dowód.

\vspace{\spaceThree}


\start \Str{51} W~tym miejscu będziemy potrzebowali następującego
twierdzenie, którego dowód znalazł Paweł Duch.

\begin{twr}
  \label{twr:miaraA}
  Jeżeli $( X, \mu )$ i~$( Y, \nu )$ są przestrzeniami z~miarą
  $\si$\dywiz skończoną i~$A \sset X \ti Y$ będzie zbiorem mierzalny
  względem miary $\mu \ot \nu$. Jeśli dla $\mu$\dywiz prawie każdego
  $x \in X$, miara $\nu$ zbioru\footnote{Standardowo w~naturalny
    sposób utożsamiany ten zbiór z~podzbiorem $Y$.}
  $( x \ti Y ) \cap A$ jest równa 0, wówczas $(\mu \ot \nu)( A ) = 0$.
\end{twr}
\begin{proof}
  W~przypadku miary dodatniej można tak sformułować teorię całki,
  że~dla każdego zbioru zachodzi\footnote{Zobacz uwagi o~całce górnej
    w~ujęciu Schwartza.}
  \begin{equation*}
    \mu( S ) = \int_{ S } \chi_{ S }( x ) \, d\mu( x ),
  \end{equation*}
  niezależnie czy jego miara jest skończona, czy nie. Skoro zaś obie
  miary~są $\si$\dywiz skończone, to zachodzi twierdzenie o~zamianie
  całki podwójnej na iterowaną, znów niezależnie od~tego czy
  poszczególne całki są skończone czy nie.

  Dostajemy więc
  \begin{equation*}
    (\mu \ot \nu)( A ) = \int_{ X \ti Y } \chi( x, y )
    \, d\mu( x ) \ot d\nu( y )
    = \int_{ X } \left( \int_{ Y } \chi( x, y ) \, d\nu( y ) \right)
    \, d\mu( x ) = 0,
  \end{equation*}
  bowiem na mocy założeń
  $\mu( ( x \ti Y ) \cap A ) = \int_{ Y } \chi( x, y ) \, d\nu( y )$
  jest równa zeru dla $\mu$\dywiz prawie wszystkich~$x$.
\end{proof}
Zauważmy, że~to twierdzenie ma prostą interpretację na płaszczyźnie
$\R^{ 2 }$. Wówczas zbiór $x \ti Y$ jest prostą pionową przechodzącą
przez punkt $x$ i~powyższe twierdzenie mówi, że~jeśli prawie każdy
przekrój zbioru $A$ prostymi pionowymi ma miarę (długość)~0, to pole
powierzchni $A$ również jest zerowe.

\vspace{\spaceThree}


\start \Str{51}

\vspace{\spaceThree}


\start \StrWg{106}{5} Nie rozumiem dlaczego w~tym miejscu dodatkowo
zaznacza~się, że~miarę dowolnego zbioru mierzalnego, można
aproksymować od~wewnątrz przez zbiory zwarte \emph{i~borelowskie}.
Skoro rozważamy tylko zwarte przestrzenie Hausdorffa, wszystkie zbiory
zwarte~są domknięte i~tym samym borelowskie.

\newpage



\CenterTB{Błędy}
\begin{center}
  \begin{tabular}{|c|c|c|c|c|}
    \hline
    & \multicolumn{2}{c|}{} & & \\
    Strona & \multicolumn{2}{c|}{Wiersz}& Jest & Powinno być \\ \cline{2-3}
    & Od góry & Od dołu &  &  \\ \hline
    2 & 8 & & surjective & \tb{surjective} \\
    2 & 9 & & \tb{surjective} & surjective \\
    14 & 10 & & $\Si_{ 2^{ n } }( f )$ & $\Si_{ 2 n }( f )$ \\
    14 & 14 & & $\Si_{ 2^{ n } }( f )$ & $\Si_{ 2 n }( f )$ \\
    17 & 3 & & $\int \;\: f_{ n } \;\: dp$ & $\IntCaD{ f_{ n } }{ p }$ \\
    40 & & 3 & a~Hilbert & a~separable Hilbert \\
    44 & & 7 & $\fr{ T( y ) }{ T( x_{ 0 } ) }$
           & $\fr{ \SP{ x_{ 0 } }{ y } }{ \SP{ x_{ 0 } }{ x_{ 0 } } }$ \\
    45 & & 2 & some & all \\
    46 & & 9 & $\SP{ v_{ 1 } }{ u_{ 1 } } v_{ k }$
           & $\SP{ v_{ 1 } }{ u_{ 2 } } v_{ 1 }$ \\
    49 & & 7 & linear & bilinear \\
    53 & & 12 & $\vp_{ k_{ \si( 1 ) } } \ot \vp_{ k_{ \si( 2 ) } } \cdots
                \ot \vp_{ k_{ n( p ) } }$
           & $\vp_{ k_{ \si( 1 ) } } \ot \vp_{ k_{ \si( 2 ) } } \ot \cdots
             \ot \vp_{ k_{ \si( p ) } }$ \\
    70 & 8 & & $\Lim_{ n \ra +\infty }$ & $\Lim_{ m \ra +\infty }$ \\ % Sprawdź czy
    % dobrze.
    73 & & 2 & $\Sum_{ k = 1 }^{ \infty } \absd{ \la_{ k } } < \infty$
           & $\Sum_{ k = 1 }^{ \infty } \absj{ \la_{ k } } \leq
             \norm{ \la_{ 0 } }_{ c_{ 0 }^{ * } } < \infty$ \\
    79 & 6 & & $\norm{ [x] }$ & $\norm{ [x] }_{ 1 }$ \\
    82 & & 15 & since$T[ B_{ r }^{ X } ]$ & since $T[ B_{ r }^{ X } ]$ \\
    104 & 5 & & $[ 0, 1 ]$ & $[ -1, 1 ]$ \\
    104 & 6 & & $||P_{ n }( f )$ & $|| \, P_{ n }( f )$ \\
    106 & 11 & & $\mu( \, Y \setm \wt{ Y } )$
           & $\mu( Y \setm \wt{ Y } )$ \\
    106 & 5 & & $C \sset O$ & $C \sset Y$ \\
    107 & 4 & & $\Real[ e^{ -i \vp } f ]$ & $e^{ -i \vp } f$ \\
    107 & 4 & & $\Real( e^{ -i \vp } f )$ & $e^{ -i \vp } f$ \\
    107 & 10 & & $\int\! fd\mu$ & $\IntCaD{ f }{ \mu }$ \\
    107 & & 10 & functions & functionals \\
    109 & 4 & & $\norm{ l }\!\sup$ & $\norm{ l } \sup$ \\
    112 & & 8 & $X$ & $F$ \\
    128 & 2 & & $| \rho_{ \al_{ 1 } }( x ) |$ & $\rho_{ \al_{ 1 } }( x )$ \\
    128 & 2 & & $| \rho_{ \al_{ n } }( x ) |$ & $\rho_{ \al_{ n } }( x )$ \\
    % & & & & \\
    % & & & & \\
    % & & & & \\
    % & & & & \\
    & & & & \\ \hline
  \end{tabular}
\end{center}
\noi
\tb{Grzbiet} \\
\Jest REVSED \\
\Pow REVISED \\
\StrWg{43}{3} \\
\Jest
\begin{equation*}
  \sup_{ \substack{ \norm{ x }_{ \Hc } = 1 } } \;\;\; \norm{ T x }_{ \Hc' }
\end{equation*}
\Pow
\begin{equation*}
  \sup_{ \substack{ \norm{ x }_{ \Hc } = 1 } } \norm{ T x }_{ \Hc' }
\end{equation*}
\StrWd{51}{13} \\
\Jest
\begin{equation*}
  \int_{ M_{ 2 } } \bigg( \int_{ M_{ 1 } } \ol{ f( x, y ) } \vp_{ k }( x ) \,
  d\mu_{ 1 }( x ) \bigg) \quad \psi_{ l }( y ) \, d\mu_{ 2 }( y )
\end{equation*}
\Pow
\begin{equation*}
  \int_{ M_{ 2 } } \bigg( \int_{ M_{ 1 } } \ol{ f( x, y ) } \vp_{ k }( x ) \,
  d\mu_{ 1 }( x ) \bigg) \psi_{ l }( y ) \, d\mu_{ 2 }( y )
\end{equation*}



% #####################################################################
% #####################################################################
\bibliographystyle{alpha} \bibliography{Bibliography}{}



\end{document}
