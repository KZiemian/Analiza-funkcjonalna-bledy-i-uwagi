\RequirePackage[l2tabu, orthodox]{nag}
% Autor: Kamil Ziemian
% Korekta: Wojciech Dyba
\documentclass[a4paper,11pt]{article}
\usepackage[utf8]{inputenc} % Pozwala pisać polskie znaki bezpośrednio.
\usepackage[polish]{babel} % Tłumaczy na polski teksty automatyczne LaTeXa
% i pomaga z typografią.
\usepackage[MeX]{polski} % Polska notacja we wzorach matematycznych.
% Ładne polskie
\usepackage{microtype}
\let\lll\undefined
\usepackage[intlimits]{amsmath}
\usepackage{amsfonts, amssymb, amscd, amsthm}
\usepackage{latexsym} % Więcej symboli.
\usepackage{upgreek} %Lepsze greckie czcionki. Przyklad skladni: pi = \uppi
\usepackage{textcomp} % Pakiet z dziwnymi symbolami.
% \usepackage{slashed} % Pozwala pisać slash Feynmana.
\usepackage{tensor} % Pozwala prosto używać notacji tensorowej.
% Albo nawet pięknej notacji tensorowej:).
\usepackage{graphicx} % Pozwala wstawiać grafikę.
\usepackage{xcolor}
\usepackage{calrsfs} % Lepsze kaligrafowane litery.
% \DeclareMathAlphabet{\pazocal}{OMS}{zplm}{m}{n}
\usepackage{vmargin}
%---------------------------------------------------------------------
% MARGINS
%---------------------------------------------------------------------
\setmarginsrb
{ 0.7in} % left margin
{ 0.6in} % top margin
{ 0.7in} % right margin
{ 0.8in} % bottom margin
{  20pt} % head height
{0.25in} % head sep
{   9pt} % foot height
{ 0.3in} % foot sep
\usepackage{hyperref}
% #####################################################################


\newtheorem{twr}{Twierdzenie}
\newtheorem{defin}{Definicja}
\newtheorem{wni}{Wniosek}


% #####################################################################

\newcommand{\spaceOne}{2em}
\newcommand{\spaceTwo}{1em}
\newcommand{\spaceThree}{0.5em}

% Ładniejszy zbiór pusty
\let\oldemptyset\emptyset
\let\emptyset\varnothing


% ####################
% Uproszczenia w pisaniu tekstu
\newcommand{\ld}{\ldots}

% Slashe w tekście
\newcommand{\tbs}{\textbackslash}

% Podstawowe oznaczenia matematyczne
\newcommand{\fr}{\frac}
\newcommand{\tfr}{\tfrac}
\newcommand{\tr}{\textrm}

% Oznaczenia ,,nad i pod''. Wymyśl lepszą nazwę
\newcommand{\til}{\tilde}
\newcommand{\ul}{\underline}
\newcommand{\ol}{\overline}
\newcommand{\wh}{\widehat}
\newcommand{\wt}{\widetilde}

% Trzcionki matematyczne
\newcommand{\mr}{\mathrm}
\newcommand{\mb}{\mathbb}
\newcommand{\mc}{\mathcal}
\newcommand{\mf}{\mathfrak}
\newcommand{\mbf}{\mathbf}
\newcommand{\bsym}{\boldsymbol}

% Strzałki
\newcommand{\ra}{\rightarrow}
\newcommand{\Ra}{\Rightarrow}
\newcommand{\lra}{\longrightarrow}
\newcommand{\xra}{\xrightarrow}

\newcommand{\wtw}{wtedy i~tylko wtedy}


% ####################
% Litery greckie
\newcommand{\al}{\alpha}
\newcommand{\be}{\beta}
\newcommand{\ga}{\gamma}
\newcommand{\del}{\delta}
\newcommand{\Del}{\Delta}
\newcommand{\la}{\uplambda}
\newcommand{\La}{\Lambda}
\newcommand{\eps}{\epsilon}
\newcommand{\veps}{\varepsilon}
\newcommand{\vp}{\varphi}
\newcommand{\om}{\omega}
\newcommand{\Om}{\Omega}
\newcommand{\si}{\sigma}
\newcommand{\Si}{\Sigma}
\newcommand{\tet}{\theta}

% Standardowe oznaczenia literowe
\newcommand{\N}{\mb{N}}
\newcommand{\R}{\mb{R}}
\newcommand{\C}{\mb{C}}
\newcommand{\D}{\mc{D}}
\let\H\undefined
\newcommand{\H}{\mc{H}}
\let\L\undefined
\newcommand{\L}{\mc{L}}
\newcommand{\Rn}{\R^{ n }}
\newcommand{\Rc}{\mc{R}}
\newcommand{\Cc}{\mc{C}}

% Mniej używane oznaczenia literowe
\newcommand{\B}{\mc{B}}
\newcommand{\Oc}{\mc{O}}
\newcommand{\Rp}{\R_{ + }}

% ####################
% Standardowe matematyczne symbole literowe
% Mathbb
% \newcommand{\C}{\mb{C}}
% \newcommand{\N}{\mb{N}}
% \newcommand{\R}{\mb{R}}
% \newcommand{\Rn}{\R^{ n }}
% \newcommand{\Rp}{\R_{ + }}

% % Mathcal
% \newcommand{\B}{\mc{B}}
% \newcommand{\D}{\mc{D}}
% \newcommand{\Fc}{\mc{F}}
% \let\H\undefined
% \newcommand{\H}{\mc{H}}
% \let\L\undefined
% \newcommand{\L}{\mc{L}}
% \newcommand{\M}{\mc{M}}
% \newcommand{\Rc}{\mc{R}}
% \newcommand{\Cc}{\mc{C}}
% \newcommand{\Oc}{\mc{O}}
% \newcommand{\T}{\mc{T}}


% ####################
% Teoria mnogości
\newcommand{\es}{\emptyset}
\newcommand{\set}[1]{\{ #1 \}}
\newcommand{\subs}{\subset}
\newcommand{\setm}{\setminus}
\newcommand{\com}{\complement}
\newcommand{\ti}{\times}


% ####################
% Algebra
\newcommand{\Real}{\mf{Re}}
\newcommand{\Imag}{\mf{Im}}
\newcommand{\ot}{\otimes}
\newcommand{\Tr}{\mr{Tr}}


% ####################
% Analiza matematyczna

% Granice
\newcommand{\Lim}{\lim\limits}
\newcommand{\Liminf}{\ul{\lim}}
\newcommand{\Limsup}{\ol{\lim}}

% Sumy
\newcommand{\Sum}{\sum\limits}

% Różniczkowanie i pochodne
\newcommand{\pr}{\partial}
\newcommand{\de}{\mr{d}}
\newcommand{\dd}[3]{\fr{ \de^{ #1 } { #2 } }{ \de { #3 }^{ #1 } }}
\newcommand{\pd}[3]{\fr{ \pr^{ #1 } { #2 } }{ \pr { #3 }^{ #1 } }}

% Powszechnie używane symbole 
\newcommand{\Cinfty}{\Cc^{ \infty }}

% Całki i całkowanie
\newcommand{\dx}{\de x}
\newcommand{\Int}{\int\limits}
\newcommand{\IntL}[3]{ \Int_{ { #1 } }^{ { #2 } } \de#3 \; }
\newcommand{\IntCa}[2]{ \Int #1 \, \de#2 } % Juz z calkowaniem
\newcommand{\IntFi}[2]{ \Int \de#1 \, #2 } % Bardziej Fizyczna notacja;)
\newcommand{\IntWie}[3]{ \Int_{ #1 } \de^{ #2 }#3 \; } % Wielowymiarowa
\newcommand{\IntDo}[5]{ \Int_{ #1 } \de^{ #2 }#3\, \de^{ #4 } #5 \; }

% ####################
% Całki w określonych granicach
\newcommand{\IntA}[1]{\Int_{ -\infty }^{ +\infty } \de #1 \;}
\newcommand{\IntB}[1]{\int_{ \R } \de #1 \;}
\newcommand{\IntC}[2]{\int_{ \R } #1 \, \de #2}
\newcommand{\IntD}[1]{\Int_{ 0 }^{ +\infty } \de #1 \;}


% ####################
% Analiza funkcjonalna
\newcommand{\da}{\dagger}

\newcommand{\supp}{\mr{supp}\,}

% Brakety
% W analizie funkcjonalnej też często się je stosuje
\newcommand{\lket}{\langle}
\newcommand{\rket}{\rangle}

% Przestrzenie L^{ p }
\newcommand{\Lj}{L^{ 1 }}
\newcommand{\Ld}{L^{ 2 }}
\newcommand{\Lp}{L^{ p }}
\newcommand{\LIj}{\L^{ 1 }}
\newcommand{\LIp}{\L^{ p }}
\newcommand{\LdJ}{\Ld( \R, \de \mu )}
\newcommand{\LdT}{\Ld( \R^{ 3 }, \de \mu )}
\newcommand{\Ldlim}{L^{ 2 }-\Lim}


% ####################
% Wartość bezwzględna i normy
\providecommand{\absj}[1]{\lvert #1 \rvert}
\providecommand{\absd}[1]{\left| #1 \right|}
\newcommand{\norm}[1]{\left|\left| #1 \right|\right|}


% ####################
% Edycja tekstu
\newcommand{\tb}{\textbf}
\newcommand{\noi}{\noindent}
\newcommand{\start}{\noi \tb{--} {}}
\newcommand{\Str}[1]{\tb{Str. #1.}}
\newcommand{\StrWg}[2]{\tb{Str. #1, wiersz #2.}}
\newcommand{\StrWd}[2]{\tb{Str. #1, wiersz #2 (od dołu).}}
\newcommand{\Dow}{\tb{Dowód}}
\newcommand{\Center}[1]{\begin{center} #1 \end{center}}
\newcommand{\CenterTB}[1]{\Center{\tb{#1}}}
\newcommand{\Jest}{\tb{Jest: }}
\newcommand{\Pow}{\tb{Powinno być: }}
\newcommand{\red}[1]{{\color{red} #1}}
\newcommand{\Prze}{{\color{red} Przemyśl.}}
\newcommand{\Pop}{{\color{red} Popraw.}}
\newcommand{\Prob}{{\color{red} Problem.}}
\newcommand{\Dok}{{\color{red} Dokończ.}}
\newcommand{\Pyt}{{\color{red} Pytanie.}}
\newcommand{\Work}[1]{ \begin{center} {\large \tb{#1}} \end{center} }
\newcommand{\Field}[1]{ \begin{center} {\Large \tb{#1} } \end{center} }
\newcommand{\Main}[1]{ \begin{center} {\LARGE \tb{#1} } \end{center} }





% % Sumy
% \newcommand{\Sum}{\sum\limits}

% % Różniczkowanie i pochodne
% \newcommand{\pr}{\partial}
% \newcommand{\de}{\mr{d}}
% \newcommand{\dd}[3]{\fr{ \de^{ #1 } { #2 } }{ \de { #3 }^{ #1 } }}
% \newcommand{\pd}[3]{\fr{ \pr^{ #1 } { #2 } }{ \pr { #3 }^{ #1 } }}

% % Całki
% \newcommand{\Int}{\int\limits}
% \newcommand{\IntA}[1]{\Int_{ -\infty }^{ +\infty } \de #1 \;}
% \newcommand{\IntB}[1]{\int_{ \R } \de #1 \;}
% \newcommand{\IntC}[1]{\Int_{ 0 }^{ +\infty } \de #1 \;}
% \newcommand{\IntCa}[2]{ \Int #1 \, \de#2 } % Juz z calkowaniem
% \newcommand{\IntFi}[2]{ \Int \de#1 \, #2 } % Bardziej Fizyczna notacja;)
% \newcommand{\IntWie}[3]{ \int_{ #1 } \de^{ #2 }#3 \; } % Wielowymiarowa.
% \newcommand{\IntDo}[5]{ \Int_{ #1 } \de^{ #2 }#3\, \de^{ #4 } #5 \; }
% \newcommand{\IntL}[3]{ \int\limits_{ { #1 } }^{ { #2 } } \d#3 \; }



\renewcommand{\arraystretch}{1.2}

% Oznaczenia z książki Zemaniana
\newcommand{\Pf}{\mr{Pf}\,}
\newcommand{\Fp}{\mr{Fp}}
\newcommand{\Rnc}{\Rc^{ n }}
\newcommand{\Hevp}{\mbf{1}_{ + }} % Oznaczenie tenty Heaviside'a
\newcommand{\Hevm}{\mbf{1}_{ - }}

\newcommand{\vpt}{\vp( t )}
\newcommand{\vpsequ}{\set{ \vp_{ \nu }( t ) }}

% Koniec komend
% ##############################



% ####################################################################
\begin{document}
% ####################################################################



% ########################################
\Main{Analiza funkcjonalna, błędy i~uwagi.}

\vspace{\spaceTwo}
% ########################################



% ##############################
\Work{
  A. V. Balakrishnan \\
  ,,Analiza funkcjonalna stosowana'', \cite{Bal92}}

% Uwagi:
% \begin{itemize}
% \item[--]
% \end{itemize}

\CenterTB{Błędy}
\begin{center}
  \begin{tabular}{|c|c|c|c|c|}
    \hline
    & \multicolumn{2}{c|}{} & & \\
    Strona & \multicolumn{2}{c|}{Wiersz}& Jest & Powinno być \\ \cline{2-3}
    & Od góry & Od dołu &  &  \\ \hline
    13 & & 10 & $\int^{ 1 }_{ -1 }$ & $2 \int^{ 1 }_{ -1 }$ \\
    14 & 15 & & $<$ & $\leq$ \\
    15 & 18 & & otrzymywaliśmy & otrzymalibyśmy \\
    & & & & \\ \hline
  \end{tabular}
\end{center}

\vspace{\spaceOne}



% ##############################
\Work{
  I. M. Gelfand, G. E. Shilov \\
  ,,Generalized Functions: Volume~I, Properties and~Operations'',
  \cite{GS64}. }

% Uwagi:
% \begin{itemize}
% \item[--]
% \end{itemize}

\CenterTB{Błędy}
\begin{center}
  \begin{tabular}{|c|c|c|c|c|}
    \hline
    & \multicolumn{2}{c|}{} & & \\
    Strona & \multicolumn{2}{c|}{Wiersz}& Jest & Powinno być \\ \cline{2-3}
    & Od góry & Od dołu &  &  \\ \hline
    12 & 4 & & $f_{ 0 } = 0$ & $x_{ 0 } = 0$ \\
    28 & 16 & & $r \leq a$ & $r \geq a$ \\
    & & & & \\ \hline
  \end{tabular}
\end{center}

\vspace{\spaceOne}



% ##############################
\newpage

\Work{
  Armen H.~Zemanian \\
  ,,Teoria dystrybucji i analiza transformat'', \cite{Zem69} }

% ###############
\CenterTB{Uwagi}

\noi \tb{Konkretne strony}

\vspace{\spaceThree}

\start \Str{16} Rozważania o~zamkniętości ze~względu na zbieżność
przestrzeni funkcji próbnych $\D$~są w~mojej ocenie trochę chaotyczne,
spróbuję więc je jakoś rozjaśnić. W~istocie chodzi o~to,
że~definiujemy zbieżność ciągu funkcji próbnych
$\set{ \vp_{ \nu }( t ) }$, żądając od~niego dwóch własności: 1)~Dla
każdego $k$, ciąg $\set{ \vp_{ \nu }^{ ( k ) }( t ) }$ jest zbieżny
jednostajnie do jakiejś funkcji
$f_{ k }( t ), \; k = 0, 1, \ld$ \\
2)~Nośniki wszystkich funkcji $\set{ \vp_{ \nu }( t ) }$ zawarte są
w~wspólnym zbiorze zwartym. \\
Zwróćmy uwagę, że~na~razie nie możemy mówić, iż~ciąg
$\set{ \vp_{ \nu }( t ) }$ jest zbieżny do funkcji
$f( t ) = f_{ 0 }( t )$, bo~nie wiemy, czy znajomość funkcji $f( t )$
pozwala nam odtworzyć funkcje $f_{ 1 }( t ), f_{ 2 }( t ), \ld$ W~tym
momencie właściwsze byłoby powiedzenie, że~ciąg
$\set{ \vp_{ \nu }( t ) }$ jest zbieżny do~rodziny funkcji
$f_{ k }( t )$.

Jednak na mocy twierdzeń z~analizy matematycznej, które można znaleźć
np.~w~książce Schwartza, str.~649--652,~\cite{Sch79}, przy tych
warunkach funkcja $f( t )$ jest klasy $\Cinfty$ i~zachodzi
$f^{ ( k ) }( t ) = f_{ k }( t )$. Tym samy pokazaliśmy, że~zbieżność
zdefiniowana wyżej, jest rzeczywiście zbieżnością do~jakiejś funkcji.
Z~drugiego warunku zbieżności ciągu funkcji próbnych wynika,
że~funkcja $f( t )$ również ma~zwarty nośnik i~tym samym należy
do~$\D$.

Tym samym możemy stwierdzić, że~jeśli istnieje funkcja zespolona,
do~której ciąg funkcji próbnych jest zbieżny w~podanym wyżej sensie,
to~funkcja graniczna również jest funkcją próbną.

\start \Str{19} Fakt, że~funkcja dana wzorem (4) jest klasy $\Cinfty$
został, chyba przyjęty w~domyśle. Należy jednak~się nad nim zatrzymać
i~dowieść tego faktu. \Dok

\start \Str{21} Warto zatrzymać~się tu na~chwilę, nad faktem który
bardzo słusznie jest stale przypominany w~kontekście teorii
dystrybucji: nie czegoś takiego jak wartość dystrybucji w~punkcie.
Dystrybucja $f$ z~przestrzeni $\D'( \Rnc )$ jest określona
na~funkcjach z~$\D( \Rnc )$, a~nie na $\Rnc$ i~nawet dla dystrybucji
regularnej wartość $f( t )$ w~konkretnym punkcie nie ma zwykle sensu.
Punkt ma bowiem miarę Lebesgue'a równą~0, więc wartość funkcji która
reprezentuje tę~dystrybucję można w~nim dowolnie zmienić.

Jedynie jeśli w~klasie abstrakcji funkcji reprezentujących daną
dystrybucję istnieje jedna funkcja wyróżniona, to~wartość tej funkcji
jest też wartością dystrybucji w~konkretnym punkcie. Jest tak
np.~jeśli jedna z~tych funkcji jest ciągła.

\start \Str{29} Użycie we wzorze (3) symbolu
\begin{equation}
  \label{eq:Zem-s01-01}
  \Lim_{ \veps \ra 0^{ + } } \Int_{ \veps }^{ b } t^{ -3/2 } \vp( t ) \, dt,
\end{equation}
jest trochę mylące. Zdaje~się bowiem sugerować, że~całka
$\int_{ 0 }^{ b } t^{ -3/2 } \vp( t ) \, dt$ istniej i~jest równa tej
granicy, jednak całka ta jest rozbieżna jeśli $\vp( 0 ) \neq 0$.
Poprawniejsze byłoby następujące
rozumowanie. %Może się nie znam ale czy tu też nie powinno być :

Najpierw rozpatrzmy całkę
\begin{equation}
  \label{eq:Zem-s01-02}
  \Int_{ \veps }^{ b } t^{ -3/2 } \vp( t ) \, dt
  = \fr{ 2 \vp( 0 ) }{ \sqrt{ \veps } } - \fr{ 2 \vp( 0 ) }{ \sqrt{ b } }
  + \Int_{ \veps }^{ b } \fr{ \psi( t ) }{ \sqrt{ t } } \, dt.
\end{equation}
Całka ta jest skończona dla każdego $\veps$ i~powyższy wzór pozwala
nam zidentyfikować źródło rozbieżności w~granicy $\veps \ra 0^{ + }$.
Dysponując tą wiedzą, możemy zdefiniować dystrybucję
$\Pf \, t^{ -3/2 } \Hevp( t )$ wzorem
\begin{equation}
  \label{eq:Zem-s01-03}
  \lket \Pf \, t^{ -3/2 } \Hevp( t ), \vp( t ) \rket = \Lim_{ \veps \ra 0^{ + } }
  \left[ \Int_{ \veps }^{ +\infty } t^{ -3/2 } \vp( t ) \, dt
    - \fr{ 2 \vp( 0 ) }{ \sqrt{ \veps } } \right].
\end{equation}

\start \Str{32} W~tym miejscu można bez trudności, i~nawet byłoby to
bardziej naturalne, wprowadzić pseudofunkcję
$\Pf \fr{ \Hevp( -t )}{ t }$.

\start \Str{33--34} Przyjmijmy najpierw, że~jeżeli dana jest krzywa
$A$, to przez $A( t )$ będziemy oznaczać taką funkcję, że~krzywa
ta~ma~przedstawienie $( t, A( t ) )$. Teraz należy dokonać takiej
zmienny w~linii 2~(od~dołu) na~stronie~34. \\
\Jest Dlatego przesunięta cześć krzywej $B$ \\
\Pow Dlatego pole pod~krzywą $\vp( t ) B( t )$ na~przedziale
$\tau \leq t \leq \veps$

\start \Str{36} W~tym miejscu po raz pierwszy chyba w~książce
pojawia~się termin ,,obszar''. Z~kontekstu wynika, że~należy przez
niego rozumieć dowolny podzbiór $\Rnc$.

\start \Str{41} Obok nazwy \emph{zbiór zer dystrybucji}, proponowałbym
również \emph{zbiór zerowy dystrybucji}.

\start \Str{42} Logiczniej byłoby zaraz po~paragrafie \S 1.6 umieścić
paragraf \S 1.8. Pojęcie zbioru zerowego i~nośnika dystrybucji nie ma
wielkiego sensu 1.8.1 i~wynikających z~niego konsekwencji. Paragraf \S
1.7 \emph{Kilka operacji na~dystrybucjach} najlepiej byłby umieścić
jako \S 1.6.

\start \Str{47--51} Przedstawiony tu dowód twierdzenia 1.8.1, które
jest niezmiernie ważne, zawiera wiele luk, które postaram~się
w~dalszych podpunktach uzupełnić, tu zaś zbiorę potrzebne do~tego
informacje.

Zacznijmy od~tego, że~aby nie obciążać Czytelnika nowymi pojęciami,
nie używa pojęcia zbioru zwartego, zamiast tego mówi o~zbiorach
domkniętych i~ograniczonych\footnote{Jak wiadomo z~twierdzenia
  Heinego\dywiz Borela, zbiór w~$\Rn$ jest zwarty \wtw, gdy jest
  domknięty i~ograniczony.} w~$\Rn$, ponieważ jednak jest to
niewygodne, będę mówił o~zbiorach zwartych. Będziemy jeszcze
potrzebowali dwóch twierdzeń odnośnie tych zbiorów.

\begin{twr}[Walter Rudin, twr. 2.7, str.~45,~\cite{Rud98}]
  \label{twr:Zem-s01-01}
  Niech $X$ będzie lokalnie zwartą przestrzenią Hausdorffa,
  $K$~zbiorem zwartym, $U$~zbiorem otwartym i~$K \subs U$. Istnieje
  zbiór otwarty o~zwartym domknięciu taki,~że
  \begin{equation}
    \label{eq:Zem-s01-04}
    K \subs V \subs \ol{ V } \subs U.
  \end{equation}
\end{twr}
Sens tego twierdzenia jest następujący. Jeśli $K \notin \{ \es, X \}$
i~przestrzeń $X$ jest spójna, to wewnątrz zbioru otwartego $U$, można
powiększyć zbiór zwarty $K$ do~zbioru zwartego $\ol{ V }$, przy czym
$K \neq \ol{ V }$. Jest tak dlatego, że~$V$ jest otwarty,
a~w~przestrzeni Hausdorffa zbiór zwarty jest domknięty. Jeśli więc
przestrzeń jest spójna to nie może zajść równość $K = V$, chyba,
że~$K = \es$ lub~$K = X$.

Jeśli przestrzeń nie jest spójna, to~może~się zdarzyć, że~$K$ jest
maksymalną składową spójną i tym samym jest otwarto-domkniętym
zbiorem. Wtedy jak najbardziej może~się zdarzyć, iż~$K = V$.

\begin{twr}[Laurent Schwartz, str??\cite{Sch79}]
  \label{twr:Zem-s01-02}
  Niech $( X, d )$ będzie przestrzenią metryczną, w~której każdy zbiór
  domknięty i~ograniczony jest zwarty. Jeśli $K$ jest zbiorem zwartym,
  $D$ zbiorem domkniętym i~$K \cup D = \es$, to odległość zbiorów $K$
  i~$D$
  \begin{equation}
    \label{eq:Zem-s01-05}
    d( K, D ) > 0,
  \end{equation}
  do~tego odległość ta przyjmuje w~pewnym punkcie minimum.
\end{twr}
\noi Schwartz formułuje równoważne założenie twierdzenia, że~każda
kula domknięta jest zwarta. Tą~równoważność łatwo pokazać.

Będę również używał oznaczenia $\supp \vp$ na~oznaczenie nośnika
funkcji $\vpt$ (z~ang.~support).

\start \Str{48--49} \tb{Lemat 1, uzupełnienie.} Na~podstawie
twierdzenia \ref{twr:Zem-s01-01} istniej zbiór $\Psi$ o~żądanych
własnościach. Teraz na~mocy twierdzenia \eqref{twr:Zem-s01-02} mamy
$d( \Theta, \com \Psi ) = d_{ 1 } > 0$, jeśli więc
$\al_{ 1 } < d_{ 1 }$, to~nośnik funkcji
$\ga_{ \al_{ 1 } }( t - \tau )$ zawiera~się $\Psi$ dla~każdego
$t \in \Theta$, tym samym $\vp( t ) = 1$. Gwarantuje to~też,
iż~funkcja podcałkowa jest klasy $\Cinfty$. Dowiedliśmy więc,
że~$\vp( t ) = 1$ na~$\Theta$.

Jeśli teraz oznaczymy odległość $d( \Psi, \com \Om ) = d_{ 2 } > 0$
i~przyjmiemy $\al_{ 2 } < d_{ 2 }$, to dla każdego $t \in \com \Om$
nośnik funkcji $\ga_{ \al_{ 1 } }( t - \tau )$ nie przecina~się
z~$\Psi$, więc $\vpt = 0$.

\start \Str{49} W~języku polskim znaczniej lepiej od~nazwy
\emph{lokalnie skończonego pokrycia} brzmi określenie \emph{pokrycie
  lokalnie skończone}. Jego też będę dalej używał.

\start \Str{49} \tb{Lemat 2, uzupełnienie.} Zauważmy, że~każdy na~mocy
założeń, każdy domknięty podzbiór $\Om_{ k }$ jest zwarty, więc
istnienie zbiorów $\La_{ k }$ o~zadanych własnościach wynika ponownie
z~twierdzenia \eqref{twr:Zem-s01-01}.

\start \Str{50} \tb{Lemat 4, uzupełnienie.} Niech $A$ będzie zbiorem
ograniczonym w~$\Rn$, czyli zawiera~się on w~kuli $K( 0, r )$. Trzeba
teraz pokazać, że~tylko skończona ilość zbiorów $\Oc_{ \al }$ z~nowego
pokrycia może przecinać~się z~tą kulą. Oznaczymy majorantę średnicy
zbiorów przez $\mr{Diam}_{ \mf{ C } }$, a~przez $M$ minimalną
odległość punktu prostopadłościanu reprezentowanego układem liczb
$\{ m_{ 1 }, \ld, m_{ n } \}$ od~punktu $0$, czyli
\begin{displaymath}
  M = \left( \Sum_{ i = 0 }^{ n } m_{ i } \right)^{ \fr{ 1 }{ 2 } }.  
\end{displaymath}
Zbiór $\Oc_{ \al }$ może przeciąć zadaną kulę, tylko jeśli
$M - \mr{Diam}_{ \mf{C} } < r$, czyli mówiąc prościej,
jeśli~prostopadłościan jest na tyle blisko, że~zbiór $\Oc_{ \al }$
może dosięgnąć kuli $K( 0, r )$. Ponieważ istnieje tylko skończona
ilość układów liczb $\{ m_{ 1 }, \ld, m_{ n } \}$ spełniających tę
nierówność, więc tylko skończoną ilość razy będziemy wybierali
skończoną rodzinę zbiorów $\Oc_{ \al }$ takich, że~jest możliwe
by~$K( 0, r ) \cap \Oc_{ \al } \neq \O$.

\start \Str{51} Cel przedstawionej tu konstrukcji pokrycia jest
następujący. Potrzebujemy pokrycia lokalnie skończonego~$\Om_{ \nu }$,
którego elementy albo nie przecinają~się z~$\Xi$, albo
$\Om_{ \nu } \subs \Theta$, gdy $\Om_{ \nu } \cap \Xi \neq \O$.
Rozkład jedności dla tego pokrycia pozwoli pokazać, że~funkcję~$\psi$
można rozłożyć na~funkcje o~nośnikach w~zbiorach otwartych na~których
dystrybucja $f$~jest równa $0$.

Wydaje mi~się, że~do tej konstrukcji zbiór $\Xi_{ 1 }$ jest
wprowadzony niepotrzebnie.

\newpage
\CenterTB{Błędy}
\begin{center}
  \begin{tabular}{|c|c|c|c|c|}
    \hline
    & \multicolumn{2}{c|}{} & & \\
    Strona & \multicolumn{2}{c|}{Wiersz}& Jest & Powinno być \\ \cline{2-3}
    & Od góry & Od dołu &  &  \\ \hline
    12 & & 6 & wartości & tylko rzeczywiste \\
    14 & 6 & & całka & całka przedstawiająca $\vp_{ \al }'( t )$ \\
    16 & & 6 & ciągi & funkcje \\
    17 & & 4 & z przestrzeni & na przestrzeni \\
    17 & & 3 & która ma & które mają \\
    17 & & 2 & \emph{przestrzeni} & \emph{funkcjonału} \\
    18 & 6 & & $\{ \vp_{ \nu } \}\nu \ra \infty$ & $\{ \vp_{ \nu } \}$,
                                                   przy $\nu \ra \infty$ \\
    18 & & 10 & dystrybuantę & dystrybucję \\
    21 & 9 & & $\D$ & $\D'$ \\
    25 & & 1 & dąży & dąży punktowo \\
    26 & 5 & & pochodnym od & pochodną \\
    27 & 7 & & $\del^{ ( 1 ) }\bsym{ ( t } )$ & $\del^{ ( 1 ) }( t )$ \\
    27 & & 3 & $\vp^{ ( k \bsym{ ) } }\bsym{ ( t ) } $
           & $\vp^{ ( k ) }( t )$ \\
    28 & & 17 & dążą & jednocześnie dążą \\
    28 & & 4 & $\del^{ ( 2 ) }( 0 )$ & $\vp^{ ( 2 ) }( 0 )$ \\
    29 & & 8 & $t = 0$ & $t \neq 0$ \\
    31 & 6 & & $\vp( )$ & $\vp( t )$ \\
    31 & 8 & & $\Pf \, | t |^{ \be } \; 1_{ + }( t )$
           & $\Pf \, | t |^{ \be } \; 1_{ + }( -t )$ \\
    33 & & 1 & skończoną & nieskończoną \\
    34 & & 1 & $\vp( 0 ) \log\veps$ & $\vp( 0 ) \log\veps.$ \\
    35 & 1 & & tj.~$\lket B, \vp \rket$. Wartość & Wartość \\
    36 & & 7 & $\pr t_{ 2 }^{ \: k_{ 2 } } \ld \pr t_{ n }^{ \: k_{ n } }$
           & $\pr t_{ 2 }^{ k_{ 2 } } \ld \pr t_{ n }^{ k_{ n } }$ \\
    37 & & 16 & \emph{mamy $\vp_{ \al }( t )$ równe}
           & \emph{$\vp_{ \al }( t )$ jest równa} \\
    37 & & 1 & wyboru układu & układu \\
    38 & 4 & & $\{ 0, 0, \ld, 0 ) \}$ & $\{ 0, 0, \ld, 0 \}$) \\
    41 & & 2 & dystrybucji & dystrybucji regularnej \\
    45 & & 10 & $\bsym{f}$ & $f$ \\
    50 & 13 & & \emph{jedną} & \emph{stałą} \\
    50 & & 3 & $C$ & $\mf{C}$ \\
    55 & 9 & & zbieżny w $\D'$ & zbieżny \\
    71 & & 5 & $M$ i $k$ stałych & $M$ rzeczywistego i $k$ naturalnego \\
    71 & & 4 & gdzie $p$ & gdzie $p = k - 1$, a $g( t )$\\
    71 & & 1 & $\leq M | \nu |^{ -2 }$ & $\leq M | \nu |^{ -1 }$\\
    & & & & \\ \hline
  \end{tabular}
\end{center}
\noi
\StrWd{21}{10} \\
\Jest opisem niewykończonego \\
\Pow niepełnym opisem pewnego \\
\StrWg{34}{2} \\
\Jest Odpowiednie nachylenie wskazuje \\
\Pow Odpowiednią funkcje przedstawia \\
\StrWd{38}{4} \\
\Jest choć może nie być sprecyzowana jego wartość \\
\Pow choć jego wartość może nie być podana jawnie \\
\StrWd{40}{6} \\
\Jest wartości liczbowych \\
\Pow wartości liczbowej w~punkcie \\
\StrWg{50}{14} \\
\Jest \emph{pokrycie $\Rn$.} \\
\Pow \emph{pokrycie $\Rn$, przy czym średnice wszystkich jego zbiorów,
  są~ograniczone przez tą samą stałą, co~dla rodziny $\mf{C}_{ T }$.} \\









% #####################################################################
% #####################################################################
\bibliographystyle{alpha} \bibliography{Bibliography}{}



\end{document}
