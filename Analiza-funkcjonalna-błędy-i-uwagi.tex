% Autor: Kamil Ziemian
% Korekta: Wojciech Dyba

% --------------------------------------------------------------------
% Podstawowe ustawienia i pakiety
% --------------------------------------------------------------------
\RequirePackage[l2tabu, orthodox]{nag} % Wykrywa przestarzałe i niewłaściwe
% sposoby używania LaTeXa. Więcej jest w l2tabu English version.
\documentclass[a4paper,11pt]{article}
% {rozmiar papieru, rozmiar fontu}[klasa dokumentu]
\usepackage[MeX]{polski} % Polonizacja LaTeXa, bez niej będzie pracował
% w języku angielskim.
\usepackage[utf8]{inputenc} % Włączenie kodowania UTF-8, co daje dostęp
% do polskich znaków.
\usepackage{lmodern} % Wprowadza fonty Latin Modern.
\usepackage[T1]{fontenc} % Potrzebne do używania fontów Latin Modern.



% ----------------------------
% Podstawowe pakiety (niezwiązane z ustawieniami języka)
% ----------------------------
\usepackage{microtype} % Twierdzi, że poprawi rozmiar odstępów w tekście.
\usepackage{graphicx} % Wprowadza bardzo potrzebne komendy do wstawiania
% grafiki.
\usepackage{verbatim} % Poprawia otoczenie VERBATIME.
\usepackage{textcomp} % Dodaje takie symbole jak stopnie Celsiusa,
% wprowadzane bezpośrednio w tekście.
\usepackage{vmargin} % Pozwala na prostą kontrolę rozmiaru marginesów,
% za pomocą komend poniżej. Rozmiar odstępów jest mierzony w calach.
% ----------------------------
% MARGINS
% ----------------------------
\setmarginsrb
{ 0.7in} % left margin
{ 0.6in} % top margin
{ 0.7in} % right margin
{ 0.8in} % bottom margin
{  20pt} % head height
{0.25in} % head sep
{   9pt} % foot height
{ 0.3in} % foot sep



% ------------------------------
% Często przydatne pakiety
% ------------------------------
\usepackage{csquotes} % Pozwala w prosty sposób wstawiać cytaty do tekstu.
\usepackage{xcolor} % Pozwala używać kolorowych czcionek (zapewne dużo
% więcej, ale ja nie potrafię nic o tym powiedzieć).



% ------------------------------
% Pakiety do tekstów z nauk przyrodniczych
% ------------------------------
\let\lll\undefined % Amsmath gryzie się z językiem pakietami do języka
% polskiego, bo oba definiują komendę \lll. Aby rozwiązać ten problem
% oddefiniowuję tę komendę, ale może tym samym pozbywam się dużego Ł.
\usepackage[intlimits]{amsmath} % Podstawowe wsparcie od American
% Mathematical Society (w skrócie AMS)
\usepackage{amsfonts, amssymb, amscd, amsthm} % Dalsze wsparcie od AMS
% \usepackage{siunitx} % Dla prostszego pisania jednostek fizycznych
\usepackage{upgreek} % Ładniejsze greckie litery
% Przykładowa składnia: pi = \uppi
\usepackage{slashed} % Pozwala w prosty sposób pisać slash Feynmana.
\usepackage{calrsfs} % Zmienia czcionkę kaligraficzną w \mathcal
% na ładniejszą. Może w innych miejscach robi to samo, ale o tym nic
% nie wiem.



% ##########
% Tworzenie otoczeń "Twierdzenie", "Definicja", "Lemat", etc.
\newtheorem{twr}{Twierdzenie} % Komenda wprowadzająca otoczenie
% ,,twr'' do pisania twierdzeń matematycznych
\newtheorem{defin}{Definicja} % Analogicznie jak powyżej
\newtheorem{wni}{Wniosek}



% ----------------------------
% Pakiety napisane przez użytkownika.
% Mają być w tym samym katalogu to ten plik .tex
% ----------------------------
\usepackage{analizafunkcjonalna}  % Pakiet napisany konkretnie dla tego
% pliku.
\usepackage{latexshortcuts}
\usepackage{mathshortcuts}



% --------------------------------------------------------------------
% Dodatkowe ustawienia dla języka polskiego
% --------------------------------------------------------------------
\renewcommand{\thesection}{\arabic{section}.}
% Kropki po numerach rozdziału (polski zwyczaj topograficzny)
\renewcommand{\thesubsection}{\thesection\arabic{subsection}}
% Brak kropki po numerach podrozdziału



% ----------------------------
% Ustawienia różnych parametrów tekstu
% ----------------------------
\renewcommand{\arraystretch}{1.2} % Ustawienie szerokości odstępów między
% wierszami w tabelach.





% ----------------------------
% Pakiet "hyperref"
% Polecano by umieszczać go na końcu preambuły.
% ----------------------------
\usepackage{hyperref} % Pozwala tworzyć hiperlinki i zamienia odwołania
% do bibliografii na hiperlinki.





% --------------------------------------------------------------------
% Tytuł, autor, data
\title{Analiza funkcjonalna --~błędy i~uwagi}

% \author{}
% \date{}
% --------------------------------------------------------------------





% ####################################################################
\begin{document}
% ####################################################################



% ######################################
\maketitle % Tytuł całego tekstu
% ######################################



% ##################
\Work{ % Autor i tytuł dzieła
  A. V. Balakrishnan \\
  ,,Analiza funkcjonalna stosowana'',
  \cite{BalakrishnanAnalizaFunkcjonalnaStosowana92} }

% Uwagi:
% \begin{itemize}
% \item[--]
% \end{itemize}

\CenterTB{Błędy}
\begin{center}
  \begin{tabular}{|c|c|c|c|c|}
    \hline
    & \multicolumn{2}{c|}{} & & \\
    Strona & \multicolumn{2}{c|}{Wiersz} & Jest
                              & Powinno być \\ \cline{2-3}
    & Od góry & Od dołu & & \\
    \hline
    13  & & 10 & $\int^{ 1 }_{ -1 }$ & $2 \int^{ 1 }_{ -1 }$ \\
    14  & 15 & & $<$ & $\leq$ \\
    15  & 18 & & otrzymywaliśmy & otrzymalibyśmy \\
    % & & & & \\
    \hline
  \end{tabular}
\end{center}

\vspace{\spaceTwo}





% ##################
\Work{ % Autorzy i tytuł dzieła
  Anton Deitmar, Siegrfied Echterhoff \\
  ,,Principles~of Harmonic Analysis'',
  \cite{DeitmarEcherhoffPrinciplesOfHarmonicAnalysis09} }


\CenterTB{Uwagi}

% \noi \tb{Konkretne strony}

% \vspace{\spaceFour}


\start \Str{42} \tb{Twierdzenie 2.2.6. Wzór na~promień spektralny.}
Aby~wykazać, że~$\la^{ n } \in \sigma_{ \mc{ A } }( a^{ n })$ należy
pokazać, iż~element
\begin{equation}
  \label{eq:DE-1}
  ( \la 1 - a ) \Sum_{ j = 0 }^{ n - 1 } \la^{ j } a^{ n - 1 - j }
\end{equation}
jest nieodwracalny. Nie jest to jednak dla~mnie oczywiste. Arkadiusz
Bochniak powiedział, żeby zobaczyć na~dowód twierdzenia widmie
wielomian operatora\footnote{Ang. polynomial spectral mapping
  theorem.}. Należy przy tym zauważyć, że~z~tego twierdzenie wynika
od~razu poszukiwana własność.

\vspace{\spaceFour}


\start \Str{46} \tb{Lemat 2.4.2.} W~dowodzie tego twierdzenie trzeba
chyba rozumować następująco. Chcemy pokazać,
że~$m( a ) \leq \norm{ a }$ dla każdego $a \in \mc{A}$.

Rozpatrujemy dwa przypadki. 1) Gdy $m( a ) = 0$, wtedy oczywiście
$m( a ) = 0 \leq \norm{ a }$. 2). Gdy $m( a ) \neq 0$. Wówczas
$m( a - m( a ) 1 ) = 0$. Gdyby istniał element $b$~odwrotny
do~$a - m( a ) 1$, to~zachodziłoby
\begin{equation}
  1 = m( 1 ) = m( b ( a - m( a ) 1 ) ) = m( b ) m( a - m( a ) )
  = m( b ) 0 = 0,
\end{equation}
co jest niemożliwe. Dalej można już postępować jak w~zaprezentowanym
dowodzie.


\CenterTB{Błędy}
\begin{center}
  \begin{tabular}{|c|c|c|c|c|}
    \hline
    & \multicolumn{2}{c|}{} & & \\
    Strona & \multicolumn{2}{c|}{Wiersz} & Jest
                              & Powinno być \\ \cline{2-3}
    & Od góry & Od dołu & & \\
    \hline
    42  & 10 & & $s\: u\: p$ & $\sup$ \\
    44  &  9 & & Banach-Algebras & Banach algebras \\
    % & & & & \\
    % & & & & \\
    \hline
  \end{tabular}
\end{center}

\vspace{\spaceTwo}





% ##################
\Work{ % Autorzy i tytuł dziełą
  I. M. Gelfand, G. E. Shilov \\
  ,,Generalized Functions: Volume~I, Properties and~Operations'',
  \cite{GS64}. }

% Uwagi:
% \begin{itemize}
% \item[--]
% \end{itemize}

\CenterTB{Błędy}
\begin{center}
  \begin{tabular}{|c|c|c|c|c|}
    \hline
    & \multicolumn{2}{c|}{} & & \\
    Strona & \multicolumn{2}{c|}{Wiersz} & Jest
                              & Powinno być \\ \cline{2-3}
    & Od góry & Od dołu & & \\
    \hline
    12  &  4 & & $f_{ 0 } = 0$ & $x_{ 0 } = 0$ \\
    28  & 16 & & $r \leq a$ & $r \geq a$ \\
    % & & & & \\
    \hline
  \end{tabular}
\end{center}

\vspace{\spaceTwo}



% ####################
\newpage
\Work{ % Autor i tytuł dzieła
  Armen H.~Zemanian \\
  ,,Teoria dystrybucji i~analiza transformat'',
  \cite{ZemanianTeoriaDystrybucji69} }


\CenterTB{Uwagi}

\noi \tb{Konkretne strony}

\vspace{\spaceFour}


\start \Str{16} Rozważania o~zamkniętości ze~względu na zbieżność
przestrzeni funkcji próbnych $\Dc$~są w~mojej ocenie trochę
chaotyczne, spróbuję więc je jakoś rozjaśnić. W~istocie chodzi o~to,
że~definiujemy zbieżność ciągu funkcji próbnych
$\set{ \vp_{ \nu }( t ) }$, żądając od~niego dwóch własności: 1)~Dla
każdego $k$, ciąg $\set{ \vp_{ \nu }^{ ( k ) }( t ) }$ jest zbieżny
jednostajnie do jakiejś funkcji
$f_{ k }( t ), \; k = 0, 1, \ld$ \\
2)~Nośniki wszystkich funkcji $\set{ \vp_{ \nu }( t ) }$ zawarte są
w~wspólnym zbiorze zwartym. \\
Zwróćmy uwagę, że~na~razie nie możemy mówić, iż~ciąg
$\set{ \vp_{ \nu }( t ) }$ jest zbieżny do funkcji
$f( t ) = f_{ 0 }( t )$, bo~nie wiemy, czy znajomość funkcji $f( t )$
pozwala nam odtworzyć funkcje $f_{ 1 }( t ), f_{ 2 }( t ), \ld$ W~tym
momencie właściwsze byłoby powiedzenie, że~ciąg
$\set{ \vp_{ \nu }( t ) }$ jest zbieżny do~rodziny funkcji
$f_{ k }( t )$.

Jednak na mocy twierdzeń z~analizy matematycznej, które można znaleźć
np.~w~książce Schwartza, str.~649--652,~\cite{Schwartz79}, przy tych
warunkach funkcja $f( t )$ jest klasy $\Cinfty$ i~zachodzi
$f^{ ( k ) }( t ) = f_{ k }( t )$. Tym samy pokazaliśmy, że~zbieżność
zdefiniowana wyżej, jest rzeczywiście zbieżnością do~jakiejś funkcji.
Z~drugiego warunku zbieżności ciągu funkcji próbnych wynika,
że~funkcja $f( t )$ również ma~zwarty nośnik i~tym samym należy
do~$\Dc$.

Tym samym możemy stwierdzić, że~jeśli istnieje funkcja zespolona,
do~której ciąg funkcji próbnych jest zbieżny w~podanym wyżej sensie,
to~funkcja graniczna również jest funkcją próbną.

\vspace{\spaceFour}


\start \Str{19} Fakt, że~funkcja dana wzorem (4) jest klasy $\CinftyR$
został, chyba przyjęty w~domyśle. Należy jednak~się nad nim zatrzymać
i~dowieść tego faktu. \Dok

\vspace{\spaceFour}


\start \Str{20} Aby wywód był pełny należy jeszcze dowieść,
że~$[ \vp( t ) ]^{ \fr{ 1 }{ n } } = \sqrt[ n ]{ \vp( t ) }$ jest
klasy $\CinftyR$.

\vspace{\spaceFour}


\start \Str{21} Warto zatrzymać~się tu na~chwilę, nad faktem który
bardzo słusznie jest stale przypominany w~kontekście teorii
dystrybucji: nie czegoś takiego jak wartość dystrybucji w~punkcie.
Dystrybucja $f$ z~przestrzeni $\Dc'( \Rnc )$ jest określona
na~funkcjach z~$\Dc( \Rnc )$, a~nie na $\Rnc$ i~nawet dla dystrybucji
regularnej wartość $f( t )$ w~konkretnym punkcie nie ma zwykle sensu.
Punkt ma bowiem miarę Lebesgue'a równą~0, więc wartość funkcji która
reprezentuje tę~dystrybucję można w~nim dowolnie zmienić.

Jedynie jeśli w~klasie abstrakcji funkcji reprezentujących daną
dystrybucję istnieje jedna funkcja wyróżniona, to~wartość tej funkcji
jest też wartością dystrybucji w~konkretnym punkcie. Jest tak
np.~jeśli jedna z~tych funkcji jest ciągła.

\vspace{\spaceFour}


\start \Str{29} Użycie we wzorze~(3) symbolu
\begin{equation}
  \label{eq:Zem-s01-01}
  \Lim_{ \veps \to 0^{ + } } \Int_{ \veps }^{ b } t^{ -3/2 } \vp( t ) \, dt,
\end{equation}
jest trochę mylące. Zdaje~się bowiem sugerować, że~całka
$\int_{ 0 }^{ b } t^{ -3/2 } \vp( t ) \, dt$ istniej i~jest równa tej
granicy, jednak całka ta jest rozbieżna jeśli $\vp( 0 ) \neq 0$.
Poprawniejsze byłoby następujące rozumowanie.

Najpierw rozpatrzmy całkę
\begin{equation}
  \label{eq:Zem-s01-02}
  \Int_{ \veps }^{ b } t^{ -3/2 } \vp( t ) \, dt
  = \fr{ 2 \vp( 0 ) }{ \sqrt{ \veps } } - \fr{ 2 \vp( 0 ) }{ \sqrt{ b } }
  + \Int_{ \veps }^{ b } \fr{ \psi( t ) }{ \sqrt{ t } } \, dt.
\end{equation}
Całka ta jest skończona dla każdego $\veps$ i~powyższy wzór pozwala
nam zidentyfikować źródło rozbieżności w~granicy $\veps \to 0^{ + }$.
Dysponując tą wiedzą, możemy zdefiniować dystrybucję
$\Pf \, t^{ -3/2 } \Hevp( t )$ wzorem
\begin{equation}
  \label{eq:Zem-s01-03}
  \lket \Pf \, t^{ -3/2 } \Hevp( t ), \vp( t ) \rket = \Lim_{ \veps \to 0^{ + } }
  \left[ \Int_{ \veps }^{ +\infty } t^{ -3/2 } \vp( t ) \, dt
    - \fr{ 2 \vp( 0 ) }{ \sqrt{ \veps } } \right].
\end{equation}

\vspace{\spaceFour}


\start \Str{32} W~tym miejscu można bez trudności, i~nawet byłoby to
bardziej naturalne, wprowadzić pseudofunkcję
$\Pf \fr{ \Hevp( -t )}{ t }$.

\vspace{\spaceFour}


\start \Str{33--34} Przyjmijmy najpierw, że~jeżeli dana jest krzywa
$A$, to przez $A( t )$ będziemy oznaczać taką funkcję, że~krzywa
ta~ma~przedstawienie $( t, A( t ) )$. Teraz należy dokonać takiej
zmienny w~linii 2~(od~dołu) na~stronie~34. \\
\Jest Dlatego przesunięta cześć krzywej $B$ \\
\Pow Dlatego pole pod~krzywą $\vp( t ) B( t )$ na~przedziale
$\tau \leq t \leq \veps$

\vspace{\spaceFour}


\start \Str{36} W~tym miejscu po raz pierwszy chyba w~książce
pojawia~się termin ,,obszar''. Z~kontekstu wynika, że~należy przez
niego rozumieć dowolny podzbiór $\Rnc$.

\vspace{\spaceFour}


\start \Str{41} Obok nazwy \emph{zbiór zer dystrybucji}, proponowałbym
również \emph{zbiór zerowy dystrybucji}.

\vspace{\spaceFour}


\start \Str{42} Logiczniej byłoby zaraz po~paragrafie \S 1.6 umieścić
paragraf \S 1.8. Pojęcie zbioru zerowego i~nośnika dystrybucji nie ma
wielkiego sensu 1.8.1 i~wynikających z~niego konsekwencji. Paragraf \S
1.7 \emph{Kilka operacji na~dystrybucjach} najlepiej byłby umieścić
jako \S 1.6.

\vspace{\spaceFour}


\start \Str{47--51} Przedstawiony tu dowód twierdzenia 1.8.1, które
jest niezmiernie ważne, zawiera wiele luk, które postaram~się
w~dalszych podpunktach uzupełnić, tu zaś zbiorę potrzebne do~tego
informacje.

Zacznijmy od~tego, że~aby nie obciążać Czytelnika nowymi pojęciami,
nie używa pojęcia zbioru zwartego, zamiast tego mówi o~zbiorach
domkniętych i~ograniczonych\footnote{Jak wiadomo z~twierdzenia
  Heinego\dywiz Borela, zbiór w~$\Rn$ jest zwarty \wtw, gdy jest
  domknięty i~ograniczony.} w~$\Rn$, ponieważ jednak jest to
niewygodne, będę mówił o~zbiorach zwartych. Będziemy jeszcze
potrzebowali dwóch twierdzeń odnośnie tych zbiorów.

\begin{twr}[Walter Rudin, twr. 2.7, str.~45,~\cite{Rudin98}]
  \label{twr:Zem-s01-01}
  Niech $X$ będzie lokalnie zwartą przestrzenią Hausdorffa,
  $K$~zbiorem zwartym, $U$~zbiorem otwartym i~$K \subs U$. Istnieje
  zbiór otwarty o~zwartym domknięciu taki,~że
  \begin{equation}
    \label{eq:Zem-s01-04}
    K \subs V \subs \ol{ V } \subs U.
  \end{equation}
\end{twr}
Sens tego twierdzenia jest następujący. Jeśli $K \notin \{ \es, X \}$
i~przestrzeń $X$ jest spójna, to wewnątrz zbioru otwartego $U$, można
powiększyć zbiór zwarty $K$ do~zbioru zwartego $\ol{ V }$, przy czym
$K \neq \ol{ V }$. Jest tak dlatego, że~$V$ jest otwarty,
a~w~przestrzeni Hausdorffa zbiór zwarty jest domknięty. Jeśli więc
przestrzeń jest spójna to nie może zajść równość $K = V$, chyba,
że~$K = \es$ lub~$K = X$.

Jeśli przestrzeń nie jest spójna, to~może~się zdarzyć, że~$K$ jest
maksymalną składową spójną i tym samym jest otwarto-domkniętym
zbiorem. Wtedy jak najbardziej może~się zdarzyć, iż~$K = V$.

\begin{twr}[Laurent Schwartz, str??\cite{Schwartz79}]
  \label{twr:Zem-s01-02}
  Niech $( X, d )$ będzie przestrzenią metryczną, w~której każdy zbiór
  domknięty i~ograniczony jest zwarty. Jeśli $K$ jest zbiorem zwartym,
  $D$ zbiorem domkniętym i~$K \cup D = \es$, to odległość zbiorów $K$
  i~$D$
  \begin{equation}
    \label{eq:Zem-s01-05}
    d( K, D ) > 0,
  \end{equation}
  do~tego odległość ta przyjmuje w~pewnym punkcie minimum.
\end{twr}
\noi Schwartz formułuje równoważne założenie twierdzenia, że~każda
kula domknięta jest zwarta. Tą~równoważność łatwo pokazać.

Będę również używał oznaczenia $\supp \vp$ na~oznaczenie nośnika
funkcji $\vpt$ (z~ang.~support).

\vspace{\spaceFour}


\start \Str{48--49} \tb{Lemat 1, uzupełnienie.} Na~podstawie
twierdzenia \ref{twr:Zem-s01-01} istniej zbiór $\Psi$ o~żądanych
własnościach. Teraz na~mocy twierdzenia \eqref{twr:Zem-s01-02} mamy
$d( \Theta, \com \Psi ) = d_{ 1 } > 0$, jeśli więc
$\al_{ 1 } < d_{ 1 }$, to~nośnik funkcji
$\ga_{ \al_{ 1 } }( t - \tau )$ zawiera~się $\Psi$ dla~każdego
$t \in \Theta$, tym samym $\vp( t ) = 1$. Gwarantuje to~też,
iż~funkcja podcałkowa jest klasy $\Cinfty$. Dowiedliśmy więc,
że~$\vp( t ) = 1$ na~$\Theta$.

Jeśli teraz oznaczymy odległość $d( \Psi, \com \Om ) = d_{ 2 } > 0$
i~przyjmiemy $\al_{ 2 } < d_{ 2 }$, to dla każdego $t \in \com \Om$
nośnik funkcji $\ga_{ \al_{ 1 } }( t - \tau )$ nie przecina~się
z~$\Psi$, więc $\vpt = 0$.

\vspace{\spaceFour}


\start \Str{49} W~języku polskim znaczniej lepiej od~nazwy
\emph{lokalnie skończonego pokrycia} brzmi określenie \emph{pokrycie
  lokalnie skończone}. Jego też będę dalej używał.

\vspace{\spaceFour}


\start \Str{49} \tb{Lemat 2, uzupełnienie.} Zauważmy, że~każdy na~mocy
założeń, każdy domknięty podzbiór $\Om_{ k }$ jest zwarty, więc
istnienie zbiorów $\La_{ k }$ o~zadanych własnościach wynika ponownie
z~twierdzenia \eqref{twr:Zem-s01-01}.

\vspace{\spaceFour}


\start \Str{50} \tb{Lemat 4, uzupełnienie.} Niech $A$ będzie zbiorem
ograniczonym w~$\Rn$, czyli zawiera~się on w~kuli $K( 0, r )$. Trzeba
teraz pokazać, że~tylko skończona ilość zbiorów $\Oc_{ \al }$ z~nowego
pokrycia może przecinać~się z~tą kulą. Oznaczymy majorantę średnicy
zbiorów przez $\mr{Diam}_{ \mf{ C } }$, a~przez $M$ minimalną
odległość punktu prostopadłościanu reprezentowanego układem liczb
$\{ m_{ 1 }, \ld, m_{ n } \}$ od~punktu $0$, czyli
\begin{equation}
  \label{eq:Zem-s01-06}
  M = \left( \Sum_{ i = 0 }^{ n } m_{ i } \right)^{ \fr{ 1 }{ 2 } }.
\end{equation}
Zbiór $\Oc_{ \al }$ może przeciąć zadaną kulę, tylko jeśli
$M - \mr{Diam}_{ \mf{C} } < r$, czyli mówiąc prościej,
jeśli~prostopadłościan jest na tyle blisko, że~zbiór $\Oc_{ \al }$
może dosięgnąć kuli $K( 0, r )$. Ponieważ istnieje tylko skończona
ilość układów liczb $\{ m_{ 1 }, \ld, m_{ n } \}$ spełniających tę
nierówność, więc tylko skończoną ilość razy będziemy wybierali
skończoną rodzinę zbiorów $\Oc_{ \al }$ takich, że~jest możliwe
by~$K( 0, r ) \cap \Oc_{ \al } \neq \O$.

\vspace{\spaceFour}


\start \Str{51} Cel przedstawionej tu konstrukcji jest następujący.
Potrzebujemy pokrycia lokalnie skończonego~$\Om_{ \nu }$, którego
elementy albo nie przecinają~się z~$\Xi$, albo
$\Om_{ \nu } \subs \Theta$, gdy $\Om_{ \nu } \cap \Xi \neq \O$.
Rozkład jedności dla tego pokrycia pozwoli pokazać, że~funkcję~$\psi$
można rozłożyć na~funkcje o~nośnikach w~zbiorach otwartych na~których
dystrybucja $f$~jest równa $0$.

Wydaje mi~się, że~do tej konstrukcji zbiór $\Xi_{ 1 }$ jest
wprowadzony niepotrzebnie.

\vspace{\spaceFour}


\start \Str{59}

\vspace{\spaceFour}


\start \Str{62}

\vspace{\spaceFour}


\start \Str{71} Wzór
\begin{equation}
  \label{eq:Zem-s01-07}
  f( t ) = f_{ c }( t ) - \Sum_{ \nu = -1 }^{ -\infty } \Del f_{ \nu }
  \Hevp( t_{ \nu } - t ) + \Sum_{ \nu = 0 }^{ \infty } \Del f_{ \nu }
  \Hevp( t - t_{ \nu } ),
\end{equation}
zawiera pewną subtelność, którą warto wyjaśnić. \Dok

\vspace{\spaceFour}


\start \Str{74} Fakt, że~iloraz różnicowy
\begin{equation}
  \label{eq:Zem-s01-08}
  \fr{ \vp( t - \Delt ) - \vp( t ) }{ \Delt }
\end{equation}
dąży do $-\vpDer$ przy $\DelToZero$ wynika od~razu z~definicji
pochodnej i~twierdzenia o~składaniu granic\footnote{Udowodnij i~zapisz
  gdzieś to twierdzenie.}, problem jest tylko taki, że~jest to
zbieżność punktowa. Aby zaś skorzystać z~ciągłości dystrybucji
potrzebujemy udowodnić zbieżność jednostajną tego ilorazy do~$-\vpDer$
oraz jednostajną zbieżność jego $k$\dywiz pochodnej
do~$-\vp^{ ( k + 1 ) }( t )$. Inaczej mówiąc dla $k = 0, 1, 2, \ld$
musi zachodzić
\begin{equation}
  \label{eq:Zem-s01-09}
  \Lim_{ \DelToZero } \fr{ \vp^{ ( k ) }( t - \Delt ) - \vp^{ ( k ) }( t ) }
  { \Delt } = -\vp^{ ( k + 1 ) }( t ),
\end{equation}
gdzie zbieżność jest rozumiana jako zbieżność jednostajna.

Dowód podany przez Zemaniana nie jest jedynym możliwym, ale~jest
bezpośredni rachunkowo, przez co stosunkowo łatwy do~zrozumienia.

\newpage
\CenterTB{Błędy}
\begin{center}
  \begin{tabular}{|c|c|c|c|c|}
    \hline
    & \multicolumn{2}{c|}{} & & \\
    Strona & \multicolumn{2}{c|}{Wiersz} & Jest
                              & Powinno być \\ \cline{2-3}
    & Od góry & Od dołu & & \\
    \hline
    % 16 & & 6 & ciągi & funkcje \\
    % 17 & & 4 & z przestrzeni & na przestrzeni \\
    % 17 & & 3 & która ma & które mają \\
    % 17 & & 2 & \emph{przestrzeni} & \emph{funkcjonału} \\
    % 18 & 6 & & $\{ \vp_{ \nu } \}\nu \to \infty$ &
    % $\{ \vp_{ \nu } \}$,
    % przy $\nu \to \infty$ \\
    % 18 & & 10 & dystrybuantę & dystrybucję \\
    % 21 & 9 & & $\Dc$ & $\Dc'$ \\
    % 25 & & 1 & dąży & dąży punktowo \\
    % 26 & 5 & & pochodnym od & pochodną \\
    12  & &  6 & wartości & tylko rzeczywiste \\
    14  &  6 & & całka & całka przedstawiająca $\vp_{ \al }'( t )$ \\
    16  & &  6 & ciągi & funkcje \\
    17  & &  4 & z przestrzeni & na przestrzeni \\
    17  & &  3 & która ma & które mają \\
    17  & &  2 & \emph{przestrzeni} & \emph{funkcjonału} \\
    18  &  6 & & $\{ \vp_{ \nu } \}\nu \to \infty$
           & $\{ \vp_{ \nu } \}$, przy $\nu \to \infty$ \\
    18  & & 10 & dystrybuantę & dystrybucję \\
    21  &  9 & & $\Dc$ & $\Dc'$ \\
    25  & &  1 & dąży & dąży punktowo \\
    26  &  5 & & pochodnym od & pochodną \\
    27  &  7 & & $\del^{ ( 1 ) }\bsym{ ( t } )$ & $\del^{ ( 1 ) }( t )$ \\
    27  & &  9 & $< f,\, \vp >$ & $\dyst{ f }{ \vp }$ \\
    27  & &  3 & $\vp^{ ( k \bsym{ ) } }\bsym{ ( t ) } $
           & $\vp^{ ( k ) }( t )$ \\
    28  & & 17 & dążą & jednocześnie dążą \\
    28  & &  4 & $\del^{ ( 2 ) }( 0 )$ & $\vp^{ ( 2 ) }( 0 )$ \\
    29  & &  8 & $t = 0$ & $t \neq 0$ \\
    31  &  6 & & $\vp( )$ & $\vp( t )$ \\
    31  &  8 & & $\Pf \, | t |^{ \be } \; 1_{ + }( t )$
           & $\Pf \, | t |^{ \be } \; 1_{ + }( -t )$ \\
    33  & &  1 & skończoną & nieskończoną \\
    34  & &  1 & $\vp( 0 ) \log\veps$ & $\vp( 0 ) \log\veps.$ \\
    35  &  1 & & tj.~$\dyst{ B }{ \vp }$. Wartość & Wartość \\
    36  & &  7 & $\pr t_{ 2 }^{ \: k_{ 2 } } \ld \pr t_{ n }^{ \: k_{ n } }$
           & $\pr t_{ 2 }^{ k_{ 2 } } \ld \pr t_{ n }^{ k_{ n } }$ \\
    37  & & 16 & \emph{mamy $\vp_{ \al }( t )$ równe}
           & \emph{$\vp_{ \al }( t )$ jest równa} \\
    37  & &  1 & wyboru układu & układu \\
    38  &  4 & & $\{ 0, 0, \ld, 0 ) \}$ & $\{ 0, 0, \ld, 0 \}$) \\
    41  & &  2 & dystrybucji & dystrybucji regularnej \\
    45  & & 10 & $\bsym{f}$ & $f$ \\
    50  & 13 & & \emph{jedną} & \emph{stałą} \\
    50  & &  3 & $C$ & $\mf{C}$ \\
    55  &  9 & & zbieżny w $\Dc'$ & zbieżny \\
    % & & & & \\
    59  &  3 & & $\theta_{ \nu }$ & $\theta$ \\
    64  &  6 & & $\nu f( \nu^{ n } t )$ & $\nu^{ n } f( \nu t )$ \\
    % & & & & \\
    % & & & & \\
    % & & & & \\
    % & & & & \\
    % & & & & \\
    % & & & & \\
    % & & & & \\
    % & & & & \\
    % & & & & \\
    \hline
  \end{tabular}
\end{center}
\noi \StrWd{21}{10} \\
\Jest opisem niewykończonego \\
\Pow niepełnym opisem pewnego \\
\StrWg{34}{2} \\
\Jest Odpowiednie nachylenie wskazuje \\
\Pow Odpowiednią funkcje przedstawia \\
\StrWd{38}{4} \\
\Jest choć może nie być sprecyzowana jego wartość \\
\Pow choć jego wartość może nie być podana jawnie \\
\StrWd{40}{6} \\
\Jest wartości liczbowych \\
\Pow wartości liczbowej w~punkcie \\
\StrWg{50}{14} \\
\Jest \emph{pokrycie $\Rn$.} \\
\Pow \emph{pokrycie $\Rn$, przy czym średnice wszystkich jego zbiorów,
  są~ograniczone przez tą samą stałą, co~dla rodziny $\mf{C}_{ T }$.} \\
\StrWd{71}{5} \\
\Jest \emph{$M$ i $k$ stałych rzeczywistych} \\
\Pow \emph{$M > 0$ i~$k$ naturalnego} \\
\StrWd{71}{4} \\
\Jest \emph{gdzie $p$} \\
\Pow \emph{gdzie $p = k + 2$, a~$g( t )$} \\

\vspace{\spaceTwo}





% ##############################
\Work{ % Autor i tytuł dzieła
  Armen H.~Zemanian, ,,Teoria dystrybucji i analiza transformat'',
  \cite{ZemanianTeoriaDystrybucji69} }

% ###############


\CenterTB{Uwagi}

\noi \tb{Konkretne strony}

\vspace{\spaceThree}

\start \Str{16} Rozważania o~zamkniętości ze~względu na zbieżność
przestrzeni funkcji próbnych $\Dc$~są w~mojej ocenie trochę
chaotyczne, spróbuję więc je jakoś rozjaśnić. W~istocie chodzi o~to,
że~definiujemy zbieżność ciągu funkcji próbnych
$\set{ \vp_{ \nu }( t ) }$, żądając od~niego dwóch własności: 1)~Dla
każdego $k$, ciąg $\set{ \vp_{ \nu }^{ ( k ) }( t ) }$ jest zbieżny
jednostajnie do jakiejś funkcji
$f_{ k }( t ), \; k = 0, 1, \ld$ \\
2)~Nośniki wszystkich funkcji $\set{ \vp_{ \nu }( t ) }$ zawarte są
w~wspólnym zbiorze zwartym. \\
Zwróćmy uwagę, że~na~razie nie możemy mówić, iż~ciąg
$\set{ \vp_{ \nu }( t ) }$ jest zbieżny do funkcji
$f( t ) = f_{ 0 }( t )$, bo~nie wiemy, czy znajomość funkcji $f( t )$
pozwala nam odtworzyć funkcje $f_{ 1 }( t ), f_{ 2 }( t ), \ld$ W~tym
momencie właściwsze byłoby powiedzenie, że~ciąg
$\set{ \vp_{ \nu }( t ) }$ jest zbieżny do~rodziny funkcji
$f_{ k }( t )$.

Jednak na mocy twierdzeń z~analizy matematycznej, które można znaleźć
np.~w~książce Schwartza, str.~649--652,~\cite{Sch79}, przy tych
warunkach funkcja $f( t )$ jest klasy $\Cinfty$ i~zachodzi
$f^{ ( k ) }( t ) = f_{ k }( t )$. Tym samy pokazaliśmy, że~zbieżność
zdefiniowana wyżej, jest rzeczywiście zbieżnością do~jakiejś funkcji.
Z~drugiego warunku zbieżności ciągu funkcji próbnych wynika,
że~funkcja $f( t )$ również ma~zwarty nośnik i~tym samym należy
do~$\Dc$.

Tym samym możemy stwierdzić, że~jeśli istnieje funkcja zespolona,
do~której ciąg funkcji próbnych jest zbieżny w~podanym wyżej sensie,
to~funkcja graniczna również jest funkcją próbną.

\start \Str{19} Fakt, że~funkcja dana wzorem (4) jest klasy $\Cinfty$
został, chyba przyjęty w~domyśle. Należy jednak~się nad nim zatrzymać
i~dowieść tego faktu. \Dok

\start \Str{21} Warto zatrzymać~się tu na~chwilę, nad faktem który
bardzo słusznie jest stale przypominany w~kontekście teorii
dystrybucji: nie czegoś takiego jak wartość dystrybucji w~punkcie.
Dystrybucja $f$ z~przestrzeni $\Dc'( \Rnc )$ jest określona
na~funkcjach z~$\Dc( \Rnc )$, a~nie na $\Rnc$ i~nawet dla dystrybucji
regularnej wartość $f( t )$ w~konkretnym punkcie nie ma zwykle sensu.
Punkt ma bowiem miarę Lebesgue'a równą~0, więc wartość funkcji która
reprezentuje tę~dystrybucję można w~nim dowolnie zmienić.

Jedynie jeśli w~klasie abstrakcji funkcji reprezentujących daną
dystrybucję istnieje jedna funkcja wyróżniona, to~wartość tej funkcji
jest też wartością dystrybucji w~konkretnym punkcie. Jest tak
np.~jeśli jedna z~tych funkcji jest ciągła.

\start \Str{29} Użycie we wzorze (3) symbolu
\begin{equation}
  \label{eq:Zem-s01-01}
  \Lim_{ \veps \to 0^{ + } } \Int_{ \veps }^{ b } t^{ -3/2 } \vp( t ) \, dt,
\end{equation}
jest trochę mylące. Zdaje~się bowiem sugerować, że~całka
$\int_{ 0 }^{ b } t^{ -3/2 } \vp( t ) \, dt$ istniej i~jest równa tej
granicy, jednak całka ta jest rozbieżna jeśli $\vp( 0 ) \neq 0$.
Poprawniejsze byłoby następujące
rozumowanie. % Może się nie znam ale czy tu też nie powinno być :

Najpierw rozpatrzmy całkę
\begin{equation}
  \label{eq:Zem-s01-02}
  \Int_{ \veps }^{ b } t^{ -3/2 } \vp( t ) \, dt
  = \fr{ 2 \vp( 0 ) }{ \sqrt{ \veps } } - \fr{ 2 \vp( 0 ) }{ \sqrt{ b } }
  + \Int_{ \veps }^{ b } \fr{ \psi( t ) }{ \sqrt{ t } } \, dt.
\end{equation}
Całka ta jest skończona dla każdego $\veps$ i~powyższy wzór pozwala
nam zidentyfikować źródło rozbieżności w~granicy $\veps \to 0^{ + }$.
Dysponując tą wiedzą, możemy zdefiniować dystrybucję
$\Pf \, t^{ -3/2 } \Hevp( t )$ wzorem
\begin{equation}
  \label{eq:Zem-s01-03}
  \lket \Pf \, t^{ -3/2 } \Hevp( t ), \vp( t ) \rket = \Lim_{ \veps \to 0^{ + } }
  \left[ \Int_{ \veps }^{ +\infty } t^{ -3/2 } \vp( t ) \, dt
    - \fr{ 2 \vp( 0 ) }{ \sqrt{ \veps } } \right].
\end{equation}

\start \Str{32} W~tym miejscu można bez trudności, i~nawet byłoby to
bardziej naturalne, wprowadzić pseudofunkcję
$\Pf \fr{ \Hevp( -t )}{ t }$.

\start \Str{33--34} Przyjmijmy najpierw, że~jeżeli dana jest krzywa
$A$, to przez $A( t )$ będziemy oznaczać taką funkcję, że~krzywa
ta~ma~przedstawienie $( t, A( t ) )$. Teraz należy dokonać takiej
zmienny w~linii 2~(od~dołu) na~stronie~34. \\
\Jest Dlatego przesunięta cześć krzywej $B$ \\
\Pow Dlatego pole pod~krzywą $\vp( t ) B( t )$ na~przedziale
$\tau \leq t \leq \veps$

\start \Str{36} W~tym miejscu po raz pierwszy chyba w~książce
pojawia~się termin ,,obszar''. Z~kontekstu wynika, że~należy przez
niego rozumieć dowolny podzbiór $\Rnc$.

\start \Str{41} Obok nazwy \emph{zbiór zer dystrybucji}, proponowałbym
również \emph{zbiór zerowy dystrybucji}.

\start \Str{42} Logiczniej byłoby zaraz po~paragrafie \S 1.6 umieścić
paragraf \S 1.8. Pojęcie zbioru zerowego i~nośnika dystrybucji nie ma
wielkiego sensu 1.8.1 i~wynikających z~niego konsekwencji. Paragraf \S
1.7 \emph{Kilka operacji na~dystrybucjach} najlepiej byłby umieścić
jako \S 1.6.

\start \Str{47--51} Przedstawiony tu dowód twierdzenia 1.8.1, które
jest niezmiernie ważne, zawiera wiele luk, które postaram~się
w~dalszych podpunktach uzupełnić, tu zaś zbiorę potrzebne do~tego
informacje.

Zacznijmy od~tego, że~aby nie obciążać Czytelnika nowymi pojęciami,
nie używa pojęcia zbioru zwartego, zamiast tego mówi o~zbiorach
domkniętych i~ograniczonych w~$\Rn$\footnote{Jak wiadomo z~twierdzenia
  Heinego\dywiz Borela, zbiór w~$\Rn$ jest zwarty \wtw, gdy jest
  domknięty i~ograniczony.}, ponieważ jednak jest to niewygodne, będę
mówił o~zbiorach zwartych. Będziemy jeszcze potrzebowali dwóch
twierdzeń odnośnie tych zbiorów.

\begin{twr}[Walter Rudin, twr. 2.7, str.~45,~\cite{Rud98}]
  \label{twr:Zem-s01-01}
  Niech $X$ będzie lokalnie zwartą przestrzenią Hausdorffa,
  $K$~zbiorem zwartym, $U$~zbiorem otwartym i~$K \subs U$. Istnieje
  zbiór otwarty o~zwartym domknięciu taki,~że
  \begin{equation}
    \label{eq:Zem-s01-04}
    K \subs V \subs \ol{ V } \subs U.
  \end{equation}
\end{twr}
Sens tego twierdzenia jest następujący. Jeśli $K \notin \{ \es, X \}$
i~przestrzeń $X$ jest spójna, to wewnątrz zbioru otwartego $U$, można
powiększyć zbiór zwarty $K$ do~zbioru zwartego $\ol{ V }$, przy czym
$K \neq \ol{ V }$. Jest tak dlatego, że~$V$ jest otwarty,
a~w~przestrzeni Hausdorffa zbiór zwarty jest domknięty. Jeśli więc
przestrzeń jest spójna to nie może zajść równość $K = V$, chyba,
że~$K = \es$ lub~$K = X$.

Jeśli przestrzeń nie jest spójna, to~może~się zdarzyć, że~$K$ jest
maksymalną składową spójną i tym samym jest otwarto-domkniętym
zbiorem. Wtedy jak najbardziej może~się zdarzyć, iż~$K = V$.

\begin{twr}[Laurent Schwartz, str??\cite{Sch79}]
  \label{twr:Zem-s01-02}
  Niech $( X, d )$ będzie przestrzenią metryczną, w~której każdy zbiór
  domknięty i~ograniczony jest zwarty. Jeśli $K$ jest zbiorem zwartym,
  $D$ zbiorem domkniętym i~$K \cup D = \es$, to odległość zbiorów $K$
  i~$D$
  \begin{equation}
    \label{eq:Zem-s01-05}
    d( K, D ) > 0,
  \end{equation}
  do~tego odległość ta przyjmuje w~pewnym punkcie minimum.
\end{twr}
\noi Schwartz formułuje równoważne założenie twierdzenia, że~każda
kula domknięta jest zwarta. Tą~równoważność łatwo pokazać.

\start \Str{48--49} \tb{Lemat 1, uzupełnienie.} Na~podstawie
twierdzenia \ref{twr:Zem-s01-01} istniej zbiór $\Psi$ o~żądanych
własnościach. Teraz na~mocy twierdzenia \ref{twr:Zem-s01-02}
$d( \Theta, \com \Psi ) = d_{ 1 } > 0$, jeśli więc
$\al_{ 1 } < d_{ 1 }$, to~nośnik funkcji
$\ga_{ \al_{ 1 } }( t - \tau )$ zawiera~się $\Psi$ dla~każdego
$t$~z~$\Theta$, tym samym $\vp( t ) = 1$. Gwarantuje to~też,
iż~funkcja podcałkowa jest klasy $\Cinfty$. Dowiedliśmy więc,
że~$\vp( t ) = 1$ na~$\Theta$.


\newpage
\CenterTB{Błędy}
\begin{center}
  \begin{tabular}{|c|c|c|c|c|}
    \hline
    & \multicolumn{2}{c|}{} & & \\
    Strona & \multicolumn{2}{c|}{Wiersz} & Jest
                              & Powinno być \\ \cline{2-3}
    & Od góry & Od dołu & & \\
    \hline
    27 & 7 & & $\del^{ ( 1 ) }\bsym{ ( t } )$ & $\del^{ ( 1 ) }( t )$ \\
    27 & & 3 & $\vp^{ ( k \bsym{ ) } }\bsym{ ( t ) } $
           & $\vp^{ ( k ) }( t )$ \\
    28 & & 17 & dążą & jednocześnie dążą \\
    28 & & 4 & $\del^{ ( 2 ) }( 0 )$ & $\vp^{ ( 2 ) }( 0 )$ \\
    29 & & 8 & $t = 0$ & $t \neq 0$ \\
    31 & 6 & & $\vp( )$ & $\vp( t )$ \\
    31 & 8 & & $\Pf \, | t |^{ \be } \; 1_{ + }( t )$
           & $\Pf \, | t |^{ \be } \; 1_{ + }( -t )$ \\
    33 & & 1 & skończoną & nieskończoną \\
    34 & & 1 & $\vp( 0 ) \log\veps$ & $\vp( 0 ) \log\veps.$ \\
    35 & 1 & & tj.~$\lket B, \vp \rket$. Wartość & Wartość \\
    36 & & 7 & $\pr t_{ 2 }^{ \: k_{ 2 } } \ld \pr t_{ n }^{ \: k_{ n } }$
           & $\pr t_{ 2 }^{ k_{ 2 } } \ld \pr t_{ n }^{ k_{ n } }$ \\
    37 & & 16 & \emph{mamy $\vp_{ \al }( t )$ równe}
           & \emph{$\vp_{ \al }( t )$ jest równa} \\
    37 & & 1 & wyboru układu & układu \\
    38 & 4 & & $\{ 0, 0, \ld, 0 ) \}$ & $\{ 0, 0, \ld, 0 \}$) \\
    41 & & 2 & dystrybucji & dystrybucji regularnej \\
    45 & & 10 & $\bsym{f}$ & $f$ \\
    50 & & 3 & $C$ & $\mf{C}$ \\
    50 & 13 & & \emph{jedną} & \emph{jedną z~nich} \\
    55 & 9 & & zbieżny w $\Dc'$ & zbieżny \\
    71 & & 5 & $M$ i $k$ stałych & $M$ rzeczywistego i $k$ naturalnego \\
    71 & & 4 & gdzie $p$ & gdzie $p = k - 1$, a $g( t )$\\
    71 & & 1 & $\leq M | \nu |^{ -2 }$ & $\leq M | \nu |^{ -1 }$\\
    % & & & & \\
    \hline
  \end{tabular}
\end{center}
\noi
\StrWd{21}{10} \\
\Jest opisem niewykończonego \\
\Pow niepełnym opisem pewnego \\
\StrWg{34}{2} \\
\Jest Odpowiednie nachylenie wskazuje \\
\Pow Odpowiednią funkcje przedstawia \\
\StrWd{38}{4} \\
\Jest choć może nie być sprecyzowana jego wartość \\
\Pow choć jego wartość może nie być podana jawnie \\
\StrWd{40}{6} \\
\Jest wartości liczbowych \\
\Pow wartości liczbowej w~punkcie \\

\vspace{\spaceTwo}





% ######################################
\section{Algebry operatorów i~algebry topologiczne}

\vspace{\spaceTwo}
% ######################################


% ####################
\Work{ % Autorzy i tytuł dzieła
  Ola Bratteli, Derek W.~Robinson \\
  ,,Operator Algebras and~Quantum Statistical Mechanics 1. \\
  $C^{ * }$- and $W^{ * }$-Algebras, Symmetry Groups. Decoposition~of
  State'', \cite{BratteliRobinsonOperatorAlgebrasTomI2002} }


\CenterTB{Uwagi}

\start \Str{19} Autorzy zapomnieli o~jednym warunku, który jest
potrzebny w~definicji algebry
\begin{equation}
  \label{eq:BR-s01-01}
  ( A + B ) C = A C + B C.
\end{equation}

\vspace{\spaceFour}


\start \Str{27} Transformacja podana dla $\lambda^{ n } - A^{ n }$ i
wynikające z niej konsekwencje, nie są wystarczająco omówione.

\vspace{\spaceFour}


\CenterTB{Błędy}
\begin{center}
  \begin{tabular}{|c|c|c|c|c|}
    \hline
    & \multicolumn{2}{c|}{} & & \\
    Strona & \multicolumn{2}{c|}{Wiersz} & Jest
                              & Powinno być \\ \cline{2-3}
    & Od góry & Od dołu & & \\
    \hline
    21  & & 10 & $ff^{ * }$ & $f^{ * }\! f$ \\
    22  & 12 & & $( \alpha, A^{ * } )$ & $( \alpha, A )^{ * }$ \\
    % & & & & \\
    % & & & & \\
    % & & & & \\
    % & & & & \\
    % & & & & \\
    % & & & & \\
    % & & & & \\
    % & & & & \\
    % & & & & \\
    % & & & & \\
    % & & & & \\
    % & & & & \\
    \hline
  \end{tabular}
\end{center}
% \noi
% \StrWd{21}{10} \\
% \Jest opisem niewykończonego \\
% \Pow niepełnym opisem pewnego \\
 Str. 31. \ldots the resolvent
  $R( \lambda ) = ( A - \lambda I )$\ldots





% \begin{center}
%   Masamichi Takesaki\\
%   ,,Theory of Operator Algebras. Volume 1.'', \cite{MTTOAI}.
% \end{center}


% % Uwagi:
% % \begin{itemize}
% % \item
% % \end{itemize}


% Błędy:\\
% \begin{tabular}{|c|c|c|c|c|}
%   \hline
%   & \multicolumn{2}{c|}{} & & \\
%   Strona & \multicolumn{2}{c|}{Wiersz}& Jest & Powinno być \\ \cline{2-3}
%   & Od góry & Od dołu &  &  \\ \hline
%   & & & & \\
%   5 & & 6 & $x_{ 0 }^{ -1 } + \sum\limits_{ n = 0 }^{ \infty } [ x_{ 0 }^{ -1 } ( x_{ 0 } - x ) ]^{ n } x_{ 0 }^{ -1 }$ & $\sum\limits_{ n = 0 }^{ \infty } [ x_{ 0 }^{ -1 } ( x_{ 0 } - x ) ]^{ n } x_{ 0 }^{ -1 }$ \\
%   7 & & 9 & $[ ( 1 / \uplambda ) x - 1 ]^{ -1 }$ & $[ 1 - ( 1 / \uplambda ) x ]^{ -1 }$ \\
%   7 & & 8 & $\frac{ 1 }{ \uplambda } \big( \frac{ 1 }{ \uplambda } - x \big)^{ -1 }$ & $-\frac{ 1 }{ \uplambda } \big( 1 - \frac{ 1 }{ \uplambda } x \big)^{ -1 }$ \\
%   7 & & 3 & $\sum\limits_{ n = 0 }^{ \infty } ( 1 / \uplambda^{ n + 1 } ) x^{ n }$ & $-\sum\limits_{ n = 0 }^{ \infty } ( 1 / \uplambda^{ n + 1 } ) x^{ n }$ \\
%   8 & 3 & & $( \uplambda - \uplambda_{ 0 } )^{ n } f( \uplambda_{ 0 } )^{ n + 1 }$ & $-( \uplambda_{ 0 } - \uplambda )^{ n } f( \uplambda_{ 0 } )^{ n + 1 }$ \\
%   & & & & \\ \hline
% \end{tabular}



% #####################################################################
% #####################################################################
% Bibliografia
\bibliographystyle{plalpha} \bibliography{LibMathInfo}{}


% ############################

% Koniec dokumentu
\end{document}
